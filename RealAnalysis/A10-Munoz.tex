% <-- a percent symbol indicates a comment which does not affect the output of LaTeX
% you can leave the preamble alone, from here ...
\documentclass[12pt]{article}

\usepackage{amssymb,amsmath,amsthm}
\usepackage[top=1in, bottom=1in, left=1.25in, right=1.25in]{geometry}
\usepackage{enumitem,palatino}
\usepackage[final]{graphicx}
\usepackage[colorlinks=true,citecolor=blue,linkcolor=red,urlcolor=blue]{hyperref}

\newtheorem{problem}{Problem}
% ... to here

% shortcuts for blackboard bold number sets (reals, integers, etc.)
\newcommand{\II}{\ensuremath{\mathbb I}}
\newcommand{\NN}{\ensuremath{\mathbb N}}
\newcommand{\QQ}{\ensuremath{\mathbb Q}}
\newcommand{\RR}{\ensuremath{\mathbb R}}
\newcommand{\ZZ}{\ensuremath{\mathbb Z}}

\newcommand{\eps}{\ensuremath{\epsilon}}
\newcommand{\ds}{\displaystyle}

% feel free to add more shortcuts here


\begin{document}
% replace with your name, but otherwise leave this header alone, from here ...
\small
\noindent \textsc{Math 401: Homework Assignment 10} \hfill Christopher Munoz

\normalsize
\bigskip
% ... to here

\setcounter{problem}{67}

\begin{problem} % Problem 68
The function $f(x) = 1/x^2$ is uniformly continuous on $(1,2)$, but it is not uniformly continuous on $(0,1)$.
\end{problem}

% PLEASE READ: The result on (1,2) can be proven directly from the definition of uniform continuity.  (You cannot use Theorem 4.4.7 since (1,2) is not compact, right?)  To show the failure of uniform continuity on (0,1), perhaps use a suitable Theorem.

\begin{proof}
  \textbf{Part 1: Uniform Continuity on (1,2): }To show that the function $f(x) = 1/x^2$ is uniformly continuous on the set $S_1 = (1,2)$ we must show that for every $\eps > 0$ there exists $\delta > 0$ such that for all $x,y \in S_1$:
  $$|x-y| < \delta \implies |f(x) - f(y)| < \eps$$
Let $\eps > 0$, Observe that
$$|f(x) - f(y)| = |\frac{1}{x^2} - \frac{1}{y^2}| = |\frac{y^2 - x^2}{x^2y^2}|= \frac{|x+y||x-y|}{x^2y^2}$$
Since $x,y \in (1,2)$, it follows that $x^2 > 1$ and $y^2 > 1$, so $x^2y^2 > 1$. Furthermore $|x + y| < 2 + 2 = 4$.
Therefore:
$$|f(x) - f(y)| = \frac{|x-y||x+y|}{x^2y^2} < \frac{|x-y|*4}{1}  = 4|x-y|$$
Choose $\delta = \frac{\eps}{4}$. Then whenever $|x - y| < \delta$, we have:
$$|f(x) - f(y)| < 4|x-y| < 4 * \frac{\eps}{4} = \eps$$
Since $\delta$ only depends on $\eps$, $f$ is uniformly continuous on $(1,2)$
\end{proof}
\begin{proof}
  \textbf{Part 2: Not Uniformly continuous on (0,1): }
  Suppose for the sake of contradiction that $f$ is uniformly continuous on the set $S_2 = (0,1)$. Choose $\eps = 1$. Then there exists a $\delta > 0$ such that for all $x,y \in S_2$ we have 
  $$|x-y| < \delta \implies |\frac{1}{x^2} - \frac{1}{y^2}| < 1$$
  Choose $x \in S_2$ with $x < \delta$ and choose $y = \frac{x}{2}$. Note that $y \in S_2$ because $x \in S_2$ so $y = \frac{x}{2} < \frac{1}{2} < 1$. Then
  $$|x-y| = |x - \frac{x}{2}| = |\frac{x}{2}| = \frac{x}{2} < \frac{\delta}{2} < \delta$$
  So $|x-y| < \delta$ holds, it follows that
  $$|\frac{1}{x^2} - \frac{1}{y^2}| = |\frac{1}{x^2} - \frac{1}{(x/2)^2}|= |\frac{1}{x^2} - \frac{1}{(x^2/4)}| = |\frac{1}{x^2} - \frac{4}{x^2}| = |\frac{-3}{x^2}| = \frac{3}{x^2} > 1$$
  So $|\frac{1}{x^2}-\frac{1}{y^2}| > 1$. A contradiction, thus $f$ is not uniformly continuous on $S_2 = (0,1)$.
\end{proof}


\begin{problem} % Problem 69
We say that a function $f:A\to \RR$ is \emph{Lipschitz} if there exists $M>0$ so that
	$$\frac{|f(x)-f(y)|}{|x-y|} \le M$$
for all $x,y\in A$.  If $f$ is Lipschitz then $f$ is uniformly continuous.
\end{problem}

% PLEASE READ: It is recommended to prove this directly from the definition of uniform continuity.

\begin{proof}
Since $f$ is Lipschitz, there exists $M > 0$ such that $|f(x) - f(y)| \leq M|x - y|$ for all $x,y \in A$.
Let $\delta = \frac{\eps}{M}$ and let $x,y \in A$ be any two points such that $|x - y| < \delta$. Then it follows that
$$|f(x) - f(y)| \leq M|x - y| \implies |f(x)-f(y)| < M*\delta $$
Since $|x-y| < \delta$. Observe that
$$|f(x)-f(y)| < M*\delta = M*\frac{\eps}{M} = \eps$$
Thus, whenever $|x - y| < \delta$, we have $|f(x)-f(y)| < \eps$. Note that $M$ is a constant and $\delta$ only depends on $\eps$. Therefore proving that if $f$ is Lipschitz, then $f$ is uniformly continuous.
\end{proof}


\begin{problem} % Problem 70
Let $f$ and $g$ be functions defined on an interval $A$.  Assume both are differentiable at some point $c\in A$, and suppose $k\in \RR$.  Then
\renewcommand{\labelenumi}{(\roman{enumi})}
\begin{enumerate}
\item $(f+g)'(c) = f'(c) + g'(c)$
\item $(kf)'(c) = k f'(c)$
\end{enumerate}
\end{problem}

% PLEASE READ:  This theorem is the first two parts of Theorem 5.2.4, the Algebraic Differentiability Theorem.  I proved the product rule in class, and the proofs here are easier.

\begin{proof}
  \textbf{(i)} Since $f$ and $g$ are differentiable at some point $c \in A$, then $f'(c) = \lim_{h \to 0} \frac{f(c+h) - f(c)}{h}$ and $g'(c) = \lim_{h \to 0} \frac{g(c+h) - g(c)}{h}$ by definition. We want to show that $(f+g)'(c) = f'(c) + g'(c)$. Observe that
  \begin{align*}
    (f+g)'(c) &= \lim_{h \to 0} \frac{(f+g)(c + h) - (f+g)(c)}{h} \\
              &= \lim_{h \to 0} \frac{[f(c+h) + g(c+h)] - [f(c) + g(c)]}{h} \\
              &= \lim_{h \to 0} \frac{[f(c+h) - f(c)] + [g(c+h)-g(c)]}{h} \\
              &= \lim_{h \to 0} \frac{f(c+h) - f(c)}{h} + \lim_{h \to 0} \frac{g(c+h) - g(c)}{h} \\
              &= f'(c) + g'(c)
  \end{align*}
  Thus $(f+g)'(c) = f'(c) + g'(c)$.
\end{proof}
\begin{proof}
  \textbf{(ii)} Note that $k \in \RR$ is constant, then
  \begin{align*}
    (kf)'(c) &= \lim_{h \to 0} \frac{(kf)(c+h) - (kf)(c)}{h} \\
             &= \lim_{h \to 0} \frac{kf(c+h) - kf(c)}{h} \\
             &= \lim_{h \to 0} \frac{k[f(c+h) - f(c)]}{h} \\
             &= \lim_{h \to 0} k*\frac{f(c+h) - f(c)}{h} \\
             &= k * \lim_{h \to 0} \frac{f(c+h) - f(c)}{h} \\
             &= kf'(c)
  \end{align*}
  Thus $(kf)'(c) = k f'(c)$.
\end{proof}

\begin{problem} % Problem 71
Let $h(x) = 1/x$ and $\ell(x) = 1/x^2$.  For $c\ne 0$, we have
	$$h'(c) = - \frac{1}{c^2}, \qquad \ell'(c) = - \frac{2}{c^3}$$ 
\end{problem}

% PLEASE READ: Prove these formulas directly from the definition of the derivative, namely Definition 5.2.1.

\begin{proof}
  \textbf{For h(x) implies h'(c): } Suppose $h(x) = 1/x$, then by definition of derivative:
  \begin{align*}
    h'(c) &= \lim_{h \to 0}\frac{h(c + h) - h(c)}{h} \\
          &= \lim_{h \to 0} \frac{ \frac{1}{c + h} - \frac{1}{c} }{h} \\
          &= \lim_{h \to 0} \frac{ \frac{c-c-h}{c(c+h)} }{h} \\
          &= \lim_{h \to 0} \frac{ \frac{-h}{c(c+h)} }{h} \\
          &= \lim_{h \to 0} \frac{-h}{h*c(c+h)} \\
          &= \frac{-1}{1*c(c+0)} \\
          &= \frac{-1}{c^2}
  \end{align*}
\end{proof}
\begin{proof}
  \textbf{For $\ell$(x) implies $\ell$'(c): } Suppose $\ell(x) = 1/x^2$, then by definition of derivative:
  \begin{align*}
    \ell'(c) &= \lim_{h \to 0}\frac{\ell(c + h) - \ell(c)}{h} \\
           &= \lim_{h \to 0} \frac{ \frac{1}{(c + h)^2} - \frac{1}{c^2} }{h} \\
           &= \lim_{h \to 0} \frac{ \frac{c^2 - (c+h)^2}{c^2(c+h)^2} }{h} \\
           &= \lim_{h \to 0} \frac{ \frac{c^2 - (c^2 + 2ch + h^2)}{c^2(c+h)^2} }{h} \\
           &= \lim_{h \to 0} \frac{ \frac{-2ch - h^2}{c^2(c+h)^2} }{h} \\
           &= \lim_{h \to 0} \frac{-2ch - h^2}{h \cdot c^2(c+h)^2} \\
           &= \lim_{h \to 0} \frac{h(-2c - h)}{h \cdot c^2(c+h)^2} \\
           &= \lim_{h \to 0} \frac{-2c - h}{c^2(c+h)^2} \\
           &= \frac{-2c - 0}{c^2(c+0)^2} \\
           &= \frac{-2c}{c^4} \\
           &= \frac{-2}{c^3}
  \end{align*}
\end{proof}


\begin{problem} % Problem 72
Let $f$ and $g$ be functions defined on an interval $A$.  Assume both are differentiable at some point $c\in A$, and suppose $g(c)\ne 0$.  Then
    $$\left(\frac{f}{g}\right)'(c) = \frac{f'(c) g(c) - f(c) g'(c)}{g(c)^2}.$$
\end{problem}

% PLEASE READ: This is the quotient rule, which is Theorem 5.2.4 (iv).  Please prove it using the following tools:
% 1) the Chain Rule, Theorem 5.2.5.
% 2) the result of Problem 71 for h(x)=1/x,
% 3) the Product Rule, Theorem 5.2.4 (iii),

\begin{proof}
  Note that $\frac{f}{g} = f \cdot \frac{1}{g}$. Let $h(x) = \frac{1}{x}$, so $\frac{f}{g} = f \cdot (h \circ g)$. By the Product Rule (Theorem 5.2.4(iii)):
  \begin{align*}
    \left(\frac{f}{g}\right)'(c) &= (f \cdot (h \circ g))'(c) \\
                                  &= f'(c) \cdot (h \circ g)(c) + f(c) \cdot (h \circ g)'(c)
  \end{align*}
  By the Chain Rule, $(h \circ g)'(c) = h'(g(c)) \cdot g'(c)$. From Problem 71, $h'(x) = -\frac{1}{x^2}$, so $h'(g(c)) = -\frac{1}{g(c)^2}$. Thus:
  \begin{align*}
    \left(\frac{f}{g}\right)'(c) &= f'(c) \cdot \frac{1}{g(c)} + f(c) \cdot \left(-\frac{1}{g(c)^2}\right) \cdot g'(c) \\
                                  &= \frac{f'(c)}{g(c)} - \frac{f(c)g'(c)}{g(c)^2} \\
                                  &= \frac{f'(c)g(c) - f(c)g'(c)}{g(c)^2}
  \end{align*}
\end{proof}


\begin{problem} % Problem 73
For $a\in\RR$, let
	$$f_a(x) = \begin{cases} x^a, & \text{if } x > 0 \\
	                         0, & \text{if } x \le 0 \end{cases}$$

\renewcommand{\labelenumi}{(\alph{enumi})}
\begin{enumerate}
\item For which values of $a$ is $f_a(x)$ continuous at $x=0$?

For continuity at $x = 0$, we need $\lim_{x \to 0} f_a(x) = f_a(0) = 0$. For $x < 0$, $f_a(x) = 0$, so $\lim_{x \to 0^-} f_a(x) = 0$. For $x > 0$, $f_a(x) = x^a$, so we need $\lim_{x \to 0^+} x^a = 0$.

If $a > 0$: $\lim_{x \to 0^+} x^a = 0$.

If $a = 0$: $\lim_{x \to 0^+} x^0 = 1 \ne 0$.

If $a < 0$: $\lim_{x \to 0^+} x^a = \lim_{x \to 0^+} \frac{1}{x^{|a|}} = +\infty$.

Therefore, $f_a(x)$ is continuous at $x = 0$ if and only if $a > 0$.
\item What is the derivative $f_a'(x)$, and what is its domain?  For which values of $a$ is $f_a(x)$ differentiable at $x=0$?  When is the derivative function $f_a'(x)$ continuous?

For $x \ne 0$: If $x > 0$, then $f_a(x) = x^a$ so $f_a'(x) = ax^{a-1}$. If $x < 0$, then $f_a(x) = 0$ so $f_a'(x) = 0$.

For $x = 0$: By definition, $f_a'(0) = \lim_{h \to 0} \frac{f_a(h) - f_a(0)}{h} = \lim_{h \to 0} \frac{f_a(h)}{h}$.

From the left: $\lim_{h \to 0^-} \frac{0}{h} = 0$.

From the right: $\lim_{h \to 0^+} \frac{h^a}{h} = \lim_{h \to 0^+} h^{a-1}$.

If $a > 1$: $\lim_{h \to 0^+} h^{a-1} = 0$, so $f_a'(0) = 0$ exists.

If $a = 1$: $\lim_{h \to 0^+} h^0 = 1 \ne 0$, so the derivative does not exist.

If $0 < a < 1$: $\lim_{h \to 0^+} h^{a-1} = +\infty$, so the derivative does not exist.

If $a \le 0$: $f_a$ is not continuous at $0$ (from part (a)), so not differentiable.

Therefore: $f_a$ is differentiable at $x = 0$ if and only if $a > 1$.

The derivative function is:
$$f_a'(x) = \begin{cases} ax^{a-1}, & \text{if } x > 0 \\
                         0, & \text{if } x < 0 \\
                         0, & \text{if } x = 0 \text{ and } a > 1 \end{cases}$$

Domain of $f_a'$: If $a > 1$, the domain is $\RR$. If $a \le 1$, the domain is $(-\infty, 0) \cup (0, \infty)$.

For continuity of $f_a'$: On $(-\infty, 0)$ and on $(0, \infty)$, the derivative $f_a'$ is continuous for all $a$. At $x = 0$ (when $a > 1$):

$\lim_{x \to 0^-} f_a'(x) = 0$ and $\lim_{x \to 0^+} f_a'(x) = \lim_{x \to 0^+} ax^{a-1}$.

If $a > 1$: $a - 1 > 0$, so $\lim_{x \to 0^+} ax^{a-1} = 0 = f_a'(0)$. Thus $f_a'$ is continuous at $0$.

Therefore, $f_a'$ is continuous on $\RR$ if and only if $a > 1$.
\end{enumerate}
\end{problem}

\end{document}
