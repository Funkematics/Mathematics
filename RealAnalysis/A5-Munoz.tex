% <-- a percent symbol indicates a comment which does not affect the output of LaTeX
% you can leave the preamble alone, from here ...
\documentclass[12pt]{article}

\usepackage{amssymb,amsmath,amsthm}
\usepackage[top=1in, bottom=1in, left=1.25in, right=1.25in]{geometry}
\usepackage{enumerate,palatino}
\usepackage[final]{graphicx}
\usepackage[colorlinks=true,citecolor=blue,linkcolor=red,urlcolor=blue]{hyperref}

\newtheorem{problem}{Problem}
% ... to here

% shortcuts for blackboard bold number sets (reals, integers, etc.)
\newcommand{\II}{\ensuremath{\mathbb I}}
\newcommand{\NN}{\ensuremath{\mathbb N}}
\newcommand{\QQ}{\ensuremath{\mathbb Q}}
\newcommand{\RR}{\ensuremath{\mathbb R}}
\newcommand{\ZZ}{\ensuremath{\mathbb Z}}

\newcommand{\eps}{\ensuremath{\epsilon}}

% feel free to add more shortcuts here


\begin{document}
% replace with your name, but otherwise leave this header alone, from here ...
\small
\noindent \textsc{Math 401: Homework Assignment 5} \hfill Christopher Munoz

\normalsize
\bigskip
% ... to here

\setcounter{problem}{28}

\begin{problem} % Problem 29
Suppose $(x_n)_{n=1}^\infty$ converges.  Let $k \in \NN$.  The new sequence $(x_{n+k})_{n=1}^\infty$ also converges, and to the same limit.
\end{problem}

% PLEASE READ:  You will have to prove this directly from the definition of the convergence of sequences.

\begin{proof}
	Let $\eps > 0$. 
	Since the sequence $(x_n)_{n=1}^\infty$ converges to $L$, there exists $N \in \NN$ such that for all $n > N$, $|x_n - L| < \eps$. Now choose $M = N$ for our shifted sequence. Then for all $n > M$, we have $n+k > N$(since $k \geq 1$), so $|x_{n+k} - L| < \eps$. Therefore $(x_{n+k})$ converges to $L$.
\end{proof}

\begin{problem} % Problem 30
Give an example of each of the following, or state that such a request is impossible.  In the latter case, identify specific theorem(s) that justify your statement.

\renewcommand{\labelenumi}{(\alph{enumi})}
\begin{enumerate}
\item sequences $(x_n)$ and $(y_n)$, which both diverge, where the sum $(x_n+y_n)$ converges

	We take the alternating harmonic series $\sum_{k=1}^\infty \frac{(-1)^{n+1}}{n}$ which famously converges to $\ln(2)$ and define $x_n$ as the sequence of positive terms and $y_n$ as the sequence of negative terms.
	\begin{align*}
		x_n = 
	\begin{cases}
		\frac{1}{n} &\text{if } n \text{ is odd} \\
		0 & \text{if } n \text{ is even}
	\end{cases} && y_n =
	\begin{cases}
		\frac{-1}{n} &\text{if } n \text{ is even} \\
		0 & \text{if } n \text{ is odd}
	\end{cases}
\end{align*}
These two sequence of partial sums converge when combined and each diverge when split this way.
\item a convergent sequence $(x_n)$, and a divergent sequence $(y_n)$, where $(x_n+y_n)$ converges

This is impossible, a consequence of the Algebraic Limit Theorem. If we suppose $(x_n)$ converges and $(x_n + y_n)$ converges, then $y_n = (x_n + y_n) - x_n$ must also converge. This leads to a consequence since we assumed $(y_n)$ does not converege.
\item a convergent sequence $(b_n)$, with $b_n\ne 0$ for all $n$, such that $(1/b_n)$ diverges

	This one is also impossible as a consequence of the Algebraic Limit Theorem. If we suppose $(b_n)$ converges to $b_n$ and $b_n \neq 0$ and choose $(a_n)$ to converge to $1$, then according to the Algebraic Limit Theorem $\frac{(a_n)}{(b_n)} = \frac{1}{(b_n)}$ must also converge.

\item sequences $(x_n)$ and $(y_n)$, where $(x_n y_n)$ and $(x_n)$ converge but $(y_n)$ does not
	
	If we let $(x_n) = \frac{1}{n^3}$ which converges and $(y_n) = n$ which diverges, we get $(x_ny_n) = \frac{1}{n^2}$ which converges. 

\end{enumerate}
\end{problem}


\begin{problem} % Problem 31
If $a\ge 0$ and $b\ge 0$ then \, $\displaystyle \sqrt{ab} \le \frac{1}{2}\left(a+b\right)$.
\end{problem}

% PLEASE READ:  This fact is called the arithmetic-geometric mean inequality.  The left side is the geometric mean (why?) and the right side is the usual mean (i.e. average).  As a hint on the proof, square the sides of what you want to prove, and play around.  The logical starting point of your proof will be the true fact that the square of a certain real number is nonnegative.

\begin{proof}
	Suppose $a \geq 0$ and $b \geq 0$, then it follows that $(a-b)^2 \geq 0$. Expanding this gives
	$$
		a^2 -2ab + b^2 \geq 0
	$$
	add $2ab$ to both sides
	$$
	a^2 + b^2 \geq 2ab
	$$
add another $2ab$ to both sides
$$
a^2 + 2ab + b^2 \geq 4ab
$$
Since $a+b$ and $\sqrt{ab}$ are non-negative, we can take square roots and get
$$ a + b \geq 2\sqrt{ab}$$
Dividing by 2:
$$ \frac{1}{2}(a + b) \geq \sqrt{ab} $$
Therefore  $\displaystyle \sqrt{ab} \le \frac{1}{2}\left(a+b\right)$.
\end{proof}


\begin{problem} % Problem 32
Consider the real sequence generated by setting $x_1=2$ and then
	$$x_{n+1} = \frac{1}{2}\left(x_n + \frac{2}{x_n}\right).$$

% PLEASE READ:  In (a) you may use the arithmetic-geometric mean inequality from Problem 31.  In (b) you will want to use (a) and a major theorem in section 2.4.  That is, be careful to establish that the limit exists before showing it is $\sqrt{2}$.  A proof by induction of monotocity is appropriate.  You will also use Problem 29.
%
% What you are proving here could be phrased as "Newton's method applied to solve x^2-2=0, starting with the guess x_1=2, converges to the root \sqrt{2}".  (Explain this, on your own time, from the general Newton's method formula.)  In fact, the convergence is stunningly fast, as you can see (on your own time) by using a calculator to get the first 6 terms.  However, your proof if (b) says nothing about *speed* of convergence.

\renewcommand{\labelenumi}{(\alph{enumi})}
\begin{enumerate}
\item The sequence $(x_n)$ is bounded below by $\sqrt{2}$.

\begin{proof}
We will prove by induction that $x_n \geq \sqrt{2}$ for all $n \geq 1$.

\textbf{Base case:} For $n = 1$, we have $x_1 = 2 > \sqrt{2}$ since $2 > 1.414...$

\textbf{Inductive step:} Assume $x_n \geq \sqrt{2}$ for some $n \geq 1$. We must show $x_{n+1} \geq \sqrt{2}$.

By the recurrence relation, $x_{n+1} = \frac{1}{2}\left(x_n + \frac{2}{x_n}\right)$. Since $x_n \geq \sqrt{2} > 0$ by the inductive hypothesis, both $x_n$ and $\frac{2}{x_n}$ are positive. By the Arithmetic-Geometric Mean Inequality (Problem 31) with $a = x_n$ and $b = \frac{2}{x_n}$:
$$x_{n+1} = \frac{1}{2}\left(x_n + \frac{2}{x_n}\right) \geq \sqrt{x_n \cdot \frac{2}{x_n}} = \sqrt{2}$$

Therefore $x_{n+1} \geq \sqrt{2}$. By induction, $x_n \geq \sqrt{2}$ for all $n \geq 1$.
\end{proof}

\item $\lim_{n\to\infty} x_n = \sqrt{2}$.

\begin{proof}
We will show that $(x_n)$ is monotone decreasing and bounded below, which by the Monotone Convergence Theorem implies the limit exists.

We first show $(x_n)$ is decreasing for $n \geq 1$. We need $x_{n+1} \leq x_n$, which is equivalent to:
$$\frac{1}{2}\left(x_n + \frac{2}{x_n}\right) \leq x_n$$

Multiplying both sides by $2x_n > 0$:
$$x_n^2 + 2 \leq 2x_n^2$$
$$2 \leq x_n^2$$

This holds since $x_n \geq \sqrt{2}$ by part (a), so $x_n^2 \geq 2$. Thus $(x_n)$ is monotone decreasing.

We know from part (a), $(x_n)$ is bounded below by $\sqrt{2}$.

By the Monotone Convergence Theorem, $(x_n)$ converges. Let $L = \lim_{n\to\infty} x_n$.

We now take the limit of both sides of the recurrence relation:
$$\lim_{n\to\infty} x_{n+1} = \lim_{n\to\infty} \frac{1}{2}\left(x_n + \frac{2}{x_n}\right)$$

We proved in problem 29 that, $\lim_{n\to\infty} x_{n+1} = L$. By the Algebraic Limit Theorem:
$$L = \frac{1}{2}\left(L + \frac{2}{L}\right)$$

Multiplying by $2L$ (note $L \geq \sqrt{2} > 0$):
$$2L^2 = L^2 + 2$$
$$L^2 = 2$$
$$L = \pm\sqrt{2}$$

Since $x_n \geq \sqrt{2} > 0$ for all $n$, we have $L > 0$, so $L = \sqrt{2}$.
\end{proof}

\end{enumerate}
\end{problem}


\begin{problem} % Problem 33
The sequence $\displaystyle \sqrt{2}, \sqrt{2+\sqrt{2}}, \sqrt{2 + \sqrt{2+\sqrt{2}}}, \dots$ converges to $X$.
\end{problem}

% PLEASE READ:  I will do a problem like this in lecture.  Step 1 is to write down a recurrence: x_1=\sqrt{2} and x_{n+1}=f(x_n).  That is, establish how each term is computed from the last and thereby find f(x).  Then show that the sequence is increasing and bounded.  Then establish that the limit exists.  Then find its limit $X$; of course, fill in a particular value for $X$.  Both in Problem 32 and this problem, it is just fine to start by computing some terms in the sequence, but that does not prove anything.

\begin{proof}
Let $x_1 = \sqrt{2}$ and $x_{n+1} = \sqrt{2 + x_n}$ for $n \geq 1$. This gives the recurrence relation for our sequence.
\newline \newline
We will first show that $(x_n)$ is bounded above. We claim $x_n < 2$ for all $n$. 
\newline \newline
For $n = 1$: $x_1 = \sqrt{2} < 2$. 
\newline \newline
Assume $x_n < 2$. Then $x_{n+1} = \sqrt{2 + x_n} < \sqrt{2 + 2} = \sqrt{4} = 2$. By induction, $x_n < 2$ for all $n$.
We next show that $(x_n)$ is increasing. We need $x_{n+1} > x_n$, i.e., $\sqrt{2 + x_n} > x_n$.
\newline \newline
Squaring both sides (valid since both are positive):
$$2 + x_n > x_n^2$$
$$x_n^2 - x_n - 2 < 0$$
$$(x_n - 2)(x_n + 1) < 0$$
Since $x_n > 0$, we have $x_n + 1 > 0$, so we need $x_n - 2 < 0$, i.e., $x_n < 2$. This holds by Step 1, so $(x_n)$ is increasing.
\newline \newline
By the Monotone Convergence Theorem, $(x_n)$ converges. Let $X = \lim_{n\to\infty} x_n$.
Taking the limit of $x_{n+1} = \sqrt{2 + x_n}$:
$$X = \sqrt{2 + X}$$
Squaring both sides:
$$X^2 = 2 + X$$
$$X^2 - X - 2 = 0$$
$$(X-2)(X+1) = 0$$
So $X = 2$ or $X = -1$. Since $x_n > 0$ for all $n$, we have $X > 0$, thus $X = 2$.
\end{proof}


\begin{problem} % Problem 34
For each series, find an explicit formula for the partial sums, and determine if the series converges.

\renewcommand{\labelenumi}{(\alph{enumi})}
\begin{enumerate}
\item $\displaystyle \sum_{n=1}^\infty \frac{1}{2^n}$

This is a geometric series with $r = \frac{1}{2}$. The partial sums are:
$$S_N = \sum_{n=1}^N \frac{1}{2^n} = \frac{1/2(1 - (1/2)^N)}{1 - 1/2} = 1 - \frac{1}{2^N}$$

As $N \to \infty$, $\frac{1}{2^N} \to 0$, so $\lim_{N\to\infty} S_N = 1$. 

The series converges to $1$.


\item $\displaystyle \sum_{n=1}^\infty \frac{1}{n(n+1)}$

Using partial fractions: $\frac{1}{n(n+1)} = \frac{1}{n} - \frac{1}{n+1}$.

The partial sums are:
\begin{align*}
S_N &= \sum_{n=1}^N \frac{1}{n(n+1)} = \sum_{n=1}^N \left(\frac{1}{n} - \frac{1}{n+1}\right) \\
&= \left(\frac{1}{1} - \frac{1}{2}\right) + \left(\frac{1}{2} - \frac{1}{3}\right) + \cdots + \left(\frac{1}{N} - \frac{1}{N+1}\right) \\
&= 1 - \frac{1}{N+1}
\end{align*}

This is a telescoping series. As $N \to \infty$, $\frac{1}{N+1} \to 0$, so $\lim_{N\to\infty} S_N = 1$.

The series converges to $1$.

\item $\displaystyle \sum_{n=1}^\infty \log\left(\frac{n+1}{n}\right)$  %  Note log(x)=ln(x) is the natural logarithm.

Using logarithm properties: $\log\left(\frac{n+1}{n}\right) = \log(n+1) - \log(n)$.(email says its halal)

The partial sums are:
\begin{align*}
S_N &= \sum_{n=1}^N \log\left(\frac{n+1}{n}\right) = \sum_{n=1}^N (\log(n+1) - \log(n)) \\
&= (\log(2) - \log(1)) + (\log(3) - \log(2)) + \cdots + (\log(N+1) - \log(N)) \\
&= \log(N+1) - \log(1) = \log(N+1)
\end{align*}

This is a telescoping series. As $N \to \infty$, $\log(N+1) \to \infty$.

The series diverges.

\end{enumerate}
\end{problem}


\begin{problem} % Problem 35
\phantom{foo}

% PLEASE READ:  In (a) you should, of course, consider the sequences of partial sums.  The idea of monotone sequences will arise.  In (b), use the known fact that the harmonic series diverges (which you do not need to prove) and part (a) to conclude.  Note that part (a) is a comparison test for series.

\renewcommand{\labelenumi}{(\alph{enumi})}
\begin{enumerate}
\item Suppose $0 \le a_n \le b_n$.  If \, $\displaystyle \sum_{n=1}^\infty a_n$ diverges then $\displaystyle \sum_{n=1}^\infty b_n$ diverges.

\begin{proof}
Let $A_N = \sum_{n=1}^N a_n$ and $B_N = \sum_{n=1}^N b_n$ be the sequences of partial sums.

Since $0 \leq a_n \leq b_n$ for all $n$, summing from $n=1$ to $N$ gives:
$$A_N = \sum_{n=1}^N a_n \leq \sum_{n=1}^N b_n = B_N$$

Both $(A_N)$ and $(B_N)$ are monotone increasing sequences since the terms are non-negative.

Suppose $\sum_{n=1}^\infty a_n$ diverges. Then $A_N \to \infty$ as $N \to \infty$, meaning $(A_N)$ is unbounded.

Since $A_N \leq B_N$ for all $N$ and $(A_N)$ is unbounded, $(B_N)$ must also be unbounded. Therefore $B_N \to \infty$ as $N \to \infty$, so $\sum_{n=1}^\infty b_n$ diverges
\end{proof}

\item $\displaystyle \sum_{n=1}^\infty \frac{1}{\sqrt{n}}$ diverges.

\begin{proof}
We know the harmonic series $\sum_{n=1}^\infty \frac{1}{n}$ diverges. 

For all $n \geq 1$, we have $\sqrt{n} \leq n$, so $\frac{1}{n} \leq \frac{1}{\sqrt{n}}$.

By part (a), since $0 \leq \frac{1}{n} \leq \frac{1}{\sqrt{n}}$ and $\sum_{n=1}^\infty \frac{1}{n}$ diverges, we conclude that $\sum_{n=1}^\infty \frac{1}{\sqrt{n}}$ diverges.
\end{proof}

\end{enumerate}
\end{problem}

\end{document}
