% <-- a percent symbol indicates a comment which does not affect the output of LaTeX
% you can leave the preamble alone, from here ...
\documentclass[12pt]{article}

\usepackage{amssymb,amsmath,amsthm}
\usepackage[top=1in, bottom=1in, left=1.25in, right=1.25in]{geometry}
\usepackage{enumerate,palatino}
\usepackage[final]{graphicx}
\usepackage[colorlinks=true,citecolor=blue,linkcolor=red,urlcolor=blue]{hyperref}

\newtheorem{problem}{Problem}
% ... to here

% shortcuts for blackboard bold number sets (reals, integers, etc.)
\newcommand{\II}{\ensuremath{\mathbb I}}
\newcommand{\NN}{\ensuremath{\mathbb N}}
\newcommand{\QQ}{\ensuremath{\mathbb Q}}
\newcommand{\RR}{\ensuremath{\mathbb R}}
\newcommand{\ZZ}{\ensuremath{\mathbb Z}}
\newtheorem{theorem}{Theorem}[section]
\newtheorem{definition}{Definition}
\newcommand{\eps}{\ensuremath{\epsilon}}

% feel free to add more shortcuts here


\begin{document}
% replace with your name, but otherwise leave this header alone, from here ...
\small
\noindent \textsc{Foundational Treatise for Real Analysis} \hfill Christopher Munoz
%Forgot this last time

\normalsize
\bigskip
% ... to here
We will begin by building theorems from the ground up from basic rules

\begin{definition} Convergence: For $A_n \to L$ means: For all $\eps > 0$, there exists $N$ such that for all $n > N$, implies $\mid a_n - L\mid < \eps$.
\end{definition}

\begin{definition} Bounded: $(a_n)$ is bounded if there exists $M > 0$ such that $|a_n| \leq M$ for all $n$.
\end{definition}

\begin{definition} Triangle inequality
  \begin{align*}
    \text{Triangle inequality :} && \mid a + b \mid \leq \mid a \mid + \mid b \mid \\
    \text{Reverse triangle :} && ||a| - |b|| \leq |a - b| \\
    \text{Product bound :} && |ab| = |a||b|
  \end{align*}
\end{definition}


\begin{theorem} If $(a_n)$ converges to $L$, then $(a_n)$ is bounded.
  \begin{proof}
    Since $(a_n)$ converges to $L$, this means that for an $\eps > 0$, there exists $N$ such that for all $n > N$, implies $\mid a_n - L\mid < \eps$. From this we get the following inequality:
    \begin{align*}
      |a_n| = |a_n - L + L| \leq |a_n - L| + |L| && \text{ By Triangle Inequality}
    \end{align*}
    Now let $\eps  = 1$, then there exists an $n > N$ such that $|a_n - L | < 1$, it follows from this that
    \begin{align*}
      |a_n| = |a_n - L + L| \leq |a_n - L| + |L| < 1 + |L|
    \end{align*}
    for $n \leq N$, let $M_1 = \max\{|a_1|, |a_2|, \cdots, |a_N|\}$.
    Now let $M = \{M_1, 1 + |L|\}$.
    Then $|a_n| \leq M$ for all $n$.
  \end{proof}
\end{theorem}

\begin{theorem} (Uniqueness of Limits) If $a_n \to L$ and $a_n \to M$ then $L = M$.
\end{theorem}
\begin{proof}
  Let $\eps > 0$ be arbitrary. Since $a_n \to L$ there exists an $N_1$ such that for all $n \geq N : |a_n -L| < \frac{\eps}{2}$.
  
  Likewise since $a_n \to M$, there exists $N_2$ such that for all $n \geq N_2: |a_n - M| < \frac{\eps}{2}$. 

  Let $N = \max\{N_1,N_2\}$. For $n \geq N$:
  \begin{align*}
    |L - M| = |L - a_n + a_n - M| \leq |a_n - L| + |a_n + M| < \frac{\eps}{2} + \frac{\eps}{2} = \eps
  \end{align*}
  Since this holds for arbitrary $\eps > 0$, we must have $|L - M| = 0$, so L = M.

\end{proof}

\begin{theorem}(Algebraic Limit Theorem)
If $x_n \to a$ and $y_n \to b$, then the algebraic limit theorem states 
  \begin{align}
    \textbf{Sum: } && \lim(x_n + y_n) = a + b  \\
    \textbf{Scalar: } && \lim(cx_n) = ca \\
    \textbf{Product: } && \lim(x_n * y_n) = a * b \\
    \textbf{Quotient: } && \lim(\frac{x_n}{y_n}) = \frac{a}{b} && \text{for $b \neq 0$}
  \end{align}
  \begin{proof}
    Sum: Recall that a sequence $(s_n)$ converges to $L$ if for every $\eps > 0$ there exists an $N \in \NN$ such that if $n \geq N$ then $|s_n - L| < \eps$. Given $x_n \to a$ and $y_n \to b$, it follows that there exists $N_1, N_2 \in \NN$ such that if $n \geq N$ we have $n \geq N_1$ and $n \geq N_2$ such that $|x_n - a| < \eps/2$ and $|y_n - b| < \eps/2$. Let $N = \max\{N_1, N_2\}$. In order to show that $\lim(x_n + y_n) = a + b$, we need to show that $|(x_n + y_n) - (a + b)| < \eps$ (epsilon definition of equality). Observe that
    \begin{align*} 
      |(x_n + y_n) - (a + b)| = |(x_n -a) + (y_n - b)| \leq |x_n - a| + |y_n - b| < \eps/2 + \eps/2 = \eps
    \end{align*}
    By the triangle inequality. Thus $\lim(x_n + y_n) = a + b$. 
  \end{proof} 
  \begin{proof}
    Scalar: 
  \end{proof}
\end{theorem}

\begin{theorem}Density of $\QQ$: For every two real numbers $a$ and $b$ with $a < b$, there exists a rational number $r \in \QQ$ satisfying $a < r < b.$
  \begin{proof}
    Let $a$ and $b$ be real numbers with $a < b$. Then $b - a > 0$. By the Archimedean property there exists $n \in \NN$ such that 
    $$n(b-a) > 1$$
    Rearranging gives 
    $$nb > na + 1$$
    Now suppose we have a set $S = \{k \in \ZZ: k > na\}$. By the well-ordered properties of the integers and because $S$ is bounded by $k > na$. The set $S$ has a least element. Denote this element as as $m = \min S$. Note that since $m \in S$, it follows that $m > na$ and $m - 1 \notin S$. So $m -1 \leq na$. From this we get
    $$m \leq na + 1$$
    noticeably 
    $$m \leq na + 1 < nb$$
    Together with $m > na$ we get
    $$na < m < nb$$
    Dividing by $n$ we get
    $$a < \frac{m}{n} < b$$
    Setting $r = \frac{m}{n} \in \QQ$, we conclude that $a < r < b$.
  \end{proof}
\end{theorem}
\begin{theorem} Continous preserves compactness: Let $f : A \to \RR$ be continous on A. If $K \subseteq A$ is compact then $f(K)$ is compact.
  \begin{proof}
    Let $K \subseteq A$ be compact. We will show that $f(K)$ is compact by showing that every sequence in $f(K)$ has a subsequence that converges to a point in $f(K)$.
    Let $(y_n)$ be a sequence in $f(K)$.

    By definition of $f(K)$, for each $n \in \NN$ there exists $f(x_n) = y_n$.

    Since $K$ is compact and $(x_n)$ is a sequence in $K$, there exists a subsequence $(x_{n_k})$ of $(x_n)$ and a point $x \in K$ such that 
    $$\lim_{k \to \infty} x_{n_k} = x$$

    Since $f$ is continous on $A$ and $x \in K \subseteq A$, the function $f$ is continuous at $x$.

    Therefore by the sequential characterization of continuity,
    
    $$\lim_{k \to \infty} f(x_{n_k}) = f(x)$$

    But $f(x_{n_k}) = y_{n_k}$ for all $k$, so we have
    $$\lim_{k \to \infty} y_{n_k} = f(x) $$
    Since $x \in K$, we have $f(x) \in f(K)$.

    Thus, we have found a subsequence $(y_{n_k})$ of $(y_n)$ that converges to a point $f(x) \in f(K)$.
    Since $(y_n)$ was an arbitrary sequence in $f(K)$, we have concluded that every sequence in $f(K)$ has a subsequence converging to a point in $f(K)$.
  \end{proof}
\end{theorem}
\end{document}

