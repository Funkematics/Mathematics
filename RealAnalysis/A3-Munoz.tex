% <-- a percent symbol indicates a comment which does not affect the output of LaTeX
% you can leave the preamble alone, from here ...
\documentclass[12pt]{article}

\usepackage{amssymb,amsmath,amsthm}
\usepackage[top=1in, bottom=1in, left=1.25in, right=1.25in]{geometry}
\usepackage{enumerate,palatino}
\usepackage[final]{graphicx}
\usepackage[colorlinks=true,citecolor=blue,linkcolor=red,urlcolor=blue]{hyperref}

\newtheorem{problem}{Problem}
% ... to here

% shortcuts for blackboard bold number sets (reals, integers, etc.)
\newcommand{\II}{\ensuremath{\mathbb I}}
\newcommand{\NN}{\ensuremath{\mathbb N}}
\newcommand{\QQ}{\ensuremath{\mathbb Q}}
\newcommand{\RR}{\ensuremath{\mathbb R}}
\newcommand{\ZZ}{\ensuremath{\mathbb Z}}

% feel free to add more shortcuts here


\begin{document}
% replace with your name, but otherwise leave this header alone, from here ...
\small
\noindent \textsc{Math 401: Homework Assignment 3} \hfill YOUR NAME HERE

\normalsize
\bigskip
% ... to here

\setcounter{problem}{13}


\begin{problem} % Problem 14
Suppose $A,B$ are disjoint sets with $A\cup B = \RR$, and suppose that $a<b$ for all $a\in A$ and $b\in B$.  Then there exists $c\in\RR$ such that $x\le c$ for $x\in A$ and $x\ge c$ for $x\in B$. 
\end{problem}

% PLEASE READ:  This is a cleaner version of Exercise 1.3.10 (a).  Prove the above claim using the Axiom of Completeness.

\begin{proof}
 Since $A$ and $B$ are non-empty sets and $a < b$ for all $a \in A$ and $b \in B$, any $b \in B$ is an upper bound for $A$. This means that by the least upper bound property of the Axiom of Completeness, $A$ has a supremum that we will denote as $c = \sup{A}$. We want to show that for any $x \in A$, that $x \leq c$ and for any $x \in B$, that $x \geq c$.

For any $x \in A$, we have $x \leq c$ by definition of supremum.

For any $x \in B$, suppose for contradiction that $x < c$. Since $c$ is the least upper bound of $A$, there exists some $a \in A$ with $x < a \leq c$. But this contradicts the given condition that $a < b$ for all $a \in A$ and $b \in B$. Therefore $x \geq c$ for all $x \in B$.
 \end{proof}


\begin{problem} % Problem 15
Here is an example which shows that the claim in Problem 14 is false if $\RR$ is replaced, in both instances, by the set of rationals $\QQ$:
% PLEASE READ:  This is a cleaner version of Exercise 1.3.10 (c).
\newline
\newline
Let $A = \{x \in \QQ : x< \pi\}$ and $B = \{x \in \QQ : x > \pi\}$.
\newline
\newline
Note that $A \cup B = \QQ $ and that $A \cap B = \emptyset$ meaning $A$ and $B$ are disjoint sets.
Consider that for all $a \in A$ and $b \in B$ we have $a < \pi < b$, so that $a < b$ like our previous problem. Unlike our previous problem, there is
no $c$ that satisfies $x \leq c$ for $x \in A$ and $ x \geq c$ for $x \in B$ because although $\pi$ is a supremum for $A$ and an infimum for $B$, it does not exist in $\QQ$. Thus the claim in Problem 14 is false for $\QQ$.
\end{problem}


\begin{problem} % Problem 16
Let $a<b$ be real numbers.  Define the set $T=\QQ \cap [a,b]$.  Then $\sup T = b$.
\end{problem}

% PLEASE READ:  This is Exercise 1.4.4.

\begin{proof}
For any $t \in T = \mathbb{Q} \cap [a,b]$, we have $t \in [a,b]$, so $t \leq b$. Thus $b$ is an upper bound of $T$.

To show $b = \sup T$, suppose $N$ is an upper bound with $N < b$. By density of rationals, there exists $r \in \mathbb{Q}$ with $N < r < b$. Since we can choose $r$ close enough to $b$, we have $r \in [a,b]$, so $r \in T$. But then $r > N$, contradicting that $N$ is an upper bound.

Therefore $b = \sup T$.
\end{proof}


\begin{problem} % Problem 17
By definition, a set $C\subseteq \RR$ is \emph{dense} if for any real numbers $a<b$ there is $c\in C$ so that $a<c<b$.  Let $T$ be the set of all rational numbers $p/q$, with $p\in\ZZ$, for which $q=2^k$ for some $k\in\NN$.  Then $T$ is dense.
\end{problem}

% PLEASE READ:  This is a cleaner version of Exercise 1.4.6 (b).  Equivalently to the definition above, a set is dense if every nonempty open interval of the real line intersects the set.  As a hint on showing T is dense, you might look at the beginning of the proof of Theorem 1.4.3, but your proof will need to come up with a rational number in (a,b) which has a power of 2 in the denominator.

\begin{proof}
  Let $a < b$ where $a \in \RR$ and $b \in \RR$. We want to show that there exists $p \in \ZZ$ and $k \in \NN$ such that $a < \frac{p}{2^k} < b$. Note that by the Archimedean Property of $\RR$, there exists a $k \in \NN$ such that $\frac{1}{2^k} < b - a$ which is equivalent to $2^k > \frac{1}{b -a}$.
  Consider the open interval $(a2^k, b2^k) \subseteq \RR$. The length of this interval is
  \begin{align*}
    b2^k - a2^k =2^k(b - a) > 2^k * \frac{1}{2^k} = 1
  \end{align*}
  Since this interval has a length greater than 1, by the density of integers in $\RR$, there exists $p \in \ZZ$ such that $a2^k < p < b2^k$. Dividing the strict inequality by $2^k$ preserves the inequality direction so that $a < \frac{p}{2^k} < b$. Since $p \in \ZZ$ and $k \in \NN$, we have $\frac{p}{2^k} \in T$. Therefore, every open interval $(a, b)$ contains an element of $T$, thus $T$ is dense in $\RR$.
\end{proof}


\begin{problem} % Problem 18
\phantom{foo}
% PLEASE READ:  This is a cleaner version of Exercise 1.4.8 (a)--(c).

\renewcommand{\labelenumi}{(\alph{enumi})}
\begin{enumerate}
\item An example of two real sets $A,B$ with $A\cap B=\emptyset$, $\sup A = \sup B$, $\sup A \notin A$, and $\sup B \notin B$ is
  \begin{align*}
    A &= \{1 - \frac{1}{2k} : k \in \NN\} = \{\frac{1}{2}, \frac{3}{4}, \frac{5}{6}, \frac{7}{8},...\} \\
    B &= \{1 - \frac{1}{2k+1} : k \in \NN\} = \{\frac{2}{3}, \frac{4}{5}, \frac{6}{7}, \frac{8}{9},... \}
\end{align*}
  The sets above satisfy the properties of $A \cap B = \emptyset$ (disjoint) and $\sup A = 1 = \sup B$, with $1$ being in neither of the sets.
\item An example of a sequence of nested open intervals $J_1\supseteq J_2 \subseteq J_3 \supseteq \dots$, with $S=\bigcap_{n=1}^\infty J_n$ nonempty and of finite cardinality, is
\begin{align*}
  J_{2n-1} &= (-\frac{1}{n}, \frac{1}{n})\\
  J_{2n} &= (-1,1) 
\end{align*}
This satisfies the properties $J_1 \supseteq J_2 \subseteq J_3 \supseteq \dots$ property by having every odd indexed interval be a subset of the even indexed intervals. It satisfies the second property because $S=\bigcap_{n=1}^\infty J_n = \{0\}$. It satisfies the third property by having a cardinality of $1$. 
\item By definition, an unbounded closed interval is of the form $[a,\infty) = \{x\in\RR \,:\, x \ge a\}$.  An example of a sequence of nested unbounded closed intervals $L_1\supseteq L_2 \subseteq L_3 \supseteq \dots$, with $\bigcap_{n=1}^\infty L_n = \emptyset$, is
\begin{align*}
  L_{2n-1} &= [n, \infty) \\
  L_{2n} &= [1, \infty)
\end{align*}
This satisfies the first property as our odd indexed intervals will be subsets to our even indexed intervals. It satisfies the second property because odd indexed intervals will just keep starting further right. Every point eventually gets excluded this way.
\end{enumerate}
\end{problem}


\begin{problem} % Problem 19
If $A\subseteq B$ and $B$ is countable then $A$ is either countable or finite.
\end{problem}

% PLEASE READ:  See Exercise 1.5.1 for suggestions on the proof.  I have written the first 3 sentences of the proof.

\begin{proof}
Assume $B$ is countable.  If $|A|<\infty$ then $A$ is finite and we are done.  So we will consider an infinite subset $A\subseteq B$ and show it is countable.

Since $B$ is countable, we can enumerate the elements of $B = \{b_1, b_2, b_3, \dots\}$. Since $A \subseteq B$, we can similarly enumerate the elements of $A$ by taking them in the order they appear in the enumeration of $B$. This gives us a bijection between $\NN$ and $A$, so $A$ is countable.
\end{proof}


\begin{problem} % Problem 20
\phantom{foo}
% PLEASE READ:  This is a shorter version of Exercise 1.5.4 (a) and (c).  For (b) below I was able to do the proof starting from (a), and from the idea of splitting S=[0,1) into disjoint rational and irrational portions: S = (S \cap \QQ) \cup (S \cap \II)

\renewcommand{\labelenumi}{(\alph{enumi})}
\begin{enumerate}
\item For any $a<b$ it follows that $(a,b) \sim \RR$.

\begin{proof}
We define $f : (-1,1) \rightarrow \RR$ as
\begin{align*}
  f(x) = \frac{x}{x^2 - 1}
\end{align*}
Note that for $x \in (-1,1)$ we have $x^2 < 1$, so $x^2 - 1 < 0$ and the function is well-defined. We will show that $f$ is bijective: \newline
\textbf{Injective: } First we will examine the derivative $$f'(x) = \frac{-x^2 - 1}{(x^2 - 1)^2}$$ Since the numerator is negative and denominator is positive, we have $f'(x) < 0$ for all $x \in (-1, 1)$, so $f$ is strictly decreasing, thus injective. \newline
\textbf{Surjective: } Observe that
\begin{align*}
  \lim_{x \to 1^-} \frac{x}{x^2-1} = -\infty \\
  \lim_{x \to (-1)^+} \frac{x}{x^2-1} = +\infty
\end{align*}
By the intermediate Value Theorem, $f$ takes on every value, thus surjective. \newline \newline
Now we define $g : (a,b) \rightarrow (-1,1)$ as
\begin{align*}
  g(x) = \frac{2x - (a + b)}{b - a}
\end{align*}
This function linearly maps $(a, b)$ to $(-1, 1)$ and is bijective. The composition $f \circ g : (a, b) \to \RR$ is a bijection, so $(a,b)$ has the same cardinality as $\RR$.
\end{proof}

\item $[0,1) \sim (0,1)$

\begin{proof}
We want a bijective function from $[0,1)$ to $(0,1)$. We'll use composition of bijections.

First, define $g: [0,1) \to [0,\infty)$ by:
$$g(x) = \frac{x}{1-x}$$

This maps $[0,1)$ bijectively to $[0,\infty)$ since $g(0) = 0$ and $\lim_{x \to 1^-} g(x) = \infty$.

Next, define $h: [0,\infty) \to (0,1)$ by:
$$h(x) = \frac{1}{x+2}$$

This maps $[0,\infty)$ bijectively to $(0,\frac{1}{2}]$ since $h(0) = \frac{1}{2}$ and $\lim_{x \to \infty} h(x) = 0$.

Finally, define $k: (0,\frac{1}{2}] \to (0,1)$ by:
$$k(x) = 2x$$

This is clearly a bijection.

The composition $k \circ h \circ g: [0,1) \to (0,1)$ is a bijection, so $[0,1)$ has the same cardinality as $(0,1)$.
\end{proof}

\end{enumerate}
\end{problem}


\begin{problem} % Problem 21
If $A\sim B$ and $B\sim C$ then $A\sim C$.
\end{problem}

% PLEASE READ:  This is Exercise 1.5.5 (c), that is, showing the transitive part of showing that $\sim$ is an equivalence relation.

\begin{proof}
Since $A$ has the same cardinality as $B$, there exists a bijection $f: A \to B$.
Since $B$ has the same cardinality as $C$, there exists a bijection $g: B \to C$.
Consider the composition $h = g \circ f: A \to C$ defined by $h(x) = g(f(x))$ for all $x \in A$.
We show $h$ is a bijection:
\newline
\textbf{Injective:} Suppose $h(x_1) = h(x_2)$ for some $x_1, x_2 \in A$. Then $g(f(x_1)) = g(f(x_2))$. Since $g$ is injective, we have $f(x_1) = f(x_2)$. Since $f$ is injective, we have $x_1 = x_2$. Therefore $h$ is injective.
\newline
\textbf{Surjective:} Let $z \in C$. Since $g$ is surjective, there exists $y \in B$ such that $g(y) = z$. Since $f$ is surjective, there exists $x \in A$ such that $f(x) = y$. Therefore $h(x) = g(f(x)) = g(y) = z$. This shows $h$ is surjective.
\newline\newline
Thus $h: A \to C$ is a bijection, and $A$ has the same cardinality as $C$.
\end{proof}


\end{document}
