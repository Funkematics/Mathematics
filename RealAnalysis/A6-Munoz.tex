% <-- a percent symbol indicates a comment which does not affect the output of LaTeX
% you can leave the preamble alone, from here ...
\documentclass[12pt]{article}

\usepackage{amssymb,amsmath,amsthm}
\usepackage[top=1in, bottom=1in, left=1.25in, right=1.25in]{geometry}
\usepackage{enumerate,palatino}
\usepackage[final]{graphicx}
\usepackage[colorlinks=true,citecolor=blue,linkcolor=red,urlcolor=blue]{hyperref}

\newtheorem{problem}{Problem}
% ... to here

% shortcuts for blackboard bold number sets (reals, integers, etc.)
\newcommand{\II}{\ensuremath{\mathbb I}}
\newcommand{\NN}{\ensuremath{\mathbb N}}
\newcommand{\QQ}{\ensuremath{\mathbb Q}}
\newcommand{\RR}{\ensuremath{\mathbb R}}
\newcommand{\ZZ}{\ensuremath{\mathbb Z}}

\newcommand{\eps}{\ensuremath{\epsilon}}
\newcommand{\ds}{\displaystyle}

% feel free to add more shortcuts here


\begin{document}
% replace with your name, but otherwise leave this header alone, from here ...
\small
\noindent \textsc{Math 401: Homework Assignment 6} \hfill Christopher Munoz

\normalsize
\bigskip
% ... to here

\setcounter{problem}{35}

\begin{problem} % Problem 36
Give a justified example of each, or argue (prove) that it is impossible.

\renewcommand{\labelenumi}{(\alph{enumi})}
\begin{enumerate}
\item A sequence that has a subsequence that is bounded, but which contains no subsequence which converges.

This is impossible by Bolzano Weierstrass. Every bounded sequence has at least one convergent subsequence.

\item A sequence that does not contain $0$ or $1$ as a term, but which contains subsequences which converge to each of these values.

$$a_n = \frac{1 + (-1)^n}{2} + \frac{1}{n}$$ is such a sequence, we can set $n$ to even or odd numbers to converge to $0$ or $1$.

\item A sequence that contains subsequences converging to every point in the infinite set $\{1,1/2,1/3,1/4,\dots\}$.

  Consider that we can construct a subsequence that convergest to a chosen arbitrary value with $k - \frac{1}{n}$ where $k$ is any number we want to converge to and $\frac{1}{n}$ just going to zero. Let our sequence be defined by $a_n = \frac{1}{k} - \frac{1}{n}$. For $k, n \in \NN$ this converges to every point in the infinite set.

  incomplete
\end{enumerate}
\end{problem}


\begin{problem} % Problem 37
Let $(a_n)$ be a bounded sequence.  Define the set
	$$S = \left\{x\in\RR\,:\, x < a_n \,\text{ for infinitely many terms } a_n\right\}.$$
Then $S$ is bounded above, and there exists a subsequence $(a_{n_k})$ which converges to $\sup S$.
\end{problem}
% Screw this problem in particular
\begin{proof}
Since $(a_n)$ is a bounded sequence, there exists an $M \in \RR$ such that $a_n \leq M$ for all $n \in \NN$. From this we have
$$x < a_n < M$$
by transitivity $x < M$ for all $x \in S$, so $S$ is bounded above by $M$. Since $S$ is a non-empty real set and bounded above, By Axiom of completeness, $s = \sup S$ exists.

Choose an arbitrary $k \in \NN$ so that we create an interval around the supremum $s$: $$s-\frac{1}{k} < s < s+\frac{1}{k}$$ Since any number smaller than $s$ is not an upper bound of $S$, there exists an $s' \in S$ so that $s-\frac{1}{k} < s'$($s'$ is in the interval below $s$). Since $s' \in S$, it follows by transitivity that $s-\frac{1}{k} < s' < a_n$, thus $s-\frac{1}{k} < a_n$ for infinitely many terms $a_n$. So we have $$s-\frac{1}{k} < a_n < s + \frac{1}{k}$$ Satisfied by every $k \in \NN$. We construct the subsequence $a_{n_k}$ recursively.
For $k = 1$, choose any $n_1 \in \NN$ such that $s - 1 < a_{n_1} \leq s + 1$. Having chosen 
$n_1 < n_2 < \cdots < n_k$, we choose $n_{k+1} > n_k$ such that 
$$s - \frac{1}{k+1} < a_{n_{k+1}} < s + \frac{1}{k+1}.$$ Now we show convergence, Let $\eps > 0$, choose $K \in \NN$ such that $\frac{1}{K} < \eps$. Then for all $k \geq K$, we have
$$\frac{1}{k} \leq \frac{1}{K} < \eps$$
By construction
$$s = \frac{1}{k} < a_{n_k} \leq s + \frac{1}{k}$$
Since $\frac{1}{k} < \eps$, we have
$$ s- \eps < s - \frac{1}{k} < a_{n_k} \leq s + \frac{1}{k} < s + \eps$$
thus $|a_{n_k} - s| < \eps$ meaning by definition there is a subsequence $a_{n_k}$ that converges to $\sup S$.
\end{proof}


\begin{problem} % Problem 38
Every convergent sequence is a Cauchy sequence.
\end{problem}

% PLEASE READ:  Provide a complete proof.  Note that a partial proof is offered in the textbook, for Theorem 2.6.2.

\begin{proof}
A sequence is Cauchy iff for every $\eps > 0$, there exists an $N \in \NN$ such that for every $m, n \in \NN$ when $m,n > N$ we have $|a_n - a_m| < \eps$.
\newline
\newline
Let $(a_n)$ be a convergent sequence and let $(a_n) \to a$. By definition this means that for $\eps > 0$, there exists an $N \in \NN$ such that when $n > N$ we have $|a_n - a| < \frac{\eps}{2}$. We now show that this is a Cauchy sequence. Let $\eps > 0$ and let $m,n > N$. Observe that
\begin{align*}
  \mid a_n - a_m \mid &= \mid (a_n - a) + (a - a_m) \mid \\
                      &= \mid a_n - a \mid + \mid a - a_m \mid \\
                      &< \frac{\eps}{2} + \frac{\eps}{2} \\
                      &= \eps
\end{align*}
Thus $(a_n)$ is a Cauchy sequence.
\end{proof}


\begin{problem} % Problem 39
Give a justified example of each, or argue (prove) that it is impossible.

\renewcommand{\labelenumi}{(\alph{enumi})}
\begin{enumerate}
\item A Cauchy sequence that is not monotone.

Since all convergent sequences are Cauchy sequences, we just need to find any sequence that converges that is not monotone.
Let $a_n = \frac{(-1)^n}{n}$. 

\item A Cauchy sequence containing an unbounded subsequence.

Boundedness is a criteria for convergence so this is impossible

\item An unbounded sequence containing a Cauchy subsequence.

Impossible for the same reason as above

\end{enumerate}
\end{problem}


\begin{problem} % Problem 40
Give a justified example of each, or explain (prove) why the request is impossible, by referencing the proper theorem(s).

\renewcommand{\labelenumi}{(\alph{enumi})}
\begin{enumerate}
\item Two series $\sum x_n$ and $\sum y_n$ which both diverge, but where $\sum x_n y_n$ converges.

  Let $x_n = y_n = \frac{1}{n}$, then $\sum x_n$ and $\sum y_n$ diverge. Consider the product $\sum x_n y_n = \sum \frac{1}{n} * \frac{1}{n} = \frac{1}{n^2}$.
  This is a $p$ series where $p > 1$ and thus converges.

\item A convergent series $\sum x_n$ and a bounded sequence $(y_n)$, such that $\sum x_n y_n$ diverges.

  Let $x_n = \frac{(-1)^n}{n}$ and let $y_n = (-1)^n$. The sum $\sum x_n = \sum \frac{(-1)^n}{n}$ is an alternating harmonic series so it converges.. The sum $\sum y_n = \sum (-1)^n$ just flips between $-1$ if odd and $1$ if even, this is also the greatest lower bound and least upper bound respectively.The product $\sum x_ny_n = \sum [\frac{(-1)^n}{n} * (-1)^n] =  \sum [\frac{(-1)^{2n}}{n}] = \sum \frac{1}{n} $ diverges.

\item Two sequences $(x_n)$ and $(y_n)$ where $\sum x_n$ and $\sum (x_n+y_n)$ both converge, but $\sum y_n$ diverges.

  Impossible
  \begin{proof}Suppose $(x_n)$ converges and $\sum (x_n + y_n)$ converge with $(y_n)$ diverging. Observe that $\sum y_n = \sum (x_n + y_n) - \sum x_n$, by the algebraic rule for series. A consequence is that $(y_n)$ converges. But this is a contradiction since we assumed $(y_n)$ diverges. Thus this is impossible. 
  \end{proof}

\item A sequence $(x_n)$ satisfying $0\le x_n \le 1/n$ where $\sum (-1)^n x_n$ diverges.

  Let 
  \begin{align*}x_n =
    \begin{cases}
      \frac{1}{n} & \text{if n is odd} \\
      0 & \text{if n even}
    \end{cases}
  \end{align*}
  This is just $\sum (-\frac{1}{n})$ which diverges without the even numbers.

\end{enumerate}
\end{problem}


\begin{problem} % Problem 41
If $\sum a_n$ converges absolutely then $\sum a_n^2$ converges absolutely.
\end{problem}

\begin{proof}
If $\sum |a_n|$ converges then $\lim |a_n| = 0$. It follows that there exists an $N \in \NN$ such that for all $n \geq N$ that $|a_n| < 1$. Since 
$|a_n|$ is positive(absolute value) we have
$$0 < |a_n| < 1$$
Being that $|a_n|$ is between $0$ and $1$, we have
$$|a_n^2| = |a_n|^2 \leq |a_n|$$
Since $\sum |a_n|$ converges and $|a_n^2| < |a_n|$, then by the Comparison Test, for all sufficiently large $n$, the sum $\sum |a_n^2|$ converges.
Thus $\sum a_n^2$ converges absolutely.
\end{proof}


\begin{problem} % Problem 42
Ratio test:  For a series $\sum a_n$, if the sequence of terms $(a_n)$ satisfies $a_n\ne 0$ for all $n$, and if
	$$\lim_{n\to\infty} \frac{|a_{n+1}|}{|a_n|} = r < 1,$$
then the series converges absolutely.
\end{problem}

% PLEASE READ:  Flesh out the following sketch of the proof.  For any $r'$ with $r < r' < 1$, explain why there exists $N$ so that if $n\ge N$ then $|a_{n+1}| \le r' |a_n|$.  Induct to get a bound on $|a_n|$, for $n\ge N$, in terms of a power of $r'$.  Explain why $|a_N| \sum_{n=N}^\infty (r')^{n-N}$ converges.  Then explain why this shows $\sum a_n$ converges absolutely.

\begin{proof}
Suppose $\lim_{n\to\infty} \frac{|a_{n+1}|}{|a_n|} = r < 1$
\newline
\newline
Choose $r'$ such that $r < r' < 1$ and let $\eps = 'r - r > 0$.
Since
$$\lim_{n\to\infty} \frac{|a_{n+1}|}{|a_n|} = r < 1$$
by the definition of limit, there exists $N \in \NN$ such that for all $n \geq N$.
$$ \mid \frac{|a_{n+1}|}{|a_n|} - r \mid < \eps = r'-r $$
This implies $$\frac{|a_{n+1}|}{|a_n|} < r + (r' - r) = r' $$ 
Thus for all $n \geq N$
$$|a_{n+1}| \leq r'|a_n|$$
We want to show that for all $n \geq N$ that
$$|a_n| \leq |a_N| * (r')^{n-N}$$
We show this by induction:
\newline
\textbf{Base case} Suppose $n = N$, then
$$|a_N| \leq |a_N| * (r')^{n-N} = |a_N| * (r')^0 = |a_N|$$
This gives is $|a_N| \leq |a_N|$ so the inequality holds.
\textbf{Inductive hypothesis} Suppose the statement holds for some particular $n \geq N$, that
$$|a_n| \leq |a_N| * (r')^{n-N}$$
\textbf{Inductive step} We want to show that the statemen holds for $n+1$. Observe that
$$a_{n+1} \leq r'|a_n| \leq r' * [|a_N| * (r')^{n-N}]$$
rearranging gives
\begin{align*}
  |a_{n+1}| & \leq |a_N| * r' * (r')^{n-N} \\
            & = |a_N| * (r')^{1+n-N} \\
            & = |a_N| * (r')^{(n+1)-N} \\
\end{align*}
Thus by induction $|a_n| \leq |a_N| * (r')^{n-N}$ holds for all $n \geq N$.
Taking the sums of the inequality we get
$$\sum_{n=N}^{\infty} |a_n| \le \sum_{n=N}^{\infty} |a_N| \cdot (r')^{n-N} 
= |a_N| \sum_{n=N}^{\infty} (r')^{n-N}$$ 
Let $k = n -N$ so we get
$$|a_n| \sum_{k=0}^\infty (r')^k$$
Since $0 < r' < 1$, this is a convergent geometric series with sum $\frac{1}{1-r}$.
Thus
$$\sum_{n=N}^\infty |a_n| \leq |a_N| * \frac{1}{1-r'} < \infty$$
Since the sume of $\sum_{n=N}^\infty |a_n|$ converges and the first $N-1$ are a finite sum we have
$$\sum_{n=1}^{\infty} |a_n| = \sum_{n=1}^{N-1} |a_n| + \sum_{n=N}^{\infty} |a_n| < \infty$$
Therefore $\sum a_n$ converges absolutely.

\end{proof}


\begin{problem} % Problem 43
Do the following series converge or diverge?  A careful proof is not needed, but a logical and correct justification or explanation is required, possibly using Theorems from Sections 2.1--2.7, or Problems above.

\renewcommand{\labelenumi}{(\alph{enumi})}
\begin{enumerate}
\item $\sum_{n=1}^\infty \frac{1}{2^n+n}$

% FILL IN

\item $\sum_{n=1}^\infty \frac{\sin(n)}{n^2}$

% FILL IN

\item $1 - \frac{3}{4} + \frac{4}{6} - \frac{5}{8} + \frac{6}{10} - \frac{7}{12} + \frac{8}{14} + \dots$

% FILL IN

\item $1 - \frac{1}{2^2} + \frac{1}{3} - \frac{1}{4^2} + \frac{1}{5} - \frac{1}{6^2} + \frac{1}{7} - \frac{1}{8^2} + \frac{1}{9} - \dots$

% FILL IN

\item $\sum_{n=1}^\infty \frac{n^2}{2^n}$

% FILL IN

\end{enumerate}
\end{problem}

\end{document}
