% <-- a percent symbol indicates a comment which does not affect the output of LaTeX
% you can leave the preamble alone, from here ...
\documentclass[12pt]{article}

\usepackage{amssymb,amsmath,amsthm}
\usepackage[top=1in, bottom=1in, left=1.25in, right=1.25in]{geometry}
\usepackage{enumerate,palatino}
\usepackage[final]{graphicx}
\usepackage[colorlinks=true,citecolor=blue,linkcolor=red,urlcolor=blue]{hyperref}

\newtheorem{problem}{Problem}
% ... to here

% shortcuts for blackboard bold number sets (reals, integers, etc.)
\newcommand{\QQ}{\ensuremath{\mathbb Q}}
\newcommand{\RR}{\ensuremath{\mathbb R}}
\newcommand{\NN}{\ensuremath{\mathbb N}}
\newcommand{\ZZ}{\ensuremath{\mathbb Z}}

% feel free to add more shortcuts around here

\begin{document}
% replace with your name, but otherwise leave this header alone, from here ...
\small
\noindent \textsc{Math 401: Homework Assignment 2} \hfill Christopher Munoz

\normalsize
\bigskip
% ... to here

\setcounter{problem}{5}



\begin{problem} % Problem 6
    $$\bigcap_{n=1}^\infty (0,1/n)=\emptyset.$$
\end{problem}

% PLEASE READ:  Some suggestions for the proof:  Give the set a name, e.g. $S = \bigcap_{n=1}^\infty (0,1/n)$.  Then show by contradiction that $x\in S$ is false for every $x\in\RR$.  What property of the real numbers helps to show that, however small, a positive number is not in $S$?

\begin{proof}
	Let $S = \bigcap\limits_{i=1}^{\infty} (0,1/n) = \emptyset$. Let $x \in \RR$. Consider the following 3 cases. \newline
	\textbf{Case 1: } Suppose $x \leq 0$, then $x \notin S$ as $x \notin (0,1)$. \newline
	\textbf{Case 2: } Suppose $x \geq 1$, then $x \notin S$ as $x \notin (0,1)$. \newline
	\textbf{Case 3: } Suppose $0 < x < 1$, Choose $n \in \NN$ so that $n > \frac{1}{x}$. Then $x > \frac{1}{n}$. so $x \notin (0, \frac{1}{n})$, thus $x \notin S$
	\newline
	These cases show that an arbitrary $x \in R$ is not in $S$.
\end{proof}


\begin{problem} % Problem 7
Given a function $f$ and a subset $A$ of its domain, consider the image $f(A) = \{f(x) : x \in A\}$.

\begin{itemize}
\item[(a)] An example of a function $f$, and two subsets $A,B$ of the domain of $f$, for which $f(A \cap B) \neq f(A) \cap f(B)$ is

% PUT EXAMPLE HERE.  When you give an example you need to clearly say what $f$ is as a function, including its precise domain.  That domain must include $A$ and $B$ as subsets.  Of course you must be precise about the sets $f(A),f(B),f(A\cap B)$ too.
	$$f(x) = |x|$$
	where set $A$ is a subset of the domain defined by $A = \{-2,-1\}$ and where set $B$ is a subset of the domain defined by $B = \{1,2\}$. Observe that $f(A \cap B) = \emptyset$ and $f(A) \cap f(B) = \{1,2\}$.
% IN PART (b), FORM A CONJECTURE, AND THEN PROVE AS A PROPOSITION, A RELATIONSHIP BETWEEN  f(A \cup B) AND f(A) \cup f(B)
\item[(b)] If $A$, $B$ are subsets of the domain of $f$ then $f(A \cup B)$ IS RELATED IN SOME WAY TO $f(A) \cup f(B)$.
\newline \newline	Proposition: If $f$ is a function and $A,B$ are subsets of the domain of $f$, then $f(A \cup B) = f(A) \cup f(B)$.
\begin{proof}
Let $y \in f(A \cup B)$. Then there exists $x \in A \cup B$ such that $f(x) = y$. Since $x \in A \cup B$, then either $x \in A$ or $x \in B$. If $x \in A$, then $y = f(x) \in f(A) \subseteq f(A) \cup f(B)$. If $x \in B$, then $y = f(x) \in f(B) \subseteq f(A) \cup f(B)$. Thus $f(A \cup B) \subseteq f(A) \cup f(B)$. \newline\newline Conversely, let $y \in f(A) \cup f(B)$. Then either $y \in f(A)$ or $y \in f(B)$. If $y \in f(A)$, then there exists $x \in A$ such that $f(x) = y$. Since $A \subseteq A \cup B$, we have $x \in A \cup B$, so $y \in f(A \cup B)$. Similarly, if $y \in f(B)$, then $y \in f(A \cup B)$. Therefore $f(A) \cup f(B) \subseteq f(A \cup B).$ \newline \newline Since  $f(A \cup B) \subseteq f(A) \cup f(B)$ and $f(A \cup B) \supseteq f(A) \cup f(B)$ have been shown to be true, it follows that $f(A \cup B) = f(A) \cup f(B)$.
\end{proof}
\end{itemize}
\end{problem}


\begin{problem} % Problem 8
If $a\in\RR$ is an upper bound for $A\subset \RR$, and if $a$ is also an element of $A$, then $a = \sup A$.
\end{problem}

\begin{proof}
Choose $b \in \RR$ to be an upper bound for A. This means that if we choose an arbitrary $c \in \RR$ such that $c \in A$, then $b \geq c$.
But since $a \in A$ is also an upper bound, by definition it must be the case that $b \geq a$.
\end{proof}


\begin{problem} % Problem 9

% FOR EACH PART, FILL IN THE INFIMA AND SUPREMA.  YOU DO NOT HAVE TO PROVE ANYTHING.

\begin{enumerate}[(a)]
\item Let $A=\{m/n: \text{ $m,n\in\NN$ with $m<n$} \}$.  Then $\inf A = 0 $ and $\sup A = 1 $.
\item Let $B=\{(-1)^m/n: n,m\in\NN\}$.  Then $\inf B = -1 $ and $\sup B = 1 $.
\item Let $C=\{n/(3n+1): n\in\NN\}$.  Then $\inf C = 1/4 $ and $\sup C = 1/3$.
\item Let $D=\{m/(m+n):m,n\in\NN\}$.  Then $\inf D = 0 $ and $\sup D = 1 $.
\end{enumerate}
\end{problem}


\begin{problem} % Problem 10

% PLEASE READ:  Decide if the following statements are true.  Give a short proof for the true statements and a counterexample for the false statements.

\begin{enumerate}[(a)]
\item If $A$ and $B$ are nonempty, bounded, and satisfy $A\subseteq B$
then $\sup A\le \sup B$.

% UNCOMMENT AND FILL-IN ONE OF THESE
 \begin{proof} True: Given $A \subseteq B$, every element of $A$ is also an element of $B$. Since $B$ is bounded above, $\sup B$ exists and is an upper bound for $B$. Therefore, $\sup B$ is also an upper bound for $A$. \newline
Since $\sup A$ is the least upper bound of $A$, then $\sup B$ is an upper bound for $A$, it must be the case that $\sup A \leq \sup B$.\end{proof}
% \bigskip This is false because

\item If $\sup A< \inf B$ for nonempty sets $A$ and $B$, then there exists
$c\in\RR$ such that $a<c<b$ for all $a\in A$ and $b\in B$.

% UNCOMMENT AND FILL-IN ONE OF THESE
\begin{proof} Given that $\sup A < \inf B$, we can choose any $c$ such that $\sup A < c < \inf B$. Choose $c$ to be the average of $\sup A$ and $\inf B$ such that $c = \frac{\sup A + \inf B}{2}$.
\newline \newline
Since $c > \sup A$ and $\sup A$ is an upper bound for $A$, we have $a \leq \sup A < c$ for all $a \in A$. \newline 
Since $c < \inf B$ and $\inf B$ is a lower bound for $B$, we have $c < \inf B \leq b$ for all $b \in B$. \newline \newline 
Therefore $a < c < b$ for all $a \in A$ and $b \in B$.
\end{proof}
% \bigskip This is false because

\item If there exists $c\in\RR$ satisfying $a<c<b$ for all
$a\in A$ and $b\in B$ then $\sup A< \inf B$. 

% UNCOMMENT AND FILL-IN ONE OF THESE
% \begin{proof}  \end{proof}
 \bigskip This is false because we could have $\sup A = c$ or $\inf B = c$ which would give us $\sup A = \inf B$.

\end{enumerate}
\end{problem}


\begin{problem} % Problem 11
\newcommand{\II}{\mathbb{I}}
Denote the irrational numbers by $\II = \RR \setminus \QQ$.
\begin{enumerate}[(a)]
\item If $a,b\in\QQ$ then $ab\in\QQ$ and $a+b\in\QQ$.

\begin{proof}
Since $a, b \in \QQ$, we can write $a = \frac{p}{q}$ and $b = \frac{r}{s}$ where $p, r \in \ZZ$ and $q, s \in \NN$.

For multiplication: $ab = \frac{p}{q} \cdot \frac{r}{s} = \frac{pr}{qs}$. Since $pr \in \ZZ$ (integers are closed under multiplication) and $qs \in \NN$ (positive integers are closed under multiplication), we have $ab \in \QQ$.

For addition: $a + b = \frac{p}{q} + \frac{r}{s} = \frac{ps + qr}{qs}$. Since $ps + qr \in \ZZ$ (integers are closed under multiplication and addition) and $qs \in \NN$, we have $a + b \in \QQ$.
\end{proof}

\item If $a\in\QQ$ and $t\in\II$ then $a+t\in\II$.  If also $a\neq 0$ then $at\in\II$.

\begin{proof}
We prove both claims by contradiction.

For addition: Suppose $a + t \in \QQ$. Since $a \in \QQ$ and $a + t \in \QQ$, by part (a) we have $(a + t) + (-a) \in \QQ$. But $(a + t) + (-a) = t$, so $t \in \QQ$, contradicting that $t \in \II$. Therefore, $a + t \in \II$.

For multiplication (with $a \neq 0$): Suppose $at \in \QQ$. Since $a \in \QQ$ with $a \neq 0$, we have $\frac{1}{a} \in \QQ$. By part (a), $(at) \cdot \frac{1}{a} \in \QQ$. But $(at) \cdot \frac{1}{a} = t$, so $t \in \QQ$, contradicting that $t \in \II$. Therefore, $at \in \II$.
\end{proof}

\item Suppose $s,t\in\II$.  Then PROPOSITION ABOUT WHETHER $st$ AND $s+t$ ARE EITHER RATIONAL OR IRRATIONAL IN GENERAL.

% PLEASE READ: Part (a) says that the rational numbers are closed under multiplication and addition.  What can be said about $st$ and $s+t$ when $s,t\in\II$?  Formulate a conjecture and/or example and prove it.
	When $s, t \in \II$, both $st$ and $s + t$ can be either rational or irrational.

Examples for $s + t$:
- If $s = \sqrt{2}$ and $t = -\sqrt{2}$, then $s + t = 0 \in \QQ$.
- If $s = \sqrt{2}$ and $t = \sqrt{3}$, then $s + t = \sqrt{2} + \sqrt{3} \in \II$ (since if $\sqrt{2} + \sqrt{3} = r \in \QQ$, then $\sqrt{3} = r - \sqrt{2}$, and squaring gives $3 = r^2 - 2r\sqrt{2} + 2$, implying $\sqrt{2} = \frac{r^2 - 1}{2r} \in \QQ$, a contradiction).

Examples for $st$:
- If $s = \sqrt{2}$ and $t = \sqrt{2}$, then $st = 2 \in \QQ$.
- If $s = \sqrt{2}$ and $t = \sqrt{3}$, then $st = \sqrt{6} \in \II$ (since if $\sqrt{6} = r \in \QQ$, then $6 = r^2 \in \QQ$, but $r^2 = 6$ has no rational solutions).

Therefore, the irrational numbers are not closed under addition or multiplication.
\end{enumerate}
\end{problem}


\begin{problem} % Problem 12
For all $n\in\NN$, $2^n \ge n$.
\end{problem}

\begin{proof}
% FILL IN THE PROOF: Maybe one could just say ``this is obvious.''  Don't do that.  The careful way to prove this infinite collection of propositions (i.e. 2^1 \ge 1, 2^2 \ge 2, 2^3 \ge 3, ...) is to use induction.
	We prove by induction on $n$.

Base case: For $n = 1$, we have $2^1 = 2 \ge 1$, which is true.

Inductive step: Assume $2^k \ge k$ for some $k \in \NN$. We need to show $2^{k+1} \ge k + 1$.

Starting from the inductive hypothesis:
\begin{align*}
2^k &\ge k \\
2 \cdot 2^k &\ge 2k \quad \text{(multiplying both sides by 2)} \\
2^{k+1} &\ge 2k \\
2^{k+1} &\ge k + k \\
2^{k+1} &\ge k + 1 \quad \text{(since $k \ge 1$ for all $k \in \NN$)}
\end{align*}

Therefore, by mathematical induction, $2^n \ge n$ for all $n \in \NN$.
\end{proof}


\begin{problem} % Problem 13
Let $y_1=6$ and, for each $n\in\NN$, let $y_{n+1} = (2 y_n - 6) / 3$.
 
\begin{enumerate}[(a)]
\item For all $n\in \NN$, $y_n \ge -6$.

\begin{proof}
% FILL IN THE PROOF.  USE INDUCTION.
	We prove by induction on $n$.

Base case: For $n = 1$, we have $y_1 = 6 \ge -6$, which is true.

Inductive step: Assume $y_k \ge -6$ for some $k \in \NN$. We need to show $y_{k+1} \ge -6$.

From the inductive hypothesis: $y_k \ge -6$
\begin{align*}
y_k &\ge -6 \\
2y_k &\ge -12 \\
2y_k - 6 &\ge -18 \\
\frac{2y_k - 6}{3} &\ge -6 \\
y_{k+1} &\ge -6
\end{align*}

Therefore, by mathematical induction, $y_n \ge -6$ for all $n \in \NN$.
\end{proof}

\item The sequence $(y_1,y_2,y_3,\dots)$ is decreasing.

\begin{proof}
% FILL IN THE PROOF.  USE INDUCTION.
	We prove by induction that $y_{n+1} < y_n$ for all $n \in \NN$.

First, we show $y_2 < y_1$:
$y_2 = \frac{2(6) - 6}{3} = \frac{6}{3} = 2 < 6 = y_1$.

Now we prove by induction that if $y_n < y_{n-1}$, then $y_{n+1} < y_n$.

Assume $y_k < y_{k-1}$ for some $k \ge 2$. We need to show $y_{k+1} < y_k$.

We have:
\begin{align*}
y_{k+1} - y_k &= \frac{2y_k - 6}{3} - y_k \\
&= \frac{2y_k - 6 - 3y_k}{3} \\
&= \frac{-y_k - 6}{3} \\
&= -\frac{y_k + 6}{3}
\end{align*}

From part (a), we know $y_k \ge -6$, so $y_k + 6 \ge 0$. Since $y_1 = 6 > -6$ and each step moves closer to $-6$ (but never reaches it), we have $y_k > -6$ for all $k$, so $y_k + 6 > 0$.

Therefore, $y_{k+1} - y_k = -\frac{y_k + 6}{3} < 0$, which means $y_{k+1} < y_k$.

By induction, the sequence is decreasing.
\end{proof}
\end{enumerate}
\end{problem}

\end{document}
