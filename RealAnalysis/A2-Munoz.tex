% <-- a percent symbol indicates a comment which does not affect the output of LaTeX
% you can leave the preamble alone, from here ...
\documentclass[12pt]{article}

\usepackage{amssymb,amsmath,amsthm}
\usepackage[top=1in, bottom=1in, left=1.25in, right=1.25in]{geometry}
\usepackage{enumerate,palatino}
\usepackage[final]{graphicx}
\usepackage[colorlinks=true,citecolor=blue,linkcolor=red,urlcolor=blue]{hyperref}

\newtheorem{problem}{Problem}
% ... to here

% shortcuts for blackboard bold number sets (reals, integers, etc.)
\newcommand{\QQ}{\ensuremath{\mathbb Q}}
\newcommand{\RR}{\ensuremath{\mathbb R}}
\newcommand{\NN}{\ensuremath{\mathbb N}}
\newcommand{\ZZ}{\ensuremath{\mathbb Z}}

% feel free to add more shortcuts around here

\begin{document}
% replace with your name, but otherwise leave this header alone, from here ...
\small
\noindent \textsc{Math 401: Homework Assignment 2} \hfill Christopher Munoz

\normalsize
\bigskip
% ... to here

\setcounter{problem}{5}



\begin{problem} % Problem 6
    $$\bigcap_{n=1}^\infty (0,1/n)=\emptyset.$$
\end{problem}

% PLEASE READ:  Some suggestions for the proof:  Give the set a name, e.g. $S = \bigcap_{n=1}^\infty (0,1/n)$.  Then show by contradiction that $x\in S$ is false for every $x\in\RR$.  What property of the real numbers helps to show that, however small, a positive number is not in $S$?

\begin{proof}
	Let $S = \bigcap\limits_{i=1}^{\infty} (0,1/n) = \emptyset$. Let $x \in \RR$. Consider the following 3 cases. \newline
	\textbf{Case 1: } Suppose $x \leq 0$, then $x \notin S$ as $x \notin (0,1)$. \newline
	\textbf{Case 2: } Suppose $x \geq 0$, then $x \notin S$ as $x \notin (0,1)$. \newline
	\textbf{Case 3: } Suppose $0 < x < 1$, Choose $n \in \NN$ so that $n > \frac{1}{x}$. Then $x > \frac{1}{n}$. so $x \in (0, \frac{1}{n})$, thus $x \notin S$
	\newline
	These cases show that an arbitrary $x \in R$ is not in $S$.
\end{proof}


\begin{problem} % Problem 7
Given a function $f$ and a subset $A$ of its domain, consider the image $f(A) = \{f(x) : x \in A\}$.

\begin{itemize}
\item[(a)] An example of a function $f$, and two subsets $A,B$ of the domain of $f$, for which $f(A \cap B) \neq f(A) \cap f(B)$ is

% PUT EXAMPLE HERE.  When you give an example you need to clearly say what $f$ is as a function, including its precise domain.  That domain must include $A$ and $B$ as subsets.  Of course you must be precise about the sets $f(A),f(B),f(A\cap B)$ too.
	$$f(x) = |x|$$
		where set $A$ is a subset of the domain defined by $A = \{x \in \RR \mid 0 < x\}$ and where set $B$ is a subset of the domain defined by $B = \{x \in \RR \mid x \geq 0\}$.
% IN PART (b), FORM A CONJECTURE, AND THEN PROVE AS A PROPOSITION, A RELATIONSHIP BETWEEN  f(A \cup B) AND f(A) \cup f(B)
\item[(b)] If $A$, $B$ are subsets of the domain of $f$ then $f(A \cup B)$ IS RELATED IN SOME WAY TO $f(A) \cup f(B)$.

\begin{proof}
% FILL IN THE PROOF
\end{proof}
\end{itemize}
\end{problem}


\begin{problem} % Problem 8
If $a\in\RR$ is an upper bound for $A\subset \RR$, and if $a$ is also an element of $A$, then $a = \sup A$.
\end{problem}

\begin{proof}
% FILL IN THE PROOF
\end{proof}


\begin{problem} % Problem 9

% FOR EACH PART, FILL IN THE INFIMA AND SUPREMA.  YOU DO NOT HAVE TO PROVE ANYTHING.

\begin{enumerate}[(a)]
\item Let $A=\{m/n: \text{ $m,n\in\NN$ with $m<n$} \}$.  Then $\inf A = $ and $\sup A = $.
\item Let $B=\{(-1)^m/n: n,m\in\NN\}$.  Then $\inf B = $ and $\sup B = $.
\item Let $C=\{n/(3n+1): n\in\NN\}$.  Then $\inf C = $ and $\sup C = $.
\item Let $D=\{m/(m+n):m,n\in\NN\}$.  Then $\inf D = $ and $\sup D = $.
\end{enumerate}
\end{problem}


\begin{problem} % Problem 10

% PLEASE READ:  Decide if the following statements are true.  Give a short proof for the true statements and a counterexample for the false statements.

\begin{enumerate}[(a)]
\item If $A$ and $B$ are nonempty, bounded, and satisfy $A\subseteq B$
then $\sup A\le \sup B$.

% UNCOMMENT AND FILL-IN ONE OF THESE
% \begin{proof}  \end{proof}
% \bigskip This is false because

\item If $\sup A< \inf B$ for nonempty sets $A$ and $B$, then there exists
$c\in\RR$ such that $a<c<b$ for all $a\in A$ and $b\in B$.

% UNCOMMENT AND FILL-IN ONE OF THESE
% \begin{proof}  \end{proof}
% \bigskip This is false because

\item If there exists $c\in\RR$ satisfying $a<c<b$ for all
$a\in A$ and $b\in B$ then $\sup A< \inf B$. 

% UNCOMMENT AND FILL-IN ONE OF THESE
% \begin{proof}  \end{proof}
% \bigskip This is false because

\end{enumerate}
\end{problem}


\begin{problem} % Problem 11
\newcommand{\II}{\mathbb{I}}
Denote the irrational numbers by $\II = \RR \setminus \QQ$.
\begin{enumerate}[(a)]
\item If $a,b\in\QQ$ then $ab\in\QQ$ and $a+b\in\QQ$.

\begin{proof}
% FILL IN THE PROOF
\end{proof}

\item If $a\in\QQ$ and $t\in\II$ then $a+t\in\II$.  If also $a\neq 0$ then $at\in\II$.

\begin{proof}
% FILL IN THE PROOF
\end{proof}

\item Suppose $s,t\in\II$.  Then PROPOSITION ABOUT WHETHER $st$ AND $s+t$ ARE EITHER RATIONAL OR IRRATIONAL IN GENERAL.

% PLEASE READ: Part (a) says that the rational numbers are closed under multiplication and addition.  What can be said about $st$ and $s+t$ when $s,t\in\II$?  Formulate a conjecture and/or example and prove it.
\end{enumerate}
\end{problem}


\begin{problem} % Problem 12
For all $n\in\NN$, $2^n \ge n$.
\end{problem}

\begin{proof}
% FILL IN THE PROOF: Maybe one could just say ``this is obvious.''  Don't do that.  The careful way to prove this infinite collection of propositions (i.e. 2^1 \ge 1, 2^2 \ge 2, 2^3 \ge 3, ...) is to use induction.
\end{proof}


\begin{problem} % Problem 13
Let $y_1=6$ and, for each $n\in\NN$, let $y_{n+1} = (2 y_n - 6) / 3$.
 
\begin{enumerate}[(a)]
\item For all $n\in \NN$, $y_n \ge -6$.

\begin{proof}
% FILL IN THE PROOF.  USE INDUCTION.
\end{proof}

\item The sequence $(y_1,y_2,y_3,\dots)$ is decreasing.

\begin{proof}
% FILL IN THE PROOF.  USE INDUCTION.
\end{proof}
\end{enumerate}
\end{problem}

\end{document}
