% <-- a percent symbol indicates a comment which does not affect the output of LaTeX
% you can leave the preamble alone, from here ...
\documentclass[12pt]{article}

\usepackage{amssymb,amsmath,amsthm}
\usepackage[top=1in, bottom=1in, left=1.25in, right=1.25in]{geometry}
\usepackage{enumitem,palatino}
\usepackage[final]{graphicx}
\usepackage[colorlinks=true,citecolor=blue,linkcolor=red,urlcolor=blue]{hyperref}

\newtheorem{problem}{Problem}
% ... to here

% shortcuts for blackboard bold number sets (reals, integers, etc.)
\newcommand{\QQ}{\ensuremath{\mathbb Q}}
\newcommand{\RR}{\ensuremath{\mathbb R}}
\newcommand{\NN}{\ensuremath{\mathbb N}}
\newcommand{\ZZ}{\ensuremath{\mathbb Z}}

% feel free to add more shortcuts around here

\begin{document}
% replace with your name, but otherwise leave this header alone, from here ...
\small
\noindent \textsc{Math 401: Homework Assignment 1} \hfill Christopher Munoz

\noindent \hfill \normalsize
\bigskip
% ... to here

\setcounter{problem}{0} % <-- set to one less than the first problem number


\begin{problem} % Problem 1
There is no rational number whose square is $2$. \end{problem}

% PLEASE READ:  The first proof in Abbott's book, of Theorem 1.1.1 on page 1, is quite long because it is really a combination of commentary and a proof.  In this course, and in the rest of mathematics, you will instead want to write proofs that contain only the logic and not the commentary.  So please write a better proof than Abbott did, by extracting just the actual mathematical steps.  I have suggested a first two sentences in the proof.  I believe the whole proof can be written in about 6 sentences in total.

\begin{proof}
Assume, for contradiction, that there exist integers $p$ and $q$ satisfying
	$$\frac{p}{q} = \sqrt{2},$$
where $p/q$ is a rational number in lowest terms.  By squaring, this is the same as $\frac{p^2}{q^2} = 2$, and by clearing denominators it is the same as
	$$p^2 = 2 q^2.$$
% CONTINUE FROM HERE
	Thus $p^2$ is divisible by 2, an even number. This implies that $p$ is also divisible by 2 and can be expressed in the form $p = 2k$ for some $k \in \ZZ$. If we substitute the $p$ in $p^2 = 2q^2$ for $2k$, we get
	$$(2k)^2 = 4(k^2) = 2q^2$$
	Further reducing this gives us
	$$2(k^2) = q^2$$
	Therefore $p$ and $q$ are both even, contradicting the assumption that $\frac{p}{q}$ is in lowest terms.

\end{proof}


\begin{problem} % Problem 2, which is a simplification of Exercise 1.2.11

% PLEASE READ:  Fill-in the PUT NEGATION HERE blanks.  The idea is to negate the statement "for all x in S, P(x) is true" into "there exists x in S for which P(x) is false", and so on.  (Here P(x) is a proposition.)  This problem does not ask you to assess whether these propositions are true or false; please just negate for exercise.

\begin{itemize}
	\item[(a)] The negation of ``For all real numbers satisfying $a<b$, there exists $n\in\NN$ such that $a+(1/n)<b$'' is "There exists a real number $a,b$ satisfying $a<b$ such that for all $n \in \NN$, $a + (1/n) \geq b$.

% PUT NEGATION HERE, BETWEEN QUOTATION MARKS LIKE ``...''
\item[(b)] The negation of ``There exists a real number $x>0$ such that $x<1/n$ for all $n\in\NN$'' is "For all real numbers $x > 0$, there exists an $n \in \NN$ such that $x \geq 1/n$.

% PUT NEGATION HERE, BETWEEN QUOTATION MARKS LIKE ``...''
\item[(b)] The negation of ``Between every two distinct real numbers there is a rational number'' is "There exists $x,y \in \RR$, where $x \neq y$, such that there is no $n \in \QQ$ that satisfies $ x < n < y$.
% PUT NEGATION HERE, BETWEEN QUOTATION MARKS LIKE ``...''
\end{itemize}
\end{problem}



\begin{problem} % Problem 3, which modifies Exercise 1.2.6
Suppose $a$ and $b$ are real numbers.  Then
\begin{itemize}
\item[(a)] $|a-b| \le |a|+|b|$
	\begin{proof}
		\textbf{Case 1:} Suppose $a > b$, then $|a - b| = a - b \text{ (since $a-b > 0$)}$. If $a > 0$ is true, then $|a - b| = a - b = |a| - b$. Since $-b \leq |b|$(because $-b = |b|$ if $b$ is negative and $-b \leq b = |b|$ if $b$ is non-negative), we have $$|a - b| = |a| - b \leq |a| + |b|$$. On the other hand, if $a < 0$, then $$|a - b| = a - b \leq |a| - b \leq |a| + |b|$$ (since $a \leq |a|$, as before). \newline
		\textbf{Case 2:} Suppose $b > a$, then $|a - b| = b - a$ (since $b - a > 0$). If $b > 0$ is also true, then $|a-b| = b- a \leq b + |a|$ (since $-a \leq |a|$). \newline
		If $b \geq 0$, then $|b| = b$, so $$|a-b| = b - a \leq b + |a| = |a| + |b| $$ (since $-a \leq |a|$). \newline If $b < 0$, then since $b > a$, we have $a < b < 0$, so $|a| = -a$ and $|b| = -b$. Thus $$|a-b| = b - a = b + (-a) \leq -b + (-a) = |b| + |a| = |a|+|b|$$ or more briefly
		$$|a-b| \leq |a | + |b|$$. \newline
		\textbf{Case 3:} Suppose $a = b$, then $|a - b| = 0 \leq |a| + |b|$ Since absolute values are non-negative.
	\end{proof}
\item[(b)] $\big||a|-|b|\big| \le |a-b|$
	\begin{proof}
		Observe that $a = (a - b) + b$. It follows that
		$$	|a| = |(a - b) + b| \leq |a - b| + |b|$$
		by the triangle inequality. Subtracting $|b|$ from the right and left sides of the inequality gives us
		$|a| - |b| \leq |a - b|$. Likewise observe that $b = (b - a) - a$. It follows that
		$$ |b| = |(b-a) + a| \leq |b - a| + |a|$$ by the triangle inequality. Since $|b - a | = |a - b|$, subtracting $|a|$ from either sides of the inequality above gives us $|b| - |a| \leq |a - b|$. Since $||a|| - |b||$ is either $|a| - |b|$ or $|b| - |a|$, and both are at most $|a - b|$, it follows that $||a| - |b|| \leq |a - b|$.
	\end{proof}
\end{itemize}
\end{problem}

% PLEASE READ:  You can prove these inequalities by considering cases where $a$ and $b$ have the same and different signs, and etc.  Or you can follow the approach and suggestions in the book's statement of Exercise 1.2.6 on page 12.  The inequalities are called "triangle inequalities" because of their generalization to vectors, and not because it is a sensible name for real numbers.


\begin{problem} % Problem 4, which is Exercise 1.2.8
Give an example of each, or state that it is impossible.
\begin{itemize}
\item[(a)] $f:\NN\to\NN$ that is one-to-one but not onto.

	\textbf{My Answer: } The function $f(n) = 2n$ is a mapping from $\NN \rightarrow \NN$ That is one-to-one since distinct $n$ produce distinct even numbers, but not onto because it misses odd numbers.
% EXAMPLE OR STATEMENT HERE
\item[(b)] $f:\NN\to\NN$ that is onto but not one-to-one.

	\textbf{My Answer: } The function $f(n) = \lfloor\frac{n+1}{2}\rfloor$ (For clarity's sake this is a floor function) is onto since every $m \in \NN$ is hit, but not one-to one because $f(1) = f(2) = 1 $. 
% EXAMPLE OR STATEMENT HERE
\item[(d)] $f:\NN\to\ZZ$ that is one-to-one and onto.

	\textbf{My Answer: } The piecewise function $$f(n) =\begin{cases} \frac{n}{2} & \text{If n is even} \\
	-\frac{n-1}{2} & \text{If n is odd}\end{cases} $$
		Is a mapping from $\NN \rightarrow \ZZ$ that is both one-to-one and onto since it uniquely maps even $n$ to non-negative integers and odd $n$ to negative integers, this covers all of $\ZZ$.
	

% EXAMPLE OR STATEMENT HERE
\end{itemize}
\end{problem}


% PLEASE READ:  Prove this by giving an example.

\begin{problem} % Problem 5, which is Exercise 1.2.4
There exists an infinite collection of sets $A_1,A_2,A_3,\dots$ with the properties that every $A_i$ has an infinite number of elements, and $A_i\cap A_j=\emptyset$ for all $i\ne j$, and $\bigcup_{i=1}^\infty A_i=\NN$. \end{problem}

\begin{proof}
	A natural number is k-almost-prime if it has $k$ prime factors where $k \in \NN$.
	Let $A_1 \subseteq \NN$ be the union of the sets $\{1\}$ and the set containing all k-almost-prime numbers $\{2,3,5,7,11,17, ...\}$, where $k = 1$ so that $A_1 = \{1,2,3,5,7,11,17,...\}$. Let $A_2 \subset \NN$ be the set containing all k-almost-prime numbers $\{4,6,9,10,14,15,21,22,...\}$, where $k = 2$. Let $A_3 \subseteq \NN$ be set containing all k-almost-prime numbers $\{8, 12, 18, 20, 27, 28, 30, … \}$, where $k = 3$. Similarly let $A_k \subseteq \NN$ be the set containing all k-almost-prime numbers for some $k \in \NN$. Suppose $m \in \NN$ and $m \in A_i \cap A_j$ for some $i, j \in \NN$. Note that by the uniqueness of prime decompositions, $m$ has a unique prime decomposition of a fixed length. Thus $i = j$. Since $A_n \subseteq \NN$ for all $n \in \NN$, it follows that $\cup_{n=1}^{\infty}A_n \subseteq \NN$. Similarly by fundamental theorem of arithmetic, $n$ has a unique prime decomposition of a fixed length $m$. Therefore $n \in A_m$ and $\NN \subseteq \cup_{n=1}^{\infty} A_n$. Thus $\cup_{n=1}^{\infty} A_n = \NN$.
\end{proof}

\end{document}
