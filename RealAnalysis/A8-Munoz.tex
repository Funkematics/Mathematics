% <-- a percent symbol indicates a comment which does not affect the output of LaTeX
% you can leave the preamble alone, from here ...
\documentclass[12pt]{article}

\usepackage{amssymb,amsmath,amsthm}
\usepackage[top=1in, bottom=1in, left=1.25in, right=1.25in]{geometry}
\usepackage{enumitem,palatino}
\usepackage[final]{graphicx}
\usepackage[colorlinks=true,citecolor=blue,linkcolor=red,urlcolor=blue]{hyperref}

\newtheorem{problem}{Problem}
% ... to here

% shortcuts for blackboard bold number sets (reals, integers, etc.)
\newcommand{\II}{\ensuremath{\mathbb I}}
\newcommand{\NN}{\ensuremath{\mathbb N}}
\newcommand{\QQ}{\ensuremath{\mathbb Q}}
\newcommand{\RR}{\ensuremath{\mathbb R}}
\newcommand{\ZZ}{\ensuremath{\mathbb Z}}

\newcommand{\eps}{\ensuremath{\epsilon}}
\newcommand{\ds}{\displaystyle}

% feel free to add more shortcuts here


\begin{document}
% replace with your name, but otherwise leave this header alone, from here ...
\small
\noindent \textsc{Math 401: Homework Assignment 8} \hfill Christopher Munoz

\normalsize
\bigskip
% ... to here

\setcounter{problem}{50}


\begin{problem} % Problem 51
If $K\subset\RR$ is compact and nonempty, then $\sup K$ and $\inf K$ both exist and are elements of $K$.
\end{problem}

\begin{proof}
We want to show that $\sup K$ and $\inf K$ exist, then we will show that they are elements of $K$.
Recall that a set is compact if it is closed and bounded, meaning that the set $K$ contains all its limit points, is non-empty and is bounded both above and below. 
Since $K$ is nonempty and bounded, there exist real numbers $m$ and $M$ such that $m \leq x \leq M$ for all $x \in K$. Thus $K$ is bounded above and below. By the Axiom of Completeness, $\sup K$ and $\inf K$ both exist. Now show that $\sup K \in K$.
\newline\newline\noindent
Let $s = \sup K$. For each $n \in \NN$, since $s$ is the least upper bound, $s - \frac{1}{n}$ is not an upper bound of $K$. Therefore, there exists $x_n \in K$ such that 
$$s - \frac{1}{n} < x_n \leq s$$
this gives $0 \leq s - x_n < \frac{1}{n}$, so $|x_n - s| < \frac{1}{n}$.
\newline
Given $\varepsilon > 0$, by the Archimedean Property, there exists $N \in \NN$ such that $\frac{1}{N} < \varepsilon$. For all $n \geq N$:
$$|x_n - s| < \frac{1}{n} \leq \frac{1}{N} < \varepsilon$$

Therefore $(x_n) \to s$. Since $x_n \in K$ for all $n$ and $K$ is closed, we have $s \in K$.
By the same argument (using $\inf K + \frac{1}{n}$ instead), $\inf K \in K$.
Therefore $\sup K$ and $\inf K$ both exist and are elements of $K$.
\end{proof}


\begin{problem} % Problem 52
What of the following sets are compact?  For those that are compact, give a brief justification.  For those that are not, show how the book's definition of compact (Definition 3.3.1) breaks down.  That is, give an example sequence in the set that does not contain a subsequence which converges to a point in the set.
\end{problem}

% PLEASE READ: Please read and think about Section 3.3.

\renewcommand{\labelenumi}{(\alph{enumi})}
\begin{enumerate}
\item $\NN$

This is not compact because $\NN$ is unbounded. This means that all sequences and subsequences of the set $\NN$ diverge. 
\item $\QQ\cap [0,1]$

  This is not compact because $\QQ \cap [0,1]$ does not contain all its limit points, we can construct a sequence of rational numbers converging to $1/\pi$ since this is an irrational number, it is not in the set.
\item The Cantor set $C$

The Cantor set is closed, since it consists of complements of the union of open sets, which is closed and is bounded by 0 and 1 which contained in the set. By Heine-Borel C is compact. 
\item $\ds \left\{1+\frac{1}{2^2}+\frac{1}{3^2}+\dots+\frac{1}{n^2}\,:\,n\in\NN\right\}$

  We should note that the set is a p-series, with the value for $p=2$. This sum converges to $\frac{\pi^2}{6}$. Consider the range of the sequence of partial sum, if $(s_n) = \left\{1+\frac{1}{2^2}+\frac{1}{3^2}+\dots+\frac{1}{n^2}\,:\,n\in\NN\right\}$, we define it as the set $S = \{s_1, s_2, s_3, \cdots\, s_n\}$. We observe that $\frac{\pi^2}{6}$ is not in $S$ but that $S$ approaches $\frac{\pi^2}{6}$ as $n$ goes to infinity, this means that $S$ does not contain all its limit points and thus $S$ is not compact. 
\item $\ds \left\{1,\frac{1}{2},\frac{2}{3},\frac{4}{5},\dots\right\}$

  Observe that this sequence is equivalent to $S = \{1\} \cup \{n/(n+1) : n \geq 1\}$. We note that $\{n/(n+1) : n \geq 1\}$ with its bounds defined by $[0, 1)$ converges to 1 as $n$ approaches infinity , all of its subsequences also converge to $\{1\}$ similarly. Normally this set would not be closed since it wouldn't contain all its limit point, but since $\{1\}$ is unioned to this set, it contains all its limit points and the bounds are defined by $[0,1]$. Thus the set is compact. 
\end{enumerate}


\begin{problem} % Problem 53
If a set $K \subset \RR$ is closed and bounded, then it is compact.
\end{problem}

% PLEASE READ: This asks you to prove the part of Theorem 3.3.4 that is not shown in the textbook.

\begin{proof}
Since $K$ is closed and bounded, we can use Bolzano-Weierstrass to construct a sequence in $K$ in which its subsequence converges.
\newline\newline\noindent
Let $(x_n)$ be an arbitrary sequence in $K$. Since $K$ is bounded, there exists $a,b \in \RR$ such that $K \subseteq [a,b]$. Thus $(x_n)$ is a bounded
sequence. By the Bolzano-Weierstrass Theorem, there exists a subsequence $(x_{n_k})$ and $L \in \RR$ such that $x_{n_k} \to L$. Since $K$ is closed it follows that $L \in K$. Thus $(x_n)$ has a subsequence converging to a point in $K$, so $K$ is compact.
\end{proof}


\begin{problem} % Problem 54
Decide whether the following propositions are true or false.  If the claim is valid, supply a short proof.  If the claim is false, provide a counterexample.
\end{problem}

% PLEASE READ: Please read and think about Section 3.3.  Each time you introduce a proof, use a LaTeX proof environment (\begin{proof} ... \end{proof}).

\renewcommand{\labelenumi}{(\alph{enumi})}
\begin{enumerate}
\item The arbitrary union of compact sets is compact.

False: Since we are speaking of arbitrary union, suppose an infinite union of compact sets such as $K_n = [n, n+1]$ for $n \in \NN$.
Each of the $K_n$ are closed and bounded and thus compact, if we union an infinite amount of these $K_n$ we get an unbounded set and thus not compact.
\item The arbitrary intersection of compact sets is compact.

True:
\begin{proof}
  Let $\{K_\alpha\}_{\alpha \in I}$ be an arbitrary collection of compact sets, and let $K = \bigcap_{\alpha \in I} K_\alpha.$
  We want to show that $K$ is compact, so we want to show $K$ is closed and bounded. We start by showing $K$ is closed:

  Each $K_\alpha$ is compact, hence by Heine-Borel each $K_\alpha$ is closed. Because an arbitrary intersection of closed sets is closed, $K$ is closed.

  Now we show that $K$ is bounded:

  Choose any $K_\alpha$ in $K$, we denote this particular $K$ with index $\alpha' \in I$. Then
  
  $$K = \bigcap_{\alpha \in I} K_\alpha \subseteq K_{\alpha'}$$

  Since $K_{\alpha'}$ is compact, it follows that $K_{\alpha'}$ is bounded. Since any subset of a bounded set is bounded that means $K$ is bounded. Since $K$ is closed and bounded, $K$ is compact by Heine-Borel.
\end{proof}
\item Let $A$ be arbitrary, and let $K$ be compact.  Then $A\cap K$ is compact.

  False: Since $A$ is arbitrary, let $A = \QQ$ and let $K = [0,4]$, the intersection $A\cap K$ would consist of every rational number between $0$ and $4$. We can construct a sequence of rational numbers in the set that converge to $\pi$ via decimal expansion $(\frac{31}{10}, \frac{314}{100}, \frac{3141}{1000}, \frac{31415}{10000}, \cdots)$. But $\pi$ is not in the set and so $A \cap K$ does not contain all its limit points. Thus $A \cap K$ is not closed let alone compact.
\item If $F_1 \supseteq F_2 \supseteq F_3 \supseteq \dots$ is a nested sequence of nonempty closed sets then the intersection $\ds F = \bigcap_{n=1}^\infty F_n$ is nonempty.

False:
We can let $F_n = [n, \infty)$ for $n \in \NN$. These are closed sets and we can verify by considering the complement of these sets $F_n^c = (-\infty,n)$ which are clearly open. Explicitly writing this out we get
$$F =\bigcap_{n=1}^\infty F_n = [1,\infty) \supseteq [2,\infty) \supseteq [3, \infty) \supseteq \cdots \supseteq [n, \infty)$$
Since the sets move infinitely rightward, there is no common value in any of the $F_n$ so the set is empty.
\end{enumerate}


\begin{problem} % Problem 55
Let $A$ and $B$ be nonempty subsets of $\RR$.  If there exist disjoint open sets $U$ and $V$ with $A\subseteq U$ and $B\subseteq V$, then $A$ and $B$ are separated.
\end{problem}

% PLEASE READ: 

\begin{proof}
  We want to show $A$ and $B$ are separated, that is $A \cap \overline{B} = \emptyset$ and $\overline{A} \cap B = \emptyset$.
\newline\noindent\newline
  For the sake of contradiction suppose $A \cap \overline{B} \neq \emptyset$. Then there is some $x \in A \cap \overline{B}$. It follows from this that if $x \in A$ and $A \subseteq U$ that $x \in U$. Since $U$ is open and $x \in U$, $x$ has an $\eps$-neighborhood contained in $U$: there exists an $\eps > 0$ such that $N_\eps(x) \subseteq U$. Now since $x \in \overline{B}$ and $\overline{B}$ is closed, $\overline{B}$ contains all its limit points meaning either $x \in B$ or $x$ is a limit point of $B$.
  \newline
  \newline
  \textbf{Case 1:} Suppose $x \in B$. Then $x \in B \subseteq V$, so $x \in V$. This is a contradiction we assumed $U$ and $V$ are disjoint and so $U \cap V = \emptyset$.
  \newline\newline
  \textbf{Case 2:} Suppose $x$ is a limit point of $B$. Then every $\eps$-neighborhood of $x$ contains a point of $B$. So $N_\eps(x)$ contains some point $b \in B$. Since $b \in B$ and $B \subseteq V$, it follows that $b \in V$. But since $b \in N_\eps \subseteq U$, that means $b$ is in $U$ and $V$ so $b \in U \cap V \neq \emptyset$. A contradiction.
\newline\noindent\newline
  Thus it must be the case that $A \cap \overline{B} = \emptyset$. Similarly by symmetry, the same argument shows that $\overline{A} \cap B = \emptyset$. Therefore $A$ and $B$ are separated.

\end{proof}


\begin{problem} % Problem 56
A set $E\subset \RR$ is \emph{totally disconnected} if, given any two distinct points $x,y\in E$, there exist separated sets $A$ and $B$ with $x\in A$ and $y\in B$ and $E=A\cup B$.

The Cantor set $C$ is totally disconnected.
\end{problem}

% PLEASE READ: Prove that the Cantor set $C$ is totally disconnected by the following steps:
% 1. In section 3.1, $C$ is defined as an intersection of closed sets $C_n$, namely $C = \bigcap_{n=0}^\infty C_n$.  As the first steps of your proof, recall how $C_n$ are constructed as a finite union of disjoint closed intervals, and recall this intersection fact.
% 2. Given $x,y \in C$, with $x<y$, set $\epsilon=y-x$.  Explain why there must exist an $N$ large enough so that it is impossible for $x$ and $y$ to both belong to the same closed interval within $C_N$.
% 3. Now argue that $C$ is totally disconnected.

\begin{proof}
  Recall that the Cantor set is defined by
  $$C = \bigcap_{n=0}^\infty C_n$$
  such that
  \begin{align*}
    C_0 &= [0,1] \\
    C_1 &= [0, \frac{1}{3}] \cup [\frac{2}{3}, 1] \\
    C_2 &= [0,\frac{1}{9}] \cup [\frac{2}{9}, \frac{1}{3}] \cup [\frac{2}{3}, \frac{7}{9}] \cup [\frac{8}{9}, 1] \\
    \vdots \\
    C_n &= [0, \frac{1}{3^n}] \cup \cdots \cup [\frac{(3^n-1)}{3^n}, 1]
  \end{align*}
  Let $x,y \in C$ with $x < y$. We want to show there exists separated sets $A$ and $B$ with $x \in A$, $y \in B$ and $C = A \cup B$.
  Let $\eps = y - x > 0$. At the $n^{th}$ iteration of our Cantor set, each interval in $C_n$ has a length $\frac{1}{3^n}$. Since the $\lim_{n\to\infty} \frac{1}{3^n} = 0$, there exists an $N \in \NN$ such that:
  $$\frac{1}{3^N} < \eps = y - x$$
  Suppose for the sake of contradiction that $x$ and $y$ are in the same interval $I \subseteq C_N$.We define $\ell(I)$ as the length of $I$. Since $I$ is an interval containing both $x$ and $y$ and $x < y$, the interval $I$ must contain the segment $[x,y]$. Therefore the length of $I$(or $\ell(I)$) must be at least
  $$\ell(I) \geq y - x $$
  But every interval in $C_N$ has a length of exactly $\frac{1}{3^N}$, so :
  $$\ell(I) = \frac{1}{3^N} < y - x$$
  This is a contradiction, therefore $x$ and $y$ are in different intervals of $C_N$
  Since $x$ and $y$ are in different intervals of $C_N$, there exists at least one removed open interval $(a,b)$ (a gap from the construction) with $x < a < b < y$.
  
  Define:
  \begin{align*}
    A &= C \cap (-\infty, a] \\
    B &= C \cap [b, \infty)
  \end{align*} 
  Then $x \in A$ (since $x < a$), $y \in B$ (since $y > b$), and $C = A \cup B$ (since the gap $(a,b)$ was removed from the Cantor set).
  \newline
  To show $A$ and $B$ are separated, note that $A \subseteq (-\infty, a]$ and $B \subseteq [b, \infty)$ where $a < b$. Since $\overline{A} \subseteq \overline{(-\infty, a]} = (-\infty, a]$ and $\overline{B} \subseteq \overline{[b, \infty)} = [b, \infty)$, we have:
  \begin{align*}
    A \cap \overline{B} &\subseteq (-\infty, a] \cap [b, \infty) = \emptyset \\
    \overline{A} \cap B &\subseteq (-\infty, a] \cap [b, \infty) = \emptyset
  \end{align*}
  
  Therefore $A$ and $B$ are separated, and $C$ is totally disconnected.
\end{proof}


\begin{problem} % Problem 57
For each stated limit, find the largest possible $\delta$-neighborhood that is a proper response to the given $\eps$ challenge.  Note that $[[x]]$ denotes the greatest integer which is less than or equal to $x$.
\end{problem}

% PLEASE READ: The language of "\epsilon challenge" is briefly explained at the beginning of Section 4.2, but it should be clear from the meaning of the functional limit.

\renewcommand{\labelenumi}{(\alph{enumi})}
\begin{enumerate}
\item $\ds \lim_{x\to 3} 5x-6 = 9$, where $\eps=1$

  \begin{align*}
    |f(x) - L | &= |(5x-6) - 9| < 1 \\
      &= |5x -15| < 1 \\
      &= 5|x - 3| < 1 \\
      &= |x - 3| < \frac{1}{5}
  \end{align*}
  So our $\delta$-neighborhood is $|x - 3| < \frac{1}{5}$.
\item $\ds \lim_{x\to 4} \sqrt{x} = 2$, where $\eps=0.5$

\begin{align*}
    |\sqrt{x} - 2| &< 0.5 \\
    -0.5 < \sqrt{x} - 2 &< 0.5 \\
    1.5 < \sqrt{x} &< 2.5 \\
    2.25 < x &< 6.25 \\
    -1.75 < x - 4 &< 2.25
  \end{align*}
  So $|x - 4| < \min\{1.75, 2.25\} = 1.75$, thus $\delta = \frac{7}{4}$
\item $\ds \lim_{x\to \pi} [[x]] = 3$, where $\eps=0.5$

 \begin{align*}
    |[[x]] - 3| &< 0.5 \\
    2.5 < [[x]] &< 3.5
  \end{align*}
  Since $[[x]]$ is an integer, we need $[[x]] = 3$, which occurs when $x \in [3, 4)$.
  
  For $|x - \pi| < \delta$ to imply $x \in [3, 4)$, we need:
  $$\delta = \min\{\pi - 3, 4 - \pi\} = \pi - 3$$
  
  Thus $\delta = \pi - 3$.
\end{enumerate}


\begin{problem} % Problem 58
Use the definition of functional limit in the textbook (Definition 4.2.1) to prove the following limit statements.
\end{problem}

\renewcommand{\labelenumi}{(\alph{enumi})}
\begin{enumerate}
\item $\ds \lim_{x\to 2} 3x+4 = 10$

\begin{proof}
Let $\eps > 0$. We want to find $\delta > 0$ such that 
$$0 < |x - 2| < \delta \implies |f(x) - 10| < \eps$$
Observe that 
\begin{align*}
  |f(x) - 10| &= |(3x + 4) - 10| < \eps\\
              &= |3x - 6| < \eps \\
              &= 3|x-2| < \eps\\
              &= |x - 2| < \frac{\eps}{3}
\end{align*}
Choose $\delta = \frac{\eps}{3}$. Then if $0 < |x-2| < \delta$, we have 
$$|f(x) - 10| = 3|x-2| < 3 \cdot \frac{\eps}{3} = \eps$$
Thus $\lim_{x \to 2} 3x+4 = 10$
\end{proof}

\item $\ds \lim_{x\to 2} x^2 + x - 1 = 5$

\begin{proof}
Let $\eps > 0$. We want to find $\delta > 0$ such that
$$0 < |x - 2| < \delta \implies |f(x) - 5| < \eps$$
Observe that
\begin{align*}
  |f(x) - 5| &= |(x^2 + x - 1) - 5| \\
              &= |x^2 + x - 6| \\
              &= |(x+3)(x-2)| \\
              &= |x+3| \cdot |x-2|
\end{align*}
Assume $\delta \leq 1$. Then $|x-2| < 1$ implies $1 < x < 3$, so $4 < x+3 < 6$, thus $|x+3| < 6$.
We want $|x+3| \cdot |x-2| < \eps$. Since $|x+3| < 6$, we need $|x-2| < \frac{\eps}{6}$.

Choose $\delta = \min\left\{1, \frac{\eps}{6}\right\}$. Then if $0 < |x-2| < \delta$, we have
$$|f(x) - 5| = |x+3| \cdot |x-2| < 6 \cdot \frac{\eps}{6} = \eps$$
Thus $\lim_{x \to 2} x^2 + x - 1 = 5$.
\end{proof}

\item $\ds \lim_{x\to 3} \frac{1}{x} = \frac{1}{3}$

\begin{proof}
Let $\eps > 0$. We want to find $\delta > 0$ such that
$$0 < |x - 3| < \delta \implies \left|f(x) - \frac{1}{3}\right| < \eps$$
Observe that
\begin{align*}
  \left|\frac{1}{x} - \frac{1}{3}\right| &= \left|\frac{3-x}{3x}\right| \\
                                           &= \frac{|x-3|}{3|x|}
\end{align*}
Assume $\delta \leq 1$. Then $|x-3| < 1$ implies $2 < x < 4$, so $|x| > 2$, thus $\frac{1}{|x|} < \frac{1}{2}$.
We want $\frac{|x-3|}{3|x|} < \eps$. Since $\frac{1}{|x|} < \frac{1}{2}$, we need $\frac{|x-3|}{6} < \eps$, so $|x-3| < 6\eps$.

Choose $\delta = \min\{1, 6\eps\}$. Then if $0 < |x-3| < \delta$, we have
$$\left|\frac{1}{x} - \frac{1}{3}\right| = \frac{|x-3|}{3|x|} < \frac{|x-3|}{6} < \frac{6\eps}{6} = \eps$$
Thus $\lim_{x \to 3} \frac{1}{x} = \frac{1}{3}$.
\end{proof}

\item $\ds \lim_{x\to 3} \frac{x^2-9}{x-3} = 6$

% PLEASE READ:  Justify any algebra which is essential to proving the limit.  What is the domain of the function f(x)=(x^2-9)/(x-3)?

\begin{proof}
  Note that the domain of $f(x) = \frac{x^2-9}{x-3}$ is $\{x \in \RR : x \neq 3\}$.

For $x \neq 3$, we have
$$f(x) = \frac{x^2-9}{x-3} = \frac{(x-3)(x+3)}{x-3} = x+3$$

Let $\eps > 0$. We want to find $\delta > 0$ such that
$$0 < |x - 3| < \delta \implies |f(x) - 6| < \eps$$
For $x \neq 3$, observe that
\begin{align*}
  |f(x) - 6| &= \left|\frac{x^2-9}{x-3} - 6\right| \\
              &= |x+3-6| \\
              &= |x-3|
\end{align*}
We want $|x-3| < \eps$.

Choose $\delta = \eps$. Then if $0 < |x-3| < \delta$ (which ensures $x \neq 3$), we have
$$|f(x) - 6| = |x-3| < \delta = \eps$$
Thus $\lim_{x \to 3} \frac{x^2-9}{x-3} = 6$.
\end{proof}

\end{enumerate}


\begin{problem} % Problem 59
Let $g:A\to\RR$ and assume that $f$ is a bounded function on $A$.  Assume $c$ is a limit point of $A$.  If $\ds \lim_{x\to c} g(x)=0$ then $\ds \lim_{x\to c} f(x) g(x)=0$.
\end{problem}

\begin{proof}
Since $f$ is bounded on $A$, by definition there exists $M > 0$ such that $|f(x)| \leq M$ for all $x \in A$.

Let $\eps > 0$. We want to find $\delta > 0$ such that for all $x \in A$
$$0 < |x - c| < \delta \implies |f(x)g(x) - 0| < \eps$$
Since $\lim_{x\to c} g(x) = 0$ for an $\eps' = \frac{\eps}{M} > 0$. there exists $\delta > 0$ such that for all $x \in A$ we have
$$0 < |x-c| < \delta \implies |g(x) - 0| < \frac{\eps}{M}$$

Now, for all $x \in A$ with $0 < | x - c| < \delta$ we have
\begin{align*}
  |f(x)g(x) - 0| &= |f(x)g(x)| \\
                  &= |f(x)| \cdot |g(x)| \\
                  &\leq M \cdot |g(x)| \\
                  &< M \cdot \frac{\varepsilon}{M} \\
                  &= \varepsilon
\end{align*}
Thus $\lim_{x \to c} f(x)g(x) = 0$.

\end{proof}


\end{document}
