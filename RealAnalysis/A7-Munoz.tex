% <-- a percent symbol indicates a comment which does not affect the output of LaTeX
% you can leave the preamble alone, from here ...
\documentclass[12pt]{article}

\usepackage{amssymb,amsmath,amsthm}
\usepackage[top=1in, bottom=1in, left=1.25in, right=1.25in]{geometry}
\usepackage{enumerate,palatino}
\usepackage[final]{graphicx}
\usepackage[colorlinks=true,citecolor=blue,linkcolor=red,urlcolor=blue]{hyperref}

\newtheorem{problem}{Problem}
% ... to here

% shortcuts for blackboard bold number sets (reals, integers, etc.)
\newcommand{\II}{\ensuremath{\mathbb I}}
\newcommand{\NN}{\ensuremath{\mathbb N}}
\newcommand{\QQ}{\ensuremath{\mathbb Q}}
\newcommand{\RR}{\ensuremath{\mathbb R}}
\newcommand{\ZZ}{\ensuremath{\mathbb Z}}

\newcommand{\eps}{\ensuremath{\epsilon}}
\newcommand{\ds}{\displaystyle}

% feel free to add more shortcuts here


\begin{document}
% replace with your name, but otherwise leave this header alone, from here ...
\small
\noindent \textsc{Math 401: Homework Assignment 7} \hfill Christopher Munoz

\normalsize
\bigskip
% ... to here

\setcounter{problem}{43}

\begin{problem} % Problem 44
(Comparison Test)

\medskip
\noindent Assume $(a_k)$ and $(b_k)$ are sequences satisfying $0\le a_k \le b_k$ for all $k\in\NN$.

\renewcommand{\labelenumi}{\emph{(\roman{enumi})}}
\begin{enumerate}
\item If $\sum_{k=1}^\infty b_k$ converges then $\sum_{k=1}^\infty a_k$ converges.
\item If $\sum_{k=1}^\infty a_k$ diverges then $\sum_{k=1}^\infty b_k$ diverges.
\end{enumerate}
\end{problem}

% PLEASE READ: This is Theorem 2.7.4 in the textbook.  The textbook suggests a proof using the Cauchy Criterion for series, namely Theorem 2.7.2.  Please flesh this out, giving separate, complete proofs for (i) and (ii).
\iffalse
\begin{proof}
	\textbf{For (i):} Define the sequences of partial sums: 
	$$A_n = \sum_{k=1}^n a_k \quad \text{and} \quad B_n = \sum_{k=1}^n b_k$$
	Since $0 \leq a_k$ for all $k$, we have $A_{n+1} = A_n + a_{n+1} \geq A_n$ for all $n \in \NN$. This means our sequence of partial sums $A_n$ is monotone and increasing. Since $0 \leq a_k \leq b_k$ for all $k \in \NN$, we have 
	$$A_n = \sum_{k=1}^n a_k \leq \sum_{k=1}^n b_k = B_n $$
	for all $n \in \NN$. Since $\sum_{k=1}^n b_k$ converges, the sequence of partial sums $B_n$ converges to a finite limit we denote with $L_b$ and is thus bounded. By definition there exists an $M > 0$ such that $B_n \leq M$ for all $n$. Therefore
	$$A_n \leq B_n \leq M$$
	for all $n \in \NN$. Thus $A_n$ is monotone increasing and bounded above. By the Monotone Convergence Theorem, $A_n$ converges which means $\sum_{k=1}^\infty a_k$ converges. \newline \newline
	\textbf{For (ii):} Since this is the contrapositive statement of part $(i)$, it is logically equivalent and already proven, we are done.
\end{proof}
\fi
% The above prove I did before actually reading the comments. its commented out, I used the Monotone Convergence Theorem and kept it because I quite like this proof and it was so far in my comfort zone that it almost made me feel smart, proof using Cauchy Criterion is below
\begin{proof}
	\textbf{For (i)} Let $\eps > 0$, Since $\sum_{k=1}^\infty b_k$ converges, by the Cauchy criterion, there exists an $N \in \NN$ such that for all $n > m \geq N$:
	$$|\sum_{k=m+1}^n b_k| = |b_{m+1} + b_{m+2} + \cdots b_n| < \eps$$ 
	Since $ 0 \leq a_k \leq b_k$ for all $k$, for all $n > m \geq N$ we have:
	$$|a_{m+1} + a_{m+2} + \cdots a_n| \leq |b_{m+1} + b_{m+2} + \cdots b_n| < \eps$$ equivalently
	$$\sum_{k = m+1}^n a_k \leq \sum_{k = m+1}^n b_k < \eps$$ 
	Since $a_k \geq 0$, we have 
	$$|\sum_{k = m+1}^n a_k |= \sum_{k = m+1}^n a_k < \eps$$
	This satisfies the Cauchy criterion, thus $\sum_{k=1}^\infty a_k$ converges. \newline \newline
	\textbf{For (ii)} Since this is a contrapositive statement for (i), it is logically equivalent and we have proved above, we are done.
\end{proof}
\begin{problem} % Problem 45
(Alternating Series Test)

\medskip
\noindent Suppose $(a_n)$ is a \emph{nonnegative} sequence which satisfies
\renewcommand{\labelenumi}{\emph{(\roman{enumi})}}
\begin{enumerate}
\item $(a_n)$ is decreasing, and
\item $\ds \lim_{n\to\infty} a_n = 0$.
\end{enumerate}
Then the alternating series $\ds \sum_{n=1}^\infty (-1)^{n+1} a_n$ converges.
\end{problem}

% PLEASE READ: The Alternating Series Test is Theorem 2.7.7 in the textbook.  As the textbook points out in Exercise 2.7.1, you can build proofs based on any of three strategies:
% (a) proving that the partial sums form a Cauchy sequence and then using the Cauchy Criterion for sequences, or
% (b) building nested intervals and applying the Nested Interval Property to get a candidate limit and then proving that the partial sums converge to that candidate limit, or
% (c) considering the even-index and odd-index subsequences of the sequence of partial sums and showing that the Monotone Convergence Theorem gives candidate limits which you show are the same and are thus the limit of the partial sums.
% Instructions:  Prove using one of these strategies!

\begin{proof}
	Let $\eps > 0$. Since $\lim_{n \to \infty} a_n = 0$, there exists $N \in \NN$ such that for all $n \geq N$:
	$$a_n < \eps$$
	Let $n > m \geq N$. We want to show that $$|\sum_{k = m+1}^n(-1)^{k+1}a_k| < \eps$$
	The sum $\sum_{k = m+1}^n(-1)^{k+1}a_k$ is a finite alternating sum that starts with either $+a_{m+1}$ or $-a_{m+1}$. 

\noindent	Without loss of generality suppose the finite sum starts with $+a_{m+1}$ (the case with $-a_{m+1}$ is identical). Then:
	$$\sum_{k=m+1}^n (-1)^{k+1} a_k = a_{m+1} - a_{m+2} + a_{m+3} - a_{m+4} + \cdots $$
	Since $(a_n)$ is decreasing we can write this as:
	$$\sum_{k=m+1}^n (-1)^{k+1} a_k = (a_{m+1} - a_{m+2}) + (a_{m+3} - a_{m+4}) + \cdots $$
	Each pair satisfies $a_k - a_{k+1} \geq 0$ by the decreasing property. Therefore the sum is non-negative. Observe that we can also write
	$$a_{m+1} - a_{m+2} + a_{m+3} - a_{m+4} + \cdots = a_{m+1} - (a_{m+2} - a_{m+3}) - (a_{m+4} - a_{m+5}) - \cdots$$
	Since we are subtracting non-negative quantities from $a_{m+1}$, the sum is at most $a_{m+1}$. Thus:
	$$0 \leq \sum_{k=m+1}^{n} (-1)^{k+1} a_k \leq a_{m+1}$$
	Therefore
	$$| \sum_{k = m+1}^n (-1)^{k+1} a_k | \leq a_{m+1} < \eps$$
	Since $m \geq N$, we have $a_{m+1} < \eps$. By the Cauchy criterion, $\sum_{n=1}^\infty (-1)^{n+1} a_n$ converges.
\end{proof}


\begin{problem} % Problem 46
For each of the subsets of $\RR$ below, decide whether it is open, closed, or neither.  If a set is not open, find a point in the set for which there is no $\eps$-neighborhood contained in the set.  If a set is not closed, find a limit point that is not contained in the set.

\renewcommand{\labelenumi}{\emph{(\alph{enumi})}}
\begin{enumerate}
\item $\QQ$

	This is neither open or closed, take for example $0 \in \QQ$, the $\eps$-neighborhood $(0 - \eps, 0 + \eps)$ contains irrational numbers like $\eps/\sqrt{2}$. In this particular case $\sqrt{2}$ is a limit point and not in $\QQ$.

\item $\NN$

	This one is closed, not open. Take any number, suppose $1 \in \NN$. For any $\eps > 0$ the nighborhood $(1 - \eps, 1 + \eps)$ contains nothing and it has no limit point since every Natual number is an isolated point.

\item $\{x\in\RR\,:\,x \ne 0\}$

	This set is open, not closed. Consider the $\eps$-neighborhood for an $x \neq 0$. Choose $\eps = |x|/2$, then for the neighborhood $(x - \eps, x + \eps)$, $0$ will never be in this neighborhood, and since 0 is the limit point for this set, it is not closed.

\item $\{1 + 1/4 + 1/9 + \dots + 1/n^2\,:\,n\in \NN\}$

Neither open or closed, The smallest element is $1$ and neighborhoods go below $1$. It also is bounded and monotone decreasing to a limit point not in the set $\pi^2 / 6$.

\item $\{1 + 1/2 + 1/3 + \dots + 1/n \,:\,n \in \NN\}$

Closed, not open, the smallest element is $1$ and neighborhoods go below $1$. It is also a Harmonic series that diverges so no limit point exists.
% FILL IN
\end{enumerate}
\end{problem}


\begin{problem} % Problem 47
Let $A\subset \RR$ be nonempty and bounded above, and let $s=\sup A$.  Then
\renewcommand{\labelenumi}{\emph{(\roman{enumi})}}
\begin{enumerate}
\item $s \in \overline{A}$, but
\item if $A$ is open then $s \notin A$.
\end{enumerate}
\end{problem}

% PLEASE READ: 

\begin{proof}
	\textbf{For (i): } We show that $s \in \overline{A}$ directly. Let $\eps > 0$. Since $s = \sup A$, we know that $s$ is the least upper bound of $A$. This means $s - \eps$ is not an upper bound of $A$. Therefore there exists an $a \in A$ such that 
	$$s - \eps < a \leq s$$
	since $a \leq s < s + \eps$, we have $a \in (s - \eps, s + \eps)$.
	This shows that every $\eps$-neighborhood of $s$ contains at least one point of $A$. By definition of closure, this means $s \in \overline{A}$. \newline \newline
	\textbf{For (ii): } Suppose $A$ is open. We prove by contradiction that $s \notin A$. Assume $s \in A$. Since $A$ is open, there exists $\eps > 0$ such that
	$$(s - \eps, s + \eps) \subseteq A$$
	Choose $s+\frac{\eps}{2} \in A$. Since $s+\frac{\eps}{2} > s$, this contradicts the fact that $s = \sup A$ is an upper bound of $A$.
	Thus if $A$ is open, then $s \notin A$.
\end{proof}


\begin{problem} % Problem 48
Decide whether the following statements are true or false.  Provide proofs for those that are true, and counterexamples for those that are false.

\renewcommand{\labelenumi}{\emph{(\alph{enumi})}}
\begin{enumerate}
\item Every nonempty open set contains a rational number.

Assuming the nonempty open set is a subset of $\RR$ then true because of the density of $\QQ$ in $\RR$.
\begin{proof}
	Let $A \subseteq \RR$ be a nonempty open set. Since $A$ is nonempty, there exists $x \in A$.
	Since $A$ is open, there exists $\eps > 0$ such that $(x - \eps, x + \eps) \subseteq A$.
	By the density of $\QQ$ in $\RR$, there exists a rational number $r \in (x - \eps, x + \eps)$.
	Therefore, $r \in A$ and $r \in \QQ$, so $A$ contains a rational number.
\end{proof}

\item The Cantor set is closed.

	This one is true, we can  show this by considering the compliment of the Cantor set.
\begin{proof}
	The Cantor Set $C$ is constructed by starting with the interval $[0,1]$ and iteratively removing the open middle third of each remaining intervals. At each stage $n$, we remove a collection of open intervals like this:
	\begin{align*}
			C_0 &= [0,1]\\
			C_1 &= [0,1] \setminus (1/3, 2/3)\\
			C_2 &= C_1 \setminus \left[(1/9, 2/9) \cup (7/9, 8/9)\right]\\
			&\vdots
		\end{align*}
	The Cantor set is the intersection of all of these denoted as $C = \cap_{n=0}^\infty C_n$. The compliment of $C$ is
	$$\RR \setminus C = \RR \setminus \cap_{n=0}^\infty C_n = \cup_{n=0}^\infty (\RR \setminus C_n)$$
	Each $\RR \setminus C_n$ is a union of open intervals. Since a union of open sets is open, then the compliment of the Cantor set is open.
	Therefore the Cantor set is closed.
\end{proof}
\item If $A\subseteq \RR$ is an open set which contains every rational ($\QQ \subset A$) then $A=\RR$.
	\begin{proof}
This is true, proof omitted, something about density and connectedness.  
	\end{proof}
% PLEASE READ:  Part (c) here will require quite a bit of thought.

\end{enumerate}
\end{problem}


\begin{problem} % Problem 49
(De Morgan's Laws for arbitrary unions and intersections)

\medskip
\noindent Let $X$ be a set, which we call the universe set.  For any $A\subset X$ we write
    $$A^c = \{x\in X\,:\,x \notin A\}$$
for the complement set.  Also let $\Lambda$ be any set, which will be used as a set of indices.  Consider
	$$\mathcal{E} = \left\{E_\lambda\subset X \,:\,\lambda \in \Lambda \right\},$$
a collection of sets.  The following equalities hold:
\renewcommand{\labelenumi}{\emph{(\roman{enumi})}}
\begin{enumerate}
\item $\ds \left(\bigcup_{\lambda\in\Lambda} E_\lambda\right)^c = \bigcap_{\lambda\in\Lambda} E_\lambda^{\,\,c}$
\item $\ds \left(\bigcap_{\lambda\in\Lambda} E_\lambda\right)^c = \bigcup_{\lambda\in\Lambda} E_\lambda^{\,\,c}$
\end{enumerate}
\end{problem}

% PLEASE READ:  Basically, chase elements from left to right and back, for each equality?  The one thing you *do not* want to do is assume that \Lambda is either finite or countable.  The whole point here is that *arbitrary* collections of sets have this property.

\begin{proof}
	\textbf{For (i) :} We will prove $(\bigcup_{\lambda \in \Lambda} E_\lambda)^c = \bigcap_{\lambda \in \Lambda} E_\lambda^c$ directly.\newline

	\noindent$(\subseteq)$ direction: Let $x \in (\bigcup_{\lambda \in \Lambda} E_\lambda)^c$. Then by definition $x \notin \bigcup_{\lambda \in \Lambda} E_\lambda$. It follows that $x \notin E_\lambda$ for all $\lambda \in \Lambda$. Therefore $x \in E_\lambda^c$ for all $\lambda \in \Lambda$. By definition of intersection this means $x \in \bigcap_{\lambda \in \Lambda} E_\lambda^c$ for all $\lambda \in \Lambda$. 
	\newline
	\noindent $(\supseteq)$ direction: Let $x \in \bigcap_{\lambda \in \Lambda} E_\lambda^c$. Then $x \in E_\lambda^c$ for all $\lambda \in \Lambda$. This means $x \notin E_\lambda$ for all $\lambda \in \Lambda$. Therefore $x \notin \bigcup_{\lambda \in \Lambda} E_\lambda$. Thus $x \in (\bigcup_{\lambda \in \Lambda} E_\lambda)^c$.  \newline\newline
	Since both directions hold  $(\bigcup_{\lambda \in \Lambda} E_\lambda)^c = \bigcap_{\lambda \in \Lambda} E_\lambda^c$
\newline \newline
\textbf{For (ii):} We will prove $\left(\bigcap_{\lambda \in \Lambda} E_\lambda\right)^c = \bigcup_{\lambda \in \Lambda} E_\lambda^c$ directly.\newline

\noindent$(\subseteq)$ direction: Let $x \in \left(\bigcap_{\lambda \in \Lambda} E_\lambda\right)^c$. Then by definition $x \notin \bigcap_{\lambda \in \Lambda} E_\lambda$. It follows that there exists some $\lambda_0 \in \Lambda$ such that $x \notin E_{\lambda_0}$. Therefore $x \in E_{\lambda_0}^c$. Since $x$ is in at least one of the sets $E_\lambda^c$, by definition of union we have $x \in \bigcup_{\lambda \in \Lambda} E_\lambda^c$.
\newline

\noindent$(\supseteq)$ direction: Let $x \in \bigcup_{\lambda \in \Lambda} E_\lambda^c$. Then there exists some $\lambda_0 \in \Lambda$ such that $x \in E_{\lambda_0}^c$. This means $x \notin E_{\lambda_0}$. Since $x$ is not in all of the sets $E_\lambda$, we have $x \notin \bigcap_{\lambda \in \Lambda} E_\lambda$. Thus $x \in \left(\bigcap_{\lambda \in \Lambda} E_\lambda\right)^c$. \newline

\noindent Since both directions hold, $\left(\bigcap_{\lambda \in \Lambda} E_\lambda\right)^c = \bigcup_{\lambda \in \Lambda} E_\lambda^c$.
\end{proof}


\begin{problem} % Problem 50
If $A \subset \RR$ is both open and closed then either $A=\emptyset$ or $A=\RR$.
\end{problem}

% PLEASE READ: It is straightfoward to confirm that the empty set and R itself are both open and closed; say a bit about that first.  Then consider any other set A.  Now both A and A^c are nonempty, right?  Thus you can pick x \noeq y where x \in A and y \in A^c.  Consider the bounded interval [x,y], noting (x,y) is nonempty.  Find a point within that interval which must both be in A and A^c, a contradition.

\begin{proof}
We will begin by noting that the empty set $\emptyset$ is open because it vacuously satisfies the definition and it is closed because its compliment $\RR$ is open. Similarly $\RR$ is also both open and closed for vacuous reasons. 
\newline\newline
Suppose $A \subseteq \RR$ is both open and closed, and suppose for the sake of contradiction that $A \neq \emptyset$ and $A \neq \RR$. Then both $A$ and $A^c = \RR \setminus A$ are nonempty. Thus there exists $x \in A$ and $y \in A^c$ with $x \neq y$. Without loss of generality, suppose $x < y$. Consider the set:
	$$S = \{t \in [x,y] : t \in A\}$$
	Since $x \in A$, we have $x \in S$, so $S$ is nonempty. Also $S$ is bounded above by $y$. By completeness of the real numbers, the supremum $s = \sup S$ exists, and $x \leq s \leq y$. We will show that $s \in A$ and $s \in A^c$ leading to a contradiction. \newline\newline
	($s \in A$): Since $s = \sup S$ and $s \leq y$, for every $\eps > 0$, there exists $t \in S$ with $s - \eps < t \leq s$.
	Since $t \in S$, we have $t \in A$. Thus, every $\eps$-neighborhood of $s$ contains a point of $A$, so $s \in \overline{A}$.
	Since $A$ is closed, $\overline{A} = A$, so $s \in A$. \newline

	\noindent$(s \in A^c)$: Since $s = \sup S$ and $y \in A^c$, we have $s < y$ (this is because if $s = y$, then $y$ would be a limit point of $A$, and since $A$ is closed, $y \in A$, contradicting $y \in A^c$). Since $s < y$ and $s = \sup S$, for any $t \in (s, y]$, we have $t \notin S$, which means $t \notin A$, so $t \in A^c$. Thus, every neighborhood of $s$ of the form $(s - \eps, s + \eps)$ with small enough $\eps$ contains points of $A^c$, specifically points in $(s, s+\eps)$. This means $s \in \overline{A^c}$. Since $A^c$ is closed (because $A$ is open), we have $s \in A^c$. \newline 

	\noindent Thus $s \in A$ and $s \in A^c$, which isn't possible since $A$ and $A^c$ are disjoint, a contradiction.
	Therefore, either $A = \emptyset$ or $A = \RR$.
\end{proof}


\end{document}
