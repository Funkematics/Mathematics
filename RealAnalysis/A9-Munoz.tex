% <-- a percent symbol indicates a comment which does not affect the output of LaTeX
% you can leave the preamble alone, from here ...
\documentclass[12pt]{article}

\usepackage{amssymb,amsmath,amsthm}
\usepackage[top=1in, bottom=1in, left=1.25in, right=1.25in]{geometry}
\usepackage{enumitem,palatino}
\usepackage[final]{graphicx}
\usepackage[colorlinks=true,citecolor=blue,linkcolor=red,urlcolor=blue]{hyperref}

\newtheorem{problem}{Problem}
% ... to here

% shortcuts for blackboard bold number sets (reals, integers, etc.)
\newcommand{\II}{\ensuremath{\mathbb I}}
\newcommand{\NN}{\ensuremath{\mathbb N}}
\newcommand{\QQ}{\ensuremath{\mathbb Q}}
\newcommand{\RR}{\ensuremath{\mathbb R}}
\newcommand{\ZZ}{\ensuremath{\mathbb Z}}

\newcommand{\eps}{\ensuremath{\epsilon}}
\newcommand{\ds}{\displaystyle}

% feel free to add more shortcuts here


\begin{document}
% replace with your name, but otherwise leave this header alone, from here ...
\small
\noindent \textsc{Math 401: Homework Assignment 9} {\footnotesize [version 2]} \hfill Christopher Munoz

\normalsize
\bigskip
% ... to here

\setcounter{problem}{59}

\begin{problem} % Problem 60
Let $f,g,h$ satisfy $f(x)\le g(x)\le h(x)$ for all $x$ in some common domain $A$.  Assume $c$ is a limit point of $A$.  If $\lim_{x\to c} f(x)=L$ and $\lim_{x\to c} h(x)=L$ then $\lim_{x\to c} g(x)=L$.
\end{problem}

% PLEASE READ: This is a squeeze theorem, and it is Exercise 4.2.11 in the book.  Compare Exercise 2.3.3.

\begin{proof}
  Let $\eps > 0$.
  Since $\lim_{x \to c} f(x) = L$ and $\lim_{x \to c} h(x) = L$ then by definition of limit in the context of functions, for every $\eps > 0$, there exists $\delta_1 > 0$ and $\delta_2 > 0$ such that
  \begin{align*}
    0 < |x - c| < \delta_1 \Rightarrow L - \eps < f(x) < L + \eps\\
    0 < |x - c| < \delta_2 \Rightarrow L - \eps < h(x) < L + \eps
  \end{align*}
  for all $x \in A$. Given $f(x) \le g(x) \le h(x)$, if we let $\delta = \min\{\delta_1, \delta_2\}$, then for all $x \in A$ with $0 < |x - c| < \delta$ we have
  $$L - \eps < f(x) \le g(x) \le h(x) < L + \eps$$
  Therefore $L - \eps < g(x) < L + \eps$ or equivalently $|g(x) - L| < \eps$. Since $\eps$ was arbitrary we conclude that 
   $$\lim_{x \to c} g(x) = L.$$

\end{proof}


\begin{problem} % Problem 61
If $h:\RR \to \RR$ is a continuous function then the set $K=\{x\in \RR\,:\,h(x)=0\}$ is closed.
\end{problem}

% PLEASE READ: As expected, prove that K contains its limit points.  (Or prove that K^c is open.)

\begin{proof}
Recall that a set is closed if it contains all of its limit points. 

Let $c \in K$ be a limit point, since $K=\{x\in \RR\,:\,h(x)=0\}$, we want to show that $h(c) = 0$. Since $c$ is a limit point of $K$, then there exists a sequence $(x_n)$ in $K$ with $x_n \to c$. Since each $x_n \in K$, it follows that $h(x_n) = 0$ for all $n$. 

Since $h: \RR \to \RR$ is a continous function at $c$ and $x_n \to c$, we have 
$$\lim_{n \to \infty} h(x_n) = h(c)$$
Since $h(x_n) = 0$ for all $n$, we have
$$\lim_{n \to \infty} h(x_n) = 0$$
Therefore $h(c) = 0$, which means $c \in K$.
Thus every limit point of $K$ belongs to $K$ meaning the set $K$ is closed.
\end{proof}


\begin{problem} % Problem 62
If $c$ is an isolated point of $A\subset\RR$, and if $f:A\to\RR$ is a function, then $f$ is continuous at $c$.
\end{problem}

% PLEASE READ: Read Definition 4.3.1 carefully.

\begin{proof}
  Suppose $c$ is an isolated point for $A \subset \RR$. Then by definition there exists $\delta_0 > 0$ such that $(c - \delta_0, c + \delta_0) \cap A = \{c\}$
  We want to show that $f$ is continous at $c$. For this we let $\eps > 0$, we must find $\delta > 0$ such that for all $x \in A$ with $|x - c| < \delta$, we have $|f(x) - f(c)| < \eps$.

  Choose $\delta = \delta_0$ and suppose $x \in A$ such that $|x - c| < \delta = \delta_0$. Then $x \in (c - \delta_0, c+ \delta_0) \cap A = \{c\}$, which means $x = c$. Therefore when $x \in A$ and $|x - c| < \delta$, we have $x = c$, so
  $$|f(x) - f(c)| = |f(c) - f(c)| = 0 < \eps$$
  Thus $f$ is continuous at $c$.
\end{proof}


\begin{problem} % Problem 63
The function $g:\RR\to\RR$ defined by $g(x) = \sqrt[3]{x}$ is continuous.
\end{problem}

% PLEASE READ: You will want to prove that g is continuous at c = 0, and separately prove it for every other point c \ne 0.  For the latter, the identity a^3 - b^3 = (a - b) (a^2 + ab + b^2) will be useful.

\begin{proof}
We need to show that $g(x) = \sqrt[3]{x}$ is continuous at every point $c \in \RR$.
\newline\noindent
\textbf{Case 1: $c = 0$:}

Let $\eps > 0$. We need to find $\delta > 0$ such that $|x - 0| < \delta$ implies $|\sqrt[3]{x} - 0| < \eps$.

Choose $\delta = \eps^3$. Then if $|x| < \delta = \eps^3$, we have
$$|\sqrt[3]{x}| = \sqrt[3]{|x|} < \sqrt[3]{\eps^3} = \eps$$
Thus $g$ is continuous at $c = 0$.
\newline\noindent
\textbf{Case 2: $c > 0$:}

Let $\eps > 0$. Using the identity $a^3 - b^3 = (a - b)(a^2 + ab + b^2)$ with $a = \sqrt[3]{x}$ and 

$b = \sqrt[3]{c}$, we have
$$x - c = (\sqrt[3]{x} - \sqrt[3]{c})((\sqrt[3]{x})^2 + \sqrt[3]{x}\sqrt[3]{c} + (\sqrt[3]{c})^2)$$

Therefore
$$\sqrt[3]{x} - \sqrt[3]{c} = \frac{x - c}{(\sqrt[3]{x})^2 + \sqrt[3]{x}\sqrt[3]{c} + (\sqrt[3]{c})^2}$$

Choose $\delta = \frac{c}{2}$. For $|x - c| < \delta = \frac{c}{2}$, we have $x > \frac{c}{2} > 0$, so $\sqrt[3]{x} > 0$ and $\sqrt[3]{c} > 0$. 

Thus
$$(\sqrt[3]{x})^2 + \sqrt[3]{x}\sqrt[3]{c} + (\sqrt[3]{c})^2 > (\sqrt[3]{c})^2$$

So for $|x - c| < \min\left\{\frac{c}{2}, \eps(\sqrt[3]{c})^2\right\}$, we have
$$|\sqrt[3]{x} - \sqrt[3]{c}| = \frac{|x - c|}{(\sqrt[3]{x})^2 + \sqrt[3]{x}\sqrt[3]{c} + (\sqrt[3]{c})^2} < \frac{\eps(\sqrt[3]{c})^2}{(\sqrt[3]{c})^2} = \eps$$

Thus $g$ is continuous at every $c > 0$.
\newline\noindent
\textbf{Case 3: $c < 0$:}

Let $\eps > 0$. Using the same identity as in Case 2, for $|x - c| < \frac{|c|}{2}$, we have 

$x < \frac{c}{2} < 0$, so $\sqrt[3]{x} < 0$ and $\sqrt[3]{c} < 0$. Thus
$$(\sqrt[3]{x})^2 + \sqrt[3]{x}\sqrt[3]{c} + (\sqrt[3]{c})^2 > (\sqrt[3]{c})^2$$

So for $|x - c| < \min\left\{\frac{|c|}{2}, \eps(\sqrt[3]{c})^2\right\}$, we have
$$|\sqrt[3]{x} - \sqrt[3]{c}| < \frac{\eps(\sqrt[3]{c})^2}{(\sqrt[3]{c})^2} = \eps$$

Thus $g$ is continuous at every $c < 0$.

Since $g$ is continuous at every point $c \in \RR$, we conclude that $g$ is continuous on 

$\RR$.
\end{proof}


\begin{problem} % Problem 64
Dirichlet's function from Section 4.1, namely
	$$g(x) = \begin{cases} 1 &\text{ if } x \in \QQ, \\
	                       0 &\text{ if } x \notin \QQ \end{cases}$$
is not continuous at any $c\in\RR$.
\end{problem}

% PLEASE READ: On page 123 there is a suggestion about a good way to show a function is not continuous; feel free to use that technique.

\begin{proof}
Recall the Criterion for Discontinuity: Let $f: A \to \RR$, and let $c \in A$ be a limit point of $A$. If there exists a sequence $(x_n) \subseteq A$ where $(x_n) \to c$ but such that $f(x_n)$ does not converge to $f(c)$, we may conclude that $f$ is not continuous at $c$.
\newline\newline \noindent
Let $c \in \RR$ be arbitrary. Note that $c$ is a limit point of $\RR$ since every neighborhood of $c$ contains infinitely many points of $\RR$. By the density of rationals in $\RR$, there exists a sequence $(r_n)$ of rational numbers with $r_n \to c$. Since $r_n \in \QQ$ for all $n$, we have $g(r_n) = 1$ for all $n$. Consider the following cases
\newline\noindent
\textbf{Case 1: $c \in \QQ$:} 

Then $g(c) = 1$. But by the density of irrationals in $\RR$, there exists a sequence 

$(s_n)$ of irrational numbers with $s_n \to c$. Since $s_n \notin \QQ$ for all $n$, we have $g(s_n) = 0$ 

for all $n$. Thus $g(s_n) \to 0 \neq 1 = g(c)$.

By the Criterion for Discontinuity, $g$ is not continuous at $c$.
\newline\noindent
\textbf{Case 2: $c \notin \QQ$:}


Then $g(c) = 0$. Since $r_n \to c$ and $g(r_n) = 1$ for all $n$, we have 

$g(r_n) \to 1 \neq 0 = g(c)$.
\noindent \newline
By the Criterion for Discontinuity, $g$ is not continuous at $c$. Since $c$ was arbitrary, $g$ is not continuous at any point in $\RR$.
\end{proof}


\begin{problem} % Problem 65
The function
	$$h(x) = \begin{cases} 0 &\text{ if } x = 0, \\
	                       \sqrt{|x|} \cos(1/x) &\text{ otherwise}, \end{cases}$$
shown in the figure below, is continuous at zero.
\end{problem}

% PLEASE READ: Please sketch the graph of h(x), or generate the graph using a computer, and include that figure in your solution.  Consider making your figure a .jpg and then using this LaTeX:
%\begin{center}
  \includegraphics[width=0.4\textwidth]{A9-Desmos.png}
%\end{center}

\begin{proof}
Let $\eps > 0$.

We must find $\delta > 0$ such that $|x - 0| < \delta$ implies $|h(x) - h(0)| < \eps$.

Since $h(0) = 0$, we need to show that $|h(x)| < \eps$ when $|x| < \delta$.

For $x \neq 0$, we have $h(x) = \sqrt{|x|} \cos(1/x)$. Since $|\cos(1/x)| \leq 1$ for all $x \neq 0$, we have
$$|h(x)| = |\sqrt{|x|} \cos(1/x)| = \sqrt{|x|} |\cos(1/x)| \leq \sqrt{|x|} \cdot 1 = \sqrt{|x|}$$

Choose $\delta = \eps^2$. Then if $|x - 0| = |x| < \delta = \eps^2$, we have
$$|h(x) - h(0)| = |h(x)| \leq \sqrt{|x|} < \sqrt{\eps^2} = \eps$$

Since $\eps$ was arbitrary, we conclude that $h$ is continuous at $0$.
\end{proof}


\begin{problem} % Problem 66
Thomae's function from Section 4.1, namely
	$$t(x) = \begin{cases} 1   &\text{ if } x = 0, \\
	                       1/n &\text{ if } x \in \QQ \setminus \{0\} \text{ and } x=\pm m/n \text{ in lowest terms, with } n>0, \\
	                       0   &\text{ if } x \notin \QQ \end{cases}$$
is not continuous at any rational point $c\in\QQ$.
\end{problem}

% PLEASE READ: If you need the fact that every rational is the limit of an irrational sequence, please *justify* (i.e. prove) this fact.

\begin{proof}
We will use the Criterion for Discontinuity again in this problem.  Let $f: A \to \RR$, and let $c \in A$ be a limit point of $A$. If there exists a sequence $(x_n) \subseteq A$ where $(x_n) \to c$ but such that $f(x_n)$ does not converge to $f(c)$, we may conclude that $f$ is not continuous at $c$.

Let $c \in \QQ$ be arbitrary. We will show that $t$ is not continuous at $c$ using the Criterion for Discontinuity.

First, we show that every rational is the limit of an irrational sequence. For each $n \in \NN$, define $s_n = c + \frac{\sqrt{2}}{n}$. Since $\sqrt{2}$ is irrational and $c$ is rational, each $s_n$ is irrational. Clearly $s_n \to c$ as $n \to \infty$ since
$$|s_n - c| = \left|\frac{\sqrt{2}}{n}\right| = \frac{\sqrt{2}}{n} \to 0$$

Since each $s_n \notin \QQ$, we have $t(s_n) = 0$ for all $n$. Thus $t(s_n) \to 0$ as $n \to \infty$.

However, since $c \in \QQ$, we have either $c = 0$ (in which case $t(c) = 1$) or $c = \pm m/n$ in lowest terms with $n > 0$ (in which case $t(c) = 1/n > 0$). In either case, $t(c) > 0$.

Since $t(s_n) \to 0$ but $t(c) > 0$, we have $t(s_n)$ does not converge to $t(c)$.

By the Criterion for Discontinuity, $t$ is not continuous at $c$.

Since $c$ was an arbitrary rational point, $t$ is not continuous at any rational point.
\end{proof}

\begin{problem} % Problem 67
Suppose $f:A \to \RR$ is continuous at $c\in A$.  Suppose that $g:B\to\RR$ has a domain satisfying $f(A) \subset B$, and that $g$ is continuous at $f(c)$.  Let
	$$h(x) = (g \circ f)(x) = g(f(x))$$
be the composition of functions. Then $h$ is continuous at $c$.
\end{problem}

% PLEASE READ: This is Theorem 4.3.9 in the book.  You can prove it either using the \epsilon-\delta definition of continuity, or the sequential characterization (Theorem 4.3.2 (iii)).

\begin{proof}
We need to show that $h$ is continuous at $c$. Let $\eps > 0$.

Since $g$ is continuous at $f(c)$ and $f(c) \in B$, there exists $\delta_1 > 0$ such that for all $y \in B$,
$$|y - f(c)| < \delta_1 \implies |g(y) - g(f(c))| < \eps$$

Since $f$ is continuous at $c \in A$, using $\delta_1 > 0$ from above, there exists $\delta > 0$ such that for all $x \in A$,
$$|x - c| < \delta \implies |f(x) - f(c)| < \delta_1$$

Now suppose $x \in A$ and $|x - c| < \delta$. Then by continuity of $f$ at $c$, we have
$$|f(x) - f(c)| < \delta_1$$

Since $f(A) \subset B$, we have $f(x) \in B$. Thus we can apply the continuity of $g$ at $f(c)$ with $y = f(x)$ to obtain
$$|g(f(x)) - g(f(c))| < \eps$$

That is, $|h(x) - h(c)| < \eps$.

Since $\eps$ was arbitrary, we conclude that $h$ is continuous at $c$.
\end{proof}

\end{document}
