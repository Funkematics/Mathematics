% <-- a percent symbol indicates a comment which does not affect the output of LaTeX
% you can leave the preamble alone, from here ...
\documentclass[12pt]{article}

\usepackage{amssymb,amsmath,amsthm}
\usepackage[top=1in, bottom=1in, left=1.25in, right=1.25in]{geometry}
\usepackage{enumerate,palatino}
\usepackage[final]{graphicx}
\usepackage[colorlinks=true,citecolor=blue,linkcolor=red,urlcolor=blue]{hyperref}

\newtheorem{problem}{Problem}
% ... to here

% shortcuts for blackboard bold number sets (reals, integers, etc.)
\newcommand{\II}{\ensuremath{\mathbb I}}
\newcommand{\NN}{\ensuremath{\mathbb N}}
\newcommand{\QQ}{\ensuremath{\mathbb Q}}
\newcommand{\RR}{\ensuremath{\mathbb R}}
\newcommand{\ZZ}{\ensuremath{\mathbb Z}}

\newcommand{\eps}{\ensuremath{\epsilon}}

% feel free to add more shortcuts here


\begin{document}
% replace with your name, but otherwise leave this header alone, from here ...
\small
\noindent \textsc{Math 401: Homework Assignment 4} \hfill YOUR NAME HERE

\normalsize
\bigskip
% ... to here

\setcounter{problem}{21}


\begin{problem} % Problem 22
If $f:A\to B$ has an inverse function then $f$ is onto and $f$ is one-to-one.
\end{problem}

% PLEASE READ:  Recall that g:B->A is an inverse function of f:A->B if g(f(a))=a for all a in A and if f(g(b))=b for all b in B.  You will want to write separate paragraphs in your proof for the two results, something like "First, to prove $f$ is onto we ..." and then "To prove $f$ is one-to-one ..."
\begin{proof}
Suppose $f : A \to B$ has an inverse function $g : B \to A$. By definition of an inverse function this means that 
\begin{align*}
	g(f(a)) &= a, \text{ for all } a \in A \\
	f(g(b)) &= b, \text{ for all } b \in B
\end{align*}
\textbf{one-to-one : }Let $a_1, a_2 \in A$ such that $f(a_1) = f(a_2)$. To prove $f$ is one-to-one(injective) we must show that $a_1 = a_2$. Observe that composing the inverse $g$ with $f$ gives
\begin{align*} 
	g(f(a_1) &= g(f(a_2)) 
\end{align*}
As a result of applying the definition of inverse $g(f(a)) = a$, we get
$$a_1 = a_2$$
Therefore $f$ is injective. \newline
\textbf{onto : } To prove that $f$ is onto, we must show that there exists an $a \in A$ such that $f(a) = b$. Since $g : B \to A$, we know that $g(b) \in A$. Let $a = g(b)$. Substituting $a$ in $f(a)$ gives
$$f(a) = f(g(b)) = b$$
As a result of applying the definition of inverse $f(g(b)) = b$.
Therefore, for every $b \in B$, there exists $a =g(b) \in A$ such that $f(a) = b$.
Hence $f$ is onto. \newline \newline
Since $f$ is both one-to-one and onto, it follows that $f$ is bijective.
\end{proof} 



% PLEASE READ:  This following problem is part of Exercise 1.5.9.  To get started on part (a), think of the number as $x$.  Squaring or cubing of $x$ is a route to a polynomial equation for which $x$ is a root.  For part (b), one way is to use the fact that every polynomial has a finite number of roots, and then the fact, proved in class, that the union of a (countable) sequence of finite sets is countable.  For part (c) you can use Theorem 1.5.8 (ii) on the result in (b).

\renewcommand{\labelenumi}{(\alph{enumi})}

\begin{problem} % Problem 23
A real number $x\in\RR$ is called \emph{algebraic} if there exists $a_0, a_1, \dots, a_{n-1}, a_n \in \ZZ$, not all zero, so that
	$$a_n x^n + a_{n-1} x^{n-1} + \dots + a_1 x + a_0 = 0$$
That is, a real number is algebraic if it is a root of a polynomial equation with integer coefficients.

\begin{enumerate}
\item The numbers $\sqrt{2}$, $\sqrt[3]{2}$, and $\sqrt{3} + \sqrt{2}$ are algebraic.
\newline \textbf{For $\sqrt{2}$ :} 
\begin{proof}
Let $p(x) = x^2 - 2$. We verify that $\sqrt{2}$ is a root of this polynomial:
	$$p(\sqrt{2}) = (\sqrt{2})^2 - 2 = 2 - 2  = 0$$
	Thus $\sqrt{2}$ is algebraic.
\end{proof}
\textbf{For $\sqrt[3]{2}$ :}
\begin{proof}
	Let $p(x) = x^3 - 2$. We verify that $\sqrt[3]{2}$ is a root of this polynomial:
	$$p(\sqrt[3]{2}) = (\sqrt[3]{2})^3 - 2 = 2 - 2 = 0$$
	Therefore $\sqrt[3]{2}$ is algebraic.
\end{proof}
\textbf{For $\sqrt{3} + \sqrt{2}$:}
\begin{proof}
	Let $p(x) = x^4 - 10x^2 + 1$. We verify that $\sqrt{3} + \sqrt{2}$ is a root of this polynomial:
	\begin{align*}
		p(\sqrt{3} + \sqrt{2}) &= (\sqrt{3}+\sqrt{2})^4 - 10(\sqrt{3} + \sqrt{2})^2 + 1 \\
													 &= (3 + 2\sqrt{6} + 2)^2 - 10(3 + 2\sqrt{6} + 2) + 1 \\
													 &= (5 + 2\sqrt{6})^2 - 30 - 20\sqrt{6} - 20 + 1 \\
													 &= 25 + 20\sqrt{6} + 24 - 30 - 20\sqrt{6} - 20 + 1 \\
													 &= 50 + 20\sqrt{6} - 50 - 20\sqrt{6} \\
													 &= 0
	\end{align*}
	Thus $\sqrt{3} + \sqrt{2}$ is algebraic.
\end{proof}

\item For fixed $n\in\NN$, let $A_n$ be the set of algebraic numbers which are roots of polynomials, with integer coefficients, of degree $n$.  Then $A_n$ is countable.

\begin{proof}
A polynomial $P \in P_n$ of degree $n$ with integer coefficients is defined as 
	$$P(x) = a_n x^n + a_{n-1} x^{n-1} + \dots + a_1 x + a_0 = 0$$ 
	where $a_i \in \ZZ$ and $a_n \neq 0$. Let $P_n$ denote the set of all such degree-n polynomials. Each polynomial is uniquely determined by the $(n+1)$-tuple $(a_n, a_{n-1}, \dots, a_1, a_0)$ where $a_n \neq 0$. This is a bijection between $P_n$ and a subset of $Z^{n+1}$. Since $Z^{n+1}$ is countable, any subset of $Z^{n+1}$ is at most countable. Therefore $P_n$ is countable. We can thus enumerate $P_n = \{P_1, P_2, P_3, \dots\}$. \newline \newline
	By the Fundamental Theorem of Algebra, each polynomial of degree $n$ has at most $n$ roots. For each $i \in \NN$, let $R_i$ denote the set of all roots of $P_i$. Then $|R_i| \leq n \leq \infty$. \newline \newline
	Every algebraic number in $A_n$ is, by definition a root of some polynomial in $P_n$. Therefore $A_n = \cup_{i=0}^{\infty} R_i$.
\end{proof}

\item The set of all algebraic numbers is countable.

\begin{proof}
% FILL IN
\end{proof}
\end{enumerate}
\end{problem}


\begin{problem} % Problem 24
There is an onto function $f:(0,1) \to S$ where $S=\{(x,y)\,:\,0<x,y<1\}$ is the unit square in the plane $\RR^2$.
\end{problem}

% PLEASE READ:  My hint is to start with $u\in (0,1)$ and $v\in (0,1)$ and consider their decimal expansions.  How might you combine the decimal expansions to generate one number in $(0,1)$?  Use this idea to define $x$ so that $f(x)=(u,v)$.  Of course, the plane $\RR^2$ is the set of pairs $(u,v)$ where $u$ and $v$ are any real numbers, so $S$ is just the part of the plane where each coordinate has a decimal expansion $0.b_1 b_2 b_3 \dots$ starting with zero.  Note that there is no claim that this function $f$ is continuous, but you should show it is well-defined and onto.  This function is a bit surprising because $S$ would seem to be so much larger than $(0,1)$.

\begin{proof}
% FILL IN
\end{proof}


% PLEASE READ:  For the following problem prove the limit directly using Definition 2.2.3.  Start your proof with ``Let $\eps>0$.  '' as shown.  However, you should do some calculations first so that you know how to choose $N$ from $\eps$ in the Definition and then complete the proof.  In (c), use the fact that $|\sin n|\le 1$; you do not have to prove this.

\begin{problem} % Problem 25
\begin{enumerate}
\item $\displaystyle \lim_{n\to\infty} \frac{2n+1}{5n+3} = \frac{2}{5}$

\begin{proof}
Let $\eps>0$.  % FILL IN
\end{proof}

\item $\displaystyle \lim_{n\to\infty} \frac{2n^2}{n^3 + 1} = 0$

\begin{proof}
Let $\eps>0$.  % FILL IN
\end{proof}

\item $\displaystyle \lim_{n\to\infty} \frac{\sin(n)}{\sqrt{n}} = o$

\begin{proof}
Let $\eps>0$.  % FILL IN
\end{proof}
\end{enumerate}
\end{problem}


% PLEASE READ:  For each part, either give an example, with justification, or prove that the request is impossible.  Edit the statements so that it is clear what you are doing, and insert a proof environment if appropriate.

\begin{problem} % Problem 26
\begin{enumerate}
\item A sequence with an infinite number of ones that does not converge to one.

\bigskip
\item A sequence with an infinite number of ones that converges to a limit not equal to one.

\bigskip
\end{enumerate}
\end{problem}


\begin{problem} % Problem 27
Let $(x_n)$ be a sequence that converges to $x$.  Suppose $p(x)$ is a polynomial.  Then
	$$\lim_{n\to\infty} p(x_n) = p(x).$$
\end{problem}

% PLEASE READ:  Prove this using the algebraic limit theorem (Theorem 2.3.3).  You will want to induct on the degree of the polynomial.  You can assume $p(x)$ is either in standard form or in Horner's nested form.

\begin{proof}
% FILL IN
\end{proof}


\begin{problem} % Problem 28
Consider three sequences $(x_n)$, $(y_n)$, and $(z_n)$ for which $x_n \le y_n \le z_n$ for each $n$.  If $x_n \to \ell$ and $z_n \to \ell$ then $y_n \to \ell$.
\end{problem}

% PLEASE READ:  This is the squeeze theorem.  This is not just an application of the order limit theorem (Theorem 2.3.4) because you need to prove that $\lim y_n$ exists.  I would just prove from the original Definition 2.2.3.

\begin{proof}
% FILL IN
\end{proof}

\end{document}
