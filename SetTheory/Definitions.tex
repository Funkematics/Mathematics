\documentclass[]{book}

%These tell TeX which packages to use.
\usepackage{array,epsfig}
\usepackage{amsmath}
\usepackage{amsfonts}
\usepackage{amssymb}
\usepackage{amsxtra}
\usepackage{amsthm}
\usepackage{mathrsfs}
\usepackage{color}

%Here I define some theorem styles and shortcut commands for symbols I use often
\theoremstyle{definition}
\newtheorem{defn}{Definition}
\newtheorem{thm}{Theorem}
\newtheorem{cor}{Corollary}
\newtheorem*{rmk}{Remark}
\newtheorem{lem}{Lemma}
\newtheorem*{joke}{Joke}
\newtheorem{ex}{Example}
\newtheorem*{soln}{Solution}
\newtheorem{prop}{Proposition}

\newcommand{\lra}{\longrightarrow}
\newcommand{\ra}{\rightarrow}
\newcommand{\surj}{\twoheadrightarrow}
\newcommand{\graph}{\mathrm{graph}}
\newcommand{\bb}[1]{\mathbb{#1}}
\newcommand{\Z}{\bb{Z}}
\newcommand{\Q}{\bb{Q}}
\newcommand{\R}{\bb{R}}
\newcommand{\C}{\bb{C}}
\newcommand{\N}{\bb{N}}
\newcommand{\M}{\mathbf{M}}
\newcommand{\m}{\mathbf{m}}
\newcommand{\MM}{\mathscr{M}}
\newcommand{\HH}{\mathscr{H}}
\newcommand{\Om}{\Omega}
\newcommand{\Ho}{\in\HH(\Om)}
\newcommand{\bd}{\partial}
\newcommand{\del}{\partial}
\newcommand{\bardel}{\overline\partial}
\newcommand{\textdf}[1]{\textbf{\textsf{#1}}\index{#1}}
\newcommand{\img}{\mathrm{img}}
\newcommand{\ip}[2]{\left\langle{#1},{#2}\right\rangle}
\newcommand{\inter}[1]{\mathrm{int}{#1}}
\newcommand{\exter}[1]{\mathrm{ext}{#1}}
\newcommand{\cl}[1]{\mathrm{cl}{#1}}
\newcommand{\ds}{\displaystyle}
\newcommand{\vol}{\mathrm{vol}}
\newcommand{\cnt}{\mathrm{ct}}
\newcommand{\osc}{\mathrm{osc}}
\newcommand{\LL}{\mathbf{L}}
\newcommand{\UU}{\mathbf{U}}
\newcommand{\support}{\mathrm{support}}
\newcommand{\AND}{\;\wedge\;}
\newcommand{\OR}{\;\vee\;}
\newcommand{\Oset}{\varnothing}
\newcommand{\st}{\ni}
\newcommand{\wh}{\widehat}

%Pagination stuff.
\setlength{\topmargin}{-.3 in}
\setlength{\oddsidemargin}{0in}
\setlength{\evensidemargin}{0in}
\setlength{\textheight}{9.in}
\setlength{\textwidth}{6.5in}
\pagestyle{empty}



\begin{document}
\pagecolor{black}
\color{white}

\begin{center}

{\Large Introduction to Abstract Mathematics $\hspace{0.3cm}$}%\\$
\textbf{FunkyCosmonaut}\\ %You should put your name here
\end{center}

\vspace{0.2 cm}

%\subsection*{Exercises for Section 1.1: Norm and Inner Product}
\noindent\rule{\textwidth}{1pt}
\begin{enumerate}
\item\label{def}
Definition 1: Given  a set {$\mathnormal{A}$}, we define the complement of {$\mathnormal{A}$}, denoted 
{$\mathnormal{A^C}$} in the following way.
\newline
\begin{center}
    {$\mathnormal{A^C} = \{ \mathnormal{x} :\mathnormal{x} \in \mathnormal{U}$ and $ \mathnormal{x} \notin \mathnormal{A} \}$}
\end{center}
Thus {$\mathnormal{x} \in \mathnormal{A^C}$} if and only if {$\mathnormal{x} \notin \mathnormal{A}$}.
\newline\noindent\rule{\textwidth}{1pt}
 

\item\label{def2}
Definition 2: Given two sets {$\mathnormal{A}$ and $\mathnormal{B}$}, we define their \emph{union} {$\mathnormal{A} \cup \mathnormal{B} $} in the following way.

\begin{center}
    {$\mathnormal{A} \cup \mathnormal{B} = \{ \mathnormal{x} :\mathnormal{x} \in \mathnormal{A}$ and $ \mathnormal{x} \in \mathnormal{B} \} $}
\end{center}
Thus {$\mathnormal{x} \in \mathnormal{A}\cup\mathnormal{B}$} if and only if {$x \in A$} and
{$x \in B$}
\newline\noindent\rule{\textwidth}{1pt}
\end{enumerate}



\end{document}


