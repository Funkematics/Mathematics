% !TEX TS-program = pdflatexmk
\documentclass[12pt]{amsart}

%\usepackage[parfill]{parskip}    % Activate to begin paragraphs with an empty line rather than an indent

\usepackage[margin=1in]{geometry}

\usepackage{amsmath,amssymb,amsthm,latexsym,graphicx}
\usepackage[normalem]{ulem}
\usepackage{setspace} %used for doublespacing, etc.
\usepackage{hyperref}
\usepackage[dvipsnames,usenames]{color}
\usepackage{fancyhdr}
\pagestyle{fancy}
	\renewcommand{\headrulewidth}{0.5pt} % and the line
	\headsep=1cm
	
\DeclareGraphicsRule{.tif}{png}{.png}{`convert #1 `dirname #1`/`basename #1 .tif`.png}

%Some useful environments.
\newtheorem{theorem}{Theorem}
\newtheorem{corollary}[theorem]{Corollary}
\newtheorem{conjecture}[theorem]{Conjecture}
\newtheorem{lemma}[theorem]{Lemma}
\newtheorem{proposition}[theorem]{Proposition}
\newtheorem{definition}[theorem]{Definition}
\newtheorem{example}[theorem]{Example}
\newtheorem{axiom}{Axiom}
\theoremstyle{remark}
\newtheorem{remark}{Remark}
\newtheorem*{exercise}{Exercise}%[section]

%Some shortcuts helpful for our assignments
\newcommand{\bx}{\begin{exercise}}
\newcommand{\ex}{\end{exercise}}

%Some useful shortcuts for our favorite sets of numbers.
%Note, you can use these WITHOUT entering math mode
\def\RR{\ensuremath{\mathbb R}} 
\def\NN{\ensuremath{\mathbb N}}
\def\ZZ{\ensuremath{\mathbb Z}}
\def\QQ{{\ensuremath\mathbb Q}}
\def\CC{\ensuremath{\mathbb C}}
\def\EE{{\ensuremath\mathbb E}}

%Some useful shortcuts for formatting lists
\newcommand{\bc}{\begin{center}}
\newcommand{\ec}{\end{center}}
\newcommand{\be}{\begin{enumerate}}
\newcommand{\ee}{\end{enumerate}}
\newcommand{\bi}{\begin{itemize}}
\newcommand{\ei}{\end{itemize}}

%Some useful shortcuts for formatting mathematical symbols
\newcommand{\ol}[1]{\overline{#1}}
\newcommand{\oimp}[1]{\overset{#1}{\iff}} %labeled iff symbol
\newcommand{\bv}[1]{\ensuremath{ \vec{\mathbf{#1}}} } %makes a vector.
\newcommand{\mc}[1]{\ensuremath{\mathcal{#1}}} %put something in caligraphic font
\newcommand{\normale}{\trianglelefteq}
\newcommand{\normal}{\triangleleft}

%Code for formatting the proofs a little nicer for submitted homework
\makeatletter
\renewenvironment{proof}[1][\proofname]{\par\doublespacing
  \pushQED{\qed}%
  \normalfont \topsep6\p@\@plus6\p@\relax
  \list{}{%
    \settowidth{\leftmargin}{\itshape\proofname:\hskip\labelsep}%
    \setlength{\labelwidth}{0pt}%
    \setlength{\itemindent}{-\leftmargin}%
  }%
  \item[\hskip\labelsep\itshape#1\@addpunct{:}]\ignorespaces
}{%
  \popQED\endlist\@endpefalse
  \singlespacing
}
\makeatother


%Commenting tools for the professor
\newcommand{\mpg}[1]{\marginpar{ #1}} %to put comments in margins
\usepackage{soul}
\definecolor{highlight}{rgb}{1,0.6,0.6}
\sethlcolor{highlight}
\newcommand{\hlm}[1]{\colorbox{highlight}{$\displaystyle #1$}}
\newtheoremstyle{mycomment}{\topsep}{-0in}{\small \itshape \sffamily}{}{\small \itshape\sffamily}{:}{.5em}{}
\theoremstyle{mycomment}
\newtheorem*{acomment}{\color{BrickRed}{Comment}}
\newcommand{\com}[1]{{\color{OliveGreen}\begin{acomment}{#1} %#2 \color{black} 
\end{acomment}\noindent}}
\newcommand{\red}[1]{{\color{BrickRed} #1}}
\newcommand{\blue}[1]{{\color{MidnightBlue}#1}}
\newcommand{\green}[1]{{\color{OliveGreen}#1}}
\newcommand{\mwrong}[2]{\red{\cancel{#1}}\green{#2}}
\newcommand{\wrong}[2]{\red{\sout{#1}}\green{#2}}
\definecolor{OliveGreen}{rgb}{.3,.5,.2}
\definecolor{MidnightBlue}{rgb}{.3,.4,.6}
\newcommand{\pts}[1]{\hfill\blue{{#1}/5}}

\chead{MATH 371}
\pagestyle{fancy}
%Modify these items:
\rhead{\emph{Christopher Munoz}}
\lhead{\emph{HW 8}}

\begin{document}

\thispagestyle{fancy}
%§12.1 1, 3, 7, 9, 12.
\section*{4.2}

\begin{exercise}[4.58]
Use Table 4, Appendix 3, to find the following probabilities for a standard normal random variable $Z$:

\begin{enumerate}
    \item[(a)] $P(0 \leq Z \leq 1.2)$
\begin{proof}[Solution]
 $.5000 - .1151 = .3849$
\end{proof}
    \item[(b)] $P(-.9 \leq Z \leq 0)$
\begin{proof}[Solution]
$.5000-.1841=.3159$ 
\end{proof}
    \item[(c)] $P(.3 \leq Z \leq 1.56)$
\begin{proof}[Solution]
$.3821-.0594 = .3227$ 
\end{proof}
    \item[(d)] $P(-.2 \leq Z \leq .2)$
\begin{proof}[Solution]
	$2*(.5000-.4207) = .1586$ 
\end{proof}
    \item[(e)] $P(-1.56 \leq Z \leq -.2)$
\begin{proof}[Solution]
	$(.5000-.0594) + (.5000-.4207) = .5199 $
\end{proof}
\end{enumerate}
\end{exercise}

\begin{exercise}[4.59]
If $Z$ is a standard normal random variable, find the value $z_0$ such that

\begin{enumerate}
    \item[(a)] $P(Z > z_0) = .5$.
\begin{proof}[Solution]
$z_0 = 0$ 
\end{proof}
    \item[(b)] $P(Z < z_0) = .8643$.
\begin{proof}[Solution]
 $z_0 = 1.1$
\end{proof}
    \item[(c)] $P(-z_0 < Z < z_0) = .90$.
\begin{proof}[Solution]
	\begin{align*}
		.90 &=  P(Z<z_0) - 1 + P(Z< z_0) \\
		    &= 2P(Z<z_0) - 1 \\
		1.90 &= 2p(Z<z_0) \\
		0.95 &= p(Z<z_0) \\
		z_0 &= 1.645 
	\end{align*}
\end{proof}
    \item[(d)] $P(-z_0 < Z < z_0) = .99$.
\begin{proof}[Solution]
	$z_0 = 2.576$
\end{proof}
\end{enumerate}
\end{exercise}

\begin{exercise}[4.61]
What is the median of a normally distributed random variable with mean $\mu$ and standard deviation $\sigma$?

\begin{proof}[Solution]
$\mu$ is the median 
\end{proof}
\end{exercise}

\begin{exercise}[4.63]
A company that manufactures and bottles apple juice uses a machine that automatically fills 16-ounce bottles. There is some variation, however, in the amounts of liquid dispensed into the bottles that are filled. The amount dispensed has been observed to be approximately normally distributed with mean 16 ounces and standard deviation 1 ounce.

\begin{enumerate}
    \item[(a)] Use Table 4, Appendix 3, to determine the proportion of bottles that will have more than 17 ounces dispensed into them.
\begin{proof}[Solution]
	\begin{align*}
		z &= \frac{\gamma - \mu}{\sigma} \\
		z &= \frac{17-16}{1} = 1 \\
		p(Z > 1) &= .1587
	\end{align*}
\end{proof}
\end{enumerate}
\end{exercise}

\begin{exercise}[4.64]
The weekly amount of money spent on maintenance and repairs by a company was observed, over a long period of time, to be approximately normally distributed with mean \$400 and standard deviation \$20. If \$450 is budgeted for next week, what is the probability that the actual costs will exceed the budgeted amount?

\begin{enumerate}
    \item[(a)] Answer the question, using Table 4, Appendix 3.
\begin{proof}[Solution]
	\begin{align*}
		z &= \frac{\gamma - \mu}{\sigma} \\
		z &= \frac{450 - 400}{20} = 2.5 \\
		P(Z > 2.5) &= .0062
	\end{align*}
\end{proof}
\end{enumerate}
\end{exercise}

\begin{exercise}[4.65]
In Exercise 4.64, how much should be budgeted for weekly repairs and maintenance to provide that the probability the budgeted amount will be exceeded in a given week is only .1?

\begin{proof}[Solution]
Choose $z = 1.28$ from table $(P = .1003)$, let $n$ denote how much we should budget
	\begin{align*}
		z = 1.28 &= \frac{n - 400}{20} \\
		25.6 &= n - 400 \\
		425.6 &= n 
	\end{align*}
\end{proof}
\end{exercise}

\begin{exercise}[4.66]
A machining operation produces bearings with diameters that are normally distributed with mean 3.0005 inches and standard deviation .0010 inch. Specifications require the bearing diameters to lie in the interval $3.000 \pm .0020$ inches. Those outside the interval are considered scrap and must be remachined. With the existing machine setting, what fraction of total production will be scrap?

\begin{enumerate}
    \item[(a)] Answer the question, using Table 4, Appendix 3.
\begin{proof}[Solution]
 
\end{proof}
    \item[(b)] Applet Exercise Obtain the answer, using the applet Normal Probabilities.
\begin{proof}[Solution]
 
\end{proof}
\end{enumerate}
\end{exercise}

\begin{exercise}[4.68]
The grade point averages (GPAs) of a large population of college students are approximately normally distributed with mean 2.4 and standard deviation .8. What fraction of the students will possess a GPA in excess of 3.0?

\begin{enumerate}
    \item[(a)] Answer the question, using Table 4, Appendix 3.
\begin{proof}[Solution]
 
\end{proof}
    \item[(b)] Applet Exercise Obtain the answer, using the applet Normal Tail Areas and Quantiles.
\begin{proof}[Solution]
 
\end{proof}
\end{enumerate}
\end{exercise}

\begin{exercise}[4.69]
Refer to Exercise 4.68. If students possessing a GPA less than 1.9 are dropped from college, what percentage of the students will be dropped?

\begin{proof}[Solution]
 
\end{proof}
\end{exercise}

\begin{exercise}[4.70]
Refer to Exercise 4.68. Suppose that three students are randomly selected from the student body. What is the probability that all three will possess a GPA in excess of 3.0?

\begin{proof}[Solution]
 
\end{proof}
\end{exercise}

\begin{exercise}[4.73]
The width of bolts of fabric is normally distributed with mean 950 mm (millimeters) and standard deviation 10 mm.

\begin{enumerate}
    \item[(a)] What is the probability that a randomly chosen bolt has a width of between 947 and 958 mm?
\begin{proof}[Solution]
 
\end{proof}
    \item[(b)] What is the appropriate value for $C$ such that a randomly chosen bolt has a width less than $C$ with probability .8531?
\begin{proof}[Solution]
 
\end{proof}
\end{enumerate}
\end{exercise}

\begin{exercise}[4.77]
The SAT and ACT college entrance exams are taken by thousands of students each year. The mathematics portions of each of these exams produce scores that are approximately normally distributed. In recent years, SAT mathematics exam scores have averaged 480 with standard deviation 100. The average and standard deviation for ACT mathematics scores are 18 and 6, respectively.

\begin{enumerate}
    \item[(a)] An engineering school sets 550 as the minimum SAT math score for new students. What percentage of students will score below 550 in a typical year?
\begin{proof}[Solution]
 
\end{proof}
    \item[(b)] What score should the engineering school set as a comparable standard on the ACT math test?
\begin{proof}[Solution]
 
\end{proof}
\end{enumerate}
\end{exercise}
 \end{document} 
