% !TEX TS-program = pdflatexmk
\documentclass[12pt]{amsart}

%\usepackage[parfill]{parskip}    % Activate to begin paragraphs with an empty line rather than an indent

\usepackage[margin=1in]{geometry}

\usepackage{amsmath,amssymb,amsthm,latexsym,graphicx}
\usepackage[normalem]{ulem}
\usepackage{setspace} %used for doublespacing, etc.
\usepackage{hyperref}
\usepackage{cancel}
\usepackage[dvipsnames,usenames]{color}
\usepackage[all]{xy}
\usepackage{fancyhdr}
\pagestyle{fancy}
	\renewcommand{\headrulewidth}{0.5pt} % and the line
	\headsep=1cm
	
\DeclareGraphicsRule{.tif}{png}{.png}{`convert #1 `dirname #1`/`basename #1 .tif`.png}

%Some useful environments.
\newtheorem{theorem}{Theorem}
\newtheorem{corollary}[theorem]{Corollary}
\newtheorem{conjecture}[theorem]{Conjecture}
\newtheorem{lemma}[theorem]{Lemma}
\newtheorem{proposition}[theorem]{Proposition}
\newtheorem{definition}[theorem]{Definition}
\newtheorem{example}[theorem]{Example}
\newtheorem{axiom}{Axiom}
\theoremstyle{remark}
\newtheorem{remark}{Remark}
\newtheorem*{exercise}{Exercise}%[section]

%Some shortcuts helpful for our assignments
\newcommand{\bx}{\begin{exercise}}
\newcommand{\ex}{\end{exercise}}

%Some useful shortcuts for our favorite sets of numbers.
%Note, you can use these WITHOUT entering math mode
\def\RR{\ensuremath{\mathbb R}} 
\def\NN{\ensuremath{\mathbb N}}
\def\ZZ{\ensuremath{\mathbb Z}}
\def\QQ{{\ensuremath\mathbb Q}}
\def\CC{\ensuremath{\mathbb C}}
\def\EE{{\ensuremath\mathbb E}}

%Some useful shortcuts for formatting lists
\newcommand{\bc}{\begin{center}}
\newcommand{\ec}{\end{center}}
\newcommand{\be}{\begin{enumerate}}
\newcommand{\ee}{\end{enumerate}}
\newcommand{\bi}{\begin{itemize}}
\newcommand{\ei}{\end{itemize}}

%Some useful shortcuts for formatting mathematical symbols
\newcommand{\ol}[1]{\overline{#1}}
\newcommand{\oimp}[1]{\overset{#1}{\iff}} %labeled iff symbol
\newcommand{\bv}[1]{\ensuremath{ \vec{\mathbf{#1}}} } %makes a vector.
\newcommand{\mc}[1]{\ensuremath{\mathcal{#1}}} %put something in caligraphic font
\newcommand{\normale}{\trianglelefteq}
\newcommand{\normal}{\triangleleft}

%Code for formatting the proofs a little nicer for submitted homework
\makeatletter
\renewenvironment{proof}[1][\proofname]{\par\doublespacing
  \pushQED{\qed}%
  \normalfont \topsep6\p@\@plus6\p@\relax
  \list{}{%
    \settowidth{\leftmargin}{\itshape\proofname:\hskip\labelsep}%
    \setlength{\labelwidth}{0pt}%
    \setlength{\itemindent}{-\leftmargin}%
  }%
  \item[\hskip\labelsep\itshape#1\@addpunct{:}]\ignorespaces
}{%
  \popQED\endlist\@endpefalse
  \singlespacing
}
\makeatother


%Commenting tools for the professor
\newcommand{\mpg}[1]{\marginpar{ #1}} %to put comments in margins
\usepackage{soul}
\definecolor{highlight}{rgb}{1,0.6,0.6}
\sethlcolor{highlight}
\newcommand{\hlm}[1]{\colorbox{highlight}{$\displaystyle #1$}}
\newtheoremstyle{mycomment}{\topsep}{-0in}{\small \itshape \sffamily}{}{\small \itshape\sffamily}{:}{.5em}{}
\theoremstyle{mycomment}
\newtheorem*{acomment}{\color{BrickRed}{Comment}}
\newcommand{\com}[1]{{\color{OliveGreen}\begin{acomment}{#1} %#2 \color{black} 
\end{acomment}\noindent}}
\newcommand{\red}[1]{{\color{BrickRed} #1}}
\newcommand{\blue}[1]{{\color{MidnightBlue}#1}}
\newcommand{\green}[1]{{\color{OliveGreen}#1}}
\newcommand{\mwrong}[2]{\red{\cancel{#1}}\green{#2}}
\newcommand{\wrong}[2]{\red{\sout{#1}}\green{#2}}
\definecolor{OliveGreen}{rgb}{.3,.5,.2}
\definecolor{MidnightBlue}{rgb}{.3,.4,.6}
\newcommand{\pts}[1]{\hfill\blue{{#1}/5}}

\chead{MATH 265F}
\pagestyle{fancy}
%Modify these items:
\rhead{\emph{Christopher Munoz}}
\lhead{\emph{HW 6}}

\begin{document}

\thispagestyle{fancy}
%§12.1 1, 3, 7, 9, 12.
\section*{Section 3.5} 



\begin{exercise}[3.92] Ten percent of the engines manufactured on an assembly line are defective. If engines are
randomly selected one at a time and tested, what is the probability that the first nondefective
engine will be found on the second trial?
\begin{proof}[Solution]
We have Y-Geom(.10) where we want the second success so
$$P(Y=2) = (1-.90)^{2-1}(.90) = 0.09$$
\end{proof}
\end{exercise}

\begin{exercise}[3.93] Refer to Exercise 3.92. What is the probability that the third nondefective engine will be found\newline

  a on the fifth trial?
\begin{proof}[Solution]

  $$P(Y=5)={5-1 \choose 3-1}(.90)^3(1-.90)^{5-3} = 0.04374$$
\end{proof}

b
on or before the fifth trial?
\begin{proof}[Solution]
  \begin{align*}
  P(Y \leq 5) &= P(Y = 3) + P(Y=4) + P(Y=5) \\
              &= {3-1 \choose 3-1}(.90)^3(1-.90)^{3-3} + {4-1 \choose 3-1}(.90)^3(1-.90)^{4-3} + 0.04374 \\
              &= 0.729 + 0.2187 + 0.04374 \\
              &= 0.99144
\end{align*}
\end{proof}
\end{exercise}

\begin{exercise}[3.97] A geological study indicates that an exploratory oil well should strike oil with probability .2.

a
What is the probability that the first strike comes on the third well drilled?
\begin{proof}[Solution]
  $$P(Y=3) = (1-.2)^{3-1}(.2) = 0.128 $$
\end{proof}
b
What is the probability that the third strike comes on the seventh well drilled?
\begin{proof}[Solution]
  Negative Binomial Distribution with $r = 3, p =.2, k = 7$.
  \begin{align*}
    P(Y=k) &= {k-1 \choose r-1}p^r(1-p)^{k-r} \\
    P(Y=7) &= {7-1 \choose 3-1}(.2)^3(.8)^4 \\
           &= 15(0.008)(0.4096) \\
           &= 0.049152
  \end{align*}
\end{proof}
c
What assumptions did you make to obtain the answers to parts (a) and (b)?
\begin{proof}[Solution]
That the trials were independent.
\end{proof}
d
Find the mean and variance of the number of wells that must be drilled if the company
wants to set up three producing wells.
\begin{proof}[Solution]
  \begin{align*}
    E(Y) &= \frac{r}{p} = \frac{3}{.2} = 15 \\
    V(Y) &= \frac{r(1-p)}{p^2} = \frac{3(.8)}{.2^2} = 6 
  \end{align*}
\end{proof}
\end{exercise}

\section*{Section 3.6}
%§12.2 1, 5, 7, 8, 10, 14.
\begin{exercise}[3.102]
An urn contains ten marbles, of which five are green, two are blue, and three are red. Three
marbles are to be drawn from the urn, one at a time without replacement. What is the probability
that all three marbles drawn will be green?
\begin{proof}[Solution]
Our variables for Hypergeo: $N = 10, K = 5, n = 3, k = 3$ 
\begin{align*}
  P(X=k) &= \frac{{K \choose k}{N - K \choose n - k}}{{N \choose n}} \\
         &= \frac{{5 \choose 3}{10 - 5 \choose 3 - 3}}{{10 \choose 3}} \\
         &= \frac{{5 \choose 3}{5 \choose 0}}{{10 \choose 3}} \\
         &= 0.08333333
\end{align*}
\end{proof}
\end{exercise}
\begin{exercise}[3.104] Twenty identical looking packets of white power are such that 15 contain cocaine and 5 do
not. Four packets were randomly selected, and the contents were tested and found to contain
cocaine. Two additional packets were selected from the remainder and sold by undercover
police officers to a single buyer. What is the probability that the 6 packets randomly selected
are such that the first 4 all contain cocaine and the 2 sold to the buyer do not?

\begin{proof}[Solution]
Event 1(first four contain cocaine) $X_1::N_1 = 20, K_1 = 15, n_1 = 4, k_1 = 4$ 

Event 2(Choose 2 no cocaine) $X_2::N_2 = 16, K_2 = 11, n_2 = 2, k_1 = 0$
\begin{align*}
  P(X_1 = 4) &= \frac{{15 \choose 4}{5 \choose 0}}{{20 \choose 4}} = .2817 \\
  P(X_2 = 0) &= \frac{{11 \choose 0}{5 \choose 2}}{{16 \choose 2}} = .0833 \\
  P(X_1 \cap X_2) &= .2817*.0833 = 0.0235
\end{align*}
\end{proof}
\end{exercise}

\begin{exercise}[3.107] 
Seed are often treated with fungicides to protect them in poor draining, wet environments.
A small-scale trial, involving five treated and five untreated seeds, was conducted prior to a
large-scale experiment to explore how much fungicide to apply. The seeds were planted in wet
soil, and the number of emerging plants were counted. If the solution was not effective and
four plants actually sprouted, what is the probability that

a
all four plants emerged from treated seeds?

b
three or fewer emerged from treated seeds?

c
at least one emerged from untreated seeds?
\begin{proof}[Solution]
Write your answer here.
\end{proof}
\end{exercise}

\textbf{Below is an exercise I did by accident and I didn't feel like omitting it.}

\begin{exercise}[3.96]The telephone lines serving an airline reservation office are all busy about 60\% of the time.
  \newline

  a
If you are calling this office, what is the probability that you will complete your call on the
first try? The second try? The third try?
\begin{proof}[Solution]
  Since this is a geometric series, we use our formula $ P(Y=k) = (1-p)^{(k-1)} * p$ for $p = .6$ and $k$ to represents our trials.
  \begin{align*}
    P(Y=1) &= .4  \\
    P(Y=2) &= .6 * .4 = .24 \\
    P(Y=3) &= (.6)^2 * .4 = .144
  \end{align*}
\end{proof}
b
If you and a friend must both complete calls to this office, what is the probability that a
total of four tries will be necessary for both of you to get through?
\begin{proof}[Solution] We use a Negative Binomial Distribution for r-th success where $p=.4, r = 2$
  \begin{align*}
    P(Y = k) &= {k-1 \choose r-1}p^r(1-p)^{k-r} \\
             &= {4-1 \choose 2-1}.4^2(1-.4)^{4-2} \\
             &= 3*(.16)(0.36) \\
             &= 0.1728
  \end{align*}
\end{proof}
\end{exercise}












 \end{document} 
