\documentclass{article}
\usepackage{xcolor}
\usepackage{parskip}
\usepackage{amsmath}
\usepackage{amsthm}
\usepackage{amssymb}
\usepackage{mathtools}
\usepackage{fancyhdr}
\usepackage[%paperheight = 59.4cm,
            %paperwidth = 42cm,
            includehead,
            nomarginpar,
            textwidth=15cm,
            headheight=10mm]{geometry}


\begin{document}

\pagestyle{fancy}
%\fancyhead{}\fancyfoot{}

\fancyhf[OHC]{Christopher Munoz WRH4 Diffeq}
\setcounter{section}{3}
\setcounter{subsection}{1}
\setcounter{subsubsection}{0}
%\section{section1}

\subsection{}
\subsubsection{8 a)}
Suppose $ a = b = 1 $ for the Gompertz differential equation: $\frac{\delta P}{\delta t} = P(a - b\ln P)$, the following are the phase portraits for cases $ P_0 > e$ and $ 0 < P_0 < e $:
\begin{align*}
    & \\
    & \\
    & \\
\end{align*}
\subsubsection*{8 b)}
For $ a = 1, b = -1 $,  cases $ P_0 > e^{-1}$ and $ 0 < P_0 < e^{-1} $:
\begin{align*}
    & \\
    & \\
    & \\
\end{align*}
az
Explicit solution for $ P(O) = P_0 $.
\begin{align*}
    & \frac{dP}{dt} &= P(a - b\ln P) \\
    & \int \frac{dP}{P(a - b\ln P)} & = \int dt \\
    & \int \frac{d(\ln P)}{(a - b \ln P)} &= \int dt \\
    & - \frac{1}{b} \ln |a - b \ln P| + c &= t
\end{align*}
When we plug in $ P(O) = P_0 $ we get $ c = \frac{1}{b}|a - b \ln P_0|$ so we get:
\begin{align*}
    & t = & - \frac{1}{b} \ln |a - b \ln P| + \frac{1}{b} \ln \frac{a - b \ln P_0}{a - b \ln P} \\
    & \ln P(t) = & \frac{a}{b} (1 - e^{-bt}) + e^{-bt} \ln P_0 \\
    &P(t) = & \Aboxed{e^{\frac{a}{b} (1 - e^{-bt})} P_0^{e^{-bt}}} 
\end{align*}
\subsubsection{}
Suppose we have the same 16 pound cannonball shot vertically upward with an initial velocity $v_0 = 300 ft/s$, the differential equation for the cannonball would be: 
\setcounter{equation}{0}
\begin{align*}
    & m \frac{dv}{dt} = -mg - kv^2 \\
\end{align*}
Now we solve using separations of variables:
\begin{align}
    & -dt = \frac{mdv}{mg + kv^2} \\
    & -dt = \frac{1}{g} \frac{dv}{1+(\sqrt{\frac{k}{mg}}v)^2} \\
    & -\int dt = \int \frac{1}{g} \sqrt{\frac{mg}{k}} \frac{\sqrt{\frac{k}{mg}}}{1 + (\sqrt{\frac{k}{mg}}v)^2} dv \\
    & -t + c = \sqrt{\frac{m}{gk}} \tan^{-1}(\sqrt{\frac{k}{mg}}v) \\
    & -\sqrt{\frac{kg}{m}}t + c = \tan^{-1}(\sqrt{\frac{k}{mg}}v) \\
    & \tan(-\sqrt{\frac{kg}{m}}t + c) = (\sqrt{\frac{k}{mg}}v)
\end{align}\setcounter{equation}{0}
Final
\begin{align*}
    & v(t) = \sqrt{\frac{mg}{k}} \tan(-\sqrt{\frac{kg}{m}}t + c)
\end{align*}
Plugging in $v(0) = 300$ we find c:
\begin{align}
    & v(0) = \sqrt{\frac{mg}{k}} \tan(-\sqrt{\frac{kg}{m}}(0) + c) \\
    & 300 =  \sqrt{\frac{mg}{k}}\tan(c) \\
    & c = \tan^{-1}(300)\sqrt{\frac{k}{mg}}
\end{align}
Plugging this in and the mass of 16 our final solution is:
\begin{align*}
     \Aboxed{v(t) = \sqrt{\frac{16g}{k}} \tan(-\sqrt{\frac{kg}{16}}t + \tan^{-1}(300)\sqrt{\frac{k}{16g}})}
\end{align*}
\setcounter{section}{4}
\setcounter{subsection}{0}
\subsection{}
\setcounter{subsubsection}{0}
%\subsection{problem 2)}
\subsubsection{}
We are given the equation $ x^2y;; - xy' + y = 0$ and its solution $ y = c_1x + c_2x \ln(x), (0, \infty)$ we are tasked with finding the member that satisfies y(1) = 3 and y'(1) = -1, first we derive and plug:
\begin{align*}
    & y = c_1x + c_2x\ln(x) & 3 = c_1\\
    & y' = c_1 + c_2(\ln(x) + 1) & -1 = c_1 + c_2\\
    & y'' = c_2 \frac{1}{x} 
\end{align*}
We get $c_1 = 3$ and $c_2 = -4$, our solution is:
\begin{align*}
    \Aboxed{3x -4x\ln(x)}
\end{align*}


\subsubsection{} 
We determine if $f_1 = 1 + x, f_2 = 3x, f_3 = -x^2$ are linearly independant using wronskian:
\begin{equation}
\begin{bmatrix}
    1+x & 3x & -x^2\\
    1 & 3 & -2x \\
    0 & 0 & - 2
\end{bmatrix}
  = \boxed{ -6} \\
\end{equation}
Thus the functions are linearly independant since -6 $\not=$ 0

\subsubsection{}
We verify that $y_1 = e^{\frac{x}{3}} and y_2 = xe^{\frac{x}{3}}$ form a fundamental set of solutions for $ 9y'' +6y' + y = 0$
\begin{align*}
    & y_1 = e^{\frac{x}{3}} & y_1'' = \frac{e^{\frac{x}{3}}}{3} && y_1'' = \frac{e^{\frac{x}{3}}}{9} \\
    & y_2 = xe^{\frac{x}{3}} & y_2'= \frac{1}{3} e^{\frac{x}{3}}(x+3) && y_2'' = \frac{1}{9} e^{\frac{x}{3}}(x+6) 
\end{align*}
plugging both equations into $ 9y'' +6y' + y = 0$ we get:
\begin{align*}
\end{align*}
RAN OUT OF TIME DANG
\end{document}
