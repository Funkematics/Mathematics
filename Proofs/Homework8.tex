% !TEX TS-program = pdflatexmk
\documentclass[12pt]{amsart}

%\usepackage[parfill]{parskip}    % Activate to begin paragraphs with an empty line rather than an indent

\usepackage[margin=1in]{geometry}

\usepackage{amsmath,amssymb,amsthm,latexsym,graphicx}
\usepackage[normalem]{ulem}
\usepackage{setspace} %used for doublespacing, etc.
\usepackage{hyperref}
\usepackage{cancel}
\usepackage[dvipsnames,usenames]{color}
\usepackage[all]{xy}
\usepackage{fancyhdr}
\pagestyle{fancy}
	\renewcommand{\headrulewidth}{0.5pt} % and the line
	\headsep=1cm
	
\DeclareGraphicsRule{.tif}{png}{.png}{`convert #1 `dirname #1`/`basename #1 .tif`.png}

%Some useful environments.
\newtheorem{theorem}{Theorem}
\newtheorem{corollary}[theorem]{Corollary}
\newtheorem{conjecture}[theorem]{Conjecture}
\newtheorem{lemma}[theorem]{Lemma}
\newtheorem{proposition}[theorem]{Proposition}
\newtheorem{definition}[theorem]{Definition}
\newtheorem{example}[theorem]{Example}
\newtheorem{axiom}{Axiom}
\theoremstyle{remark}
\newtheorem{remark}{Remark}
\newtheorem*{exercise}{Exercise}%[section]

%Some shortcuts helpful for our assignments
\newcommand{\bx}{\begin{exercise}}
\newcommand{\ex}{\end{exercise}}

%Some useful shortcuts for our favorite sets of numbers.
%Note, you can use these WITHOUT entering math mode
\def\RR{\ensuremath{\mathbb R}} 
\def\NN{\ensuremath{\mathbb N}}
\def\ZZ{\ensuremath{\mathbb Z}}
\def\QQ{{\ensuremath\mathbb Q}}
\def\CC{\ensuremath{\mathbb C}}
\def\EE{{\ensuremath\mathbb E}}

%Some useful shortcuts for formatting lists
\newcommand{\bc}{\begin{center}}
\newcommand{\ec}{\end{center}}
\newcommand{\be}{\begin{enumerate}}
\newcommand{\ee}{\end{enumerate}}
\newcommand{\bi}{\begin{itemize}}
\newcommand{\ei}{\end{itemize}}

%Some useful shortcuts for formatting mathematical symbols
\newcommand{\ol}[1]{\overline{#1}}
\newcommand{\oimp}[1]{\overset{#1}{\iff}} %labeled iff symbol
\newcommand{\bv}[1]{\ensuremath{ \vec{\mathbf{#1}}} } %makes a vector.
\newcommand{\mc}[1]{\ensuremath{\mathcal{#1}}} %put something in caligraphic font
\newcommand{\normale}{\trianglelefteq}
\newcommand{\normal}{\triangleleft}

%Code for formatting the proofs a little nicer for submitted homework
\makeatletter
\renewenvironment{proof}[1][\proofname]{\par\doublespacing
  \pushQED{\qed}%
  \normalfont \topsep6\p@\@plus6\p@\relax
  \list{}{%
    \settowidth{\leftmargin}{\itshape\proofname:\hskip\labelsep}%
    \setlength{\labelwidth}{0pt}%
    \setlength{\itemindent}{-\leftmargin}%
  }%
  \item[\hskip\labelsep\itshape#1\@addpunct{:}]\ignorespaces
}{%
  \popQED\endlist\@endpefalse
  \singlespacing
}
\makeatother

%Commenting tools for the professor
\newcommand{\mpg}[1]{\marginpar{ #1}} %to put comments in margins
\usepackage{soul}
\definecolor{highlight}{rgb}{1,0.6,0.6}
\sethlcolor{highlight}
\newcommand{\hlm}[1]{\colorbox{highlight}{$\displaystyle #1$}}
\newtheoremstyle{mycomment}{\topsep}{-0in}{\small \itshape \sffamily}{}{\small \itshape\sffamily}{:}{.5em}{}
\theoremstyle{mycomment}
\newtheorem*{acomment}{\color{BrickRed}{Comment}}
\newcommand{\com}[1]{{\color{OliveGreen}\begin{acomment}{#1} %#2 \color{black} 
\end{acomment}\noindent}}
\newcommand{\red}[1]{{\color{BrickRed} #1}}
\newcommand{\blue}[1]{{\color{MidnightBlue}#1}}
\newcommand{\green}[1]{{\color{OliveGreen}#1}}
\newcommand{\mwrong}[2]{\red{\cancel{#1}}\green{#2}}
\newcommand{\wrong}[2]{\red{\sout{#1}}\green{#2}}
\definecolor{OliveGreen}{rgb}{.3,.5,.2}
\definecolor{MidnightBlue}{rgb}{.3,.4,.6}
\newcommand{\pts}[1]{\hfill\blue{{#1}/5}}

\chead{MATH 265F}
\pagestyle{fancy}
%Modify these items:
\rhead{\emph{Christopher Munoz}}
\lhead{\emph{HW \#8}}

\begin{document}

\thispagestyle{fancy}

\section*{Chapter 7} Prove the following statements.
\begin{exercise}[12] There exists a positive real number $x$ for which $x^{2}<\sqrt{x}$.
\begin{proof}
  Suppose that $x = \frac{1}{4}$. Observe that substituting for $x$ in our inequality $x^2 < \sqrt{x}$ gives $(\frac{1}{4})^2 = \frac{1}{16} < \frac{1}{2} = \sqrt{\frac{1}{4}}$. Thus $x = \frac{1}{4}$ is such a positive real number.
\end{proof}
\end{exercise}

\begin{exercise}[18] There is a set $X$ for which $\NN\in X$ and $\NN\subseteq X$.
\begin{proof}
  Suppose that the set $X = \NN \cup \{\NN\}$. Observe that $\NN \in X$ and that $\NN \subseteq X$. Thus $X = \NN \cup \{\NN\}$ is such a set.
\end{proof}
\end{exercise}

\begin{exercise}[21] Every real solution of $x^{3}+x+3=0$ is irrational.
\begin{proof}
  (By Contradiction) Suppose for the sake of contradiction that there exists a rational solution to $x^{3} + x +3 = 0$, that is to say that there is an $x = \frac{a}{b}$ where $a,b \in \ZZ$ in its most reduced form such that $(\frac{a}{b})^3 + \frac{a}{b} + 3 = 0$. Observe that multiplying our equation by $b^3$ gives $a^3 + ab^2 + 3b^3 = 0$. Consider these 3 cases:\\ 
  \underline{Case 1:} Suppose $a$ is odd and $b$ is odd. Then the left-hand side is a sum of 3 odd numbers, which is odd, meaning 0 is odd. This is a contradiction. \\
  \underline{Case 2:} Suppose $a$ is odd and $b$ is even. Then the left-hand side is a sum of 2 even numbers and an odd number, meaning 0 is odd. This is also contradiction. \\
  \underline{Case 3:} Suppose $a$ is even and $b$ is odd, likewise the left-hand side is a sum of 2 even numbers and an odd number, meaning 0 is odd. This is yet again another contradiction. \\
  Thus it follows that every real solution of $x^3 + x + 3 = 0$ must be irrational.
\end{proof}
\end{exercise}

\begin{exercise}[31] If $n\in\ZZ$, then $\gcd(n,n+1)=1$.
\begin{proof}
Suppose $d$ is an integer and that $d \mid n$ and $d \mid (n+1)$. Then it follows that $d \mid (n+1) - n$ which implies $d \mid 1$. Thus the greatest common divisor of $n$ and $n+1$ is in fact $1$.
\end{proof}
\end{exercise}

\begin{exercise}[35] Suppose $a,b\in\NN$. Then $a=\gcd(a,b)$ if and only if $a\mid b$.
\begin{proof}
Suppose $a = \gcd(a,b)$. Then by definition $a \mid a$ and more importantly $a \mid b$. \\
Conversely suppose $ a \mid b$. Then it must be the case that $a \leq \gcd(a,b)$ since $a$ divides itself and $a \mid b$. Since $\gcd(a,b) \mid a$ then $a = \gcd(a,b) * x$ where $x \in \ZZ$. As all integers are positive, it follows that $a \geq \gcd(a,b)$. \\
Since $a \leq \gcd(a,b)$ and $a \geq \gcd(a,b)$, then $a = \gcd(a,b)$.
\end{proof}
\end{exercise}

\section*{Chapter 8}
Use the methods introduced in this chapter to prove the following statements.
\begin{exercise}[4] If $m,n\in\ZZ$, then $\{x\in \ZZ:mn\mid x\}\subseteq (\{x\in\ZZ:m\mid x\}\cap\{x\in\ZZ:n\mid x\}$.
\begin{proof}
Suppose $a \in \{x\in \ZZ:mn\mid x\}$. This means $a \in \ZZ$ and $mn \mid a$. By definition of divisibility, there is an integer $k$ such that $a = mn*k$. Therefore $a = m(n*k)$ and $a = n(m*k)$. From $a = m(n*k)$, it follows that $m \mid a$ so that $a \in \{x\in\ZZ:m\mid x\}$. Similarly from $a = n(m *k)$, it follows that $n \mid a$ so that $a \in \{x\in\ZZ:n\mid x \}$. Thus by the definition of the intersection of two sets, we have $a \in \{x\in\ZZ:m\mid x\}\cap\{x\in\ZZ:n\mid x\}$. Thus $\{x\in \ZZ:mn\mid x\}\subseteq (\{x\in\ZZ:m\mid x\}\cap\{x\in\ZZ:n\mid x\}$.
\end{proof}
\end{exercise}

\begin{exercise}[6] Suppose $A,B$ and $C$ are sets. Prove that if $A\subseteq B$, then $A-C\subseteq B-C$.
\begin{proof}
Suppose $A \subseteq B$. Let $x \in (A - C)$, by definition this means $x \in A \land x \notin C$. Since $x \in A$ and $A \subseteq B$, this means $x \in B$. Since $x \in B$ and $x \notin C$ it follows that $x \in B - C$. Thus $A - C \subseteq B - C$.
\end{proof}
\end{exercise}

\begin{exercise}[7] Suppose $A,B$ and $C$ are sets. If $B\subseteq C$, then $A\times B\subseteq A\times C$.
\begin{proof}
Suppose $B \subseteq C$ and let $(x,y) \in A \times B$. Then by definition of the Cartesian product $x \in A$ and $y \in B$. Since $B \subseteq C$ it follows that $y \in C$. Thus $x \in A$ and $y \in C$ implies $(x,y) \in A \times C$. Therefore $(x,y) \in A \times B$ implies $(x,y) \in A \times C$. Hence $A \times B \subseteq A \times C$.
\end{proof}
\end{exercise}

\begin{exercise}[9] If $A,B$ and $C$ are sets, then $A\cap(B\cup C)=(A\cap B)\cup (A\cap C)$.
\begin{proof}
Write your answer here.
\end{proof}
\end{exercise}

\begin{exercise}[10] If $A$ and $B$ are sets in a universal set $U$, then $\overline{A\cap B}=\overline A\cup \overline B$.
\begin{proof}
Write your answer here.
\end{proof}
\end{exercise}

\begin{exercise}[14] If $A,B$ and $C$ are sets, then $(A\cup B)-C=(A-C)\cup (B-C)$.
\begin{proof}
Write your answer here.
\end{proof}
\end{exercise}



\begin{exercise}[Reflection Problem]

\begin{itemize}
\item How long did it take you to complete each problem? What part of the assignment took the most time? Why?
\begin{proof}[Response]
Write your answer here.
\end{proof}
\item What was easy for you? Why do you think that was so?
\begin{proof}[Response]
Write your answer here.
\end{proof}
\item What was challenging for you? What made it challenging?
\begin{proof}[Response]
Write your answer here.
\end{proof}
\item Compare your answers to the odd numbered exercises to those in the back of the textbook. What did you learn from this comparison?
\begin{proof}[Response]
Write your answer here.
\end{proof}
\end{itemize}

\end{exercise}













 \end{document} 
