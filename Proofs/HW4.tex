% !TEX TS-program = pdflatexmk
\documentclass[12pt]{amsart}

%\usepackage[parfill]{parskip}    % Activate to begin paragraphs with an empty line rather than an indent

\usepackage[margin=1in]{geometry}

\usepackage{amsmath,amssymb,amsthm,latexsym,graphicx}
\usepackage[normalem]{ulem}
\usepackage{setspace} %used for doublespacing, etc.
\usepackage{hyperref}
\usepackage{cancel}
\usepackage[dvipsnames,usenames]{color}
\usepackage[all]{xy}
\usepackage{fancyhdr}
\pagestyle{fancy}
	\renewcommand{\headrulewidth}{0.5pt} % and the line
	\headsep=1cm
	
\DeclareGraphicsRule{.tif}{png}{.png}{`convert #1 `dirname #1`/`basename #1 .tif`.png}

%Some useful environments.
\newtheorem{theorem}{Theorem}
\newtheorem{corollary}[theorem]{Corollary}
\newtheorem{conjecture}[theorem]{Conjecture}
\newtheorem{lemma}[theorem]{Lemma}
\newtheorem{proposition}[theorem]{Proposition}
\newtheorem{definition}[theorem]{Definition}
\newtheorem{example}[theorem]{Example}
\newtheorem{axiom}{Axiom}
\theoremstyle{remark}
\newtheorem{remark}{Remark}
\newtheorem*{exercise}{Exercise}%[section]

%Some useful shortcuts for our favorite sets of numbers
\newcommand{\RR}{\ensuremath{\mathbb R}} %Note, you can use these WITHOUT entering math mode
\newcommand{\NN}{\ensuremath{\mathbb N}}
\newcommand{\ZZ}{\ensuremath{\mathbb Z}}
\newcommand{\QQ}{{\ensuremath\mathbb Q}}
\newcommand{\CC}{\ensuremath{\mathbb C}}
\newcommand{\EE}{{\ensuremath\mathbb E}}
\newcommand{\PP}{{\ensuremath\mathbb P}}

%Some useful shortcuts for formatting lists
\newcommand{\bc}{\begin{center}}
\newcommand{\ec}{\end{center}}
\newcommand{\be}{\begin{enumerate}}
\newcommand{\ee}{\end{enumerate}}
\newcommand{\bi}{\begin{itemize}}
\newcommand{\ei}{\end{itemize}}

%Some useful shortcuts for formatting mathematical symbols
\newcommand{\ol}[1]{\overline{#1}}
\newcommand{\oimp}[1]{\overset{#1}{\iff}} %labeled iff symbol
\newcommand{\bv}[1]{\ensuremath{ \vec{\mathbf{#1}}} } %makes a vector.
\newcommand{\mc}[1]{\ensuremath{\mathcal{#1}}} %put something in caligraphic font
\newcommand{\mpg}[1]{\marginpar{ #1}} %to put comments in margins
\newcommand{\bsl}[1]{\texttt{\symbol{92}{\em #1}}} %for backslashes.
\newcommand{\normale}{\trianglelefteq}
\newcommand{\normal}{\triangleleft}

%Commenting tools --- You can ignore these, but if you have a question about latex and send me your source file, I'll use them to explain stuff to you.
\usepackage{soul}
\definecolor{highlight}{rgb}{1,0.6,0.6}
\sethlcolor{highlight}
\newcommand{\hlm}[1]{\colorbox{highlight}{$\displaystyle #1$}}
\newtheoremstyle{mycomment}{\topsep}{-0in}{\small \itshape \sffamily}{}{\small \itshape\sffamily}{:}{.5em}{}
\theoremstyle{mycomment}
\newtheorem*{acomment}{\color{BrickRed}{Comment}}
\newcommand{\com}[1]{{\color{OliveGreen}\begin{acomment}{#1} %#2 \color{black} 
\end{acomment}\noindent}}
%\newcommand{\com}[1]{{\color{BrickRed}{\\ Comment:}\color{OliveGreen}{#1} \\}}
\newcommand{\red}[1]{{\color{BrickRed} #1}}
\newcommand{\blue}[1]{{\color{MidnightBlue}#1}}
\newcommand{\green}[1]{{\color{OliveGreen}#1}}
\newcommand{\mwrong}[2]{\red{\cancel{#1}}\green{#2}}
\newcommand{\wrong}[2]{\red{\sout{#1}}\green{#2}}
\definecolor{OliveGreen}{rgb}{.3,.5,.2}
\definecolor{MidnightBlue}{rgb}{.3,.4,.6}
\newcommand{\pts}[1]{\hfill\blue{{#1}/5}}

\chead{MATH F401}
\pagestyle{fancy}
%Modify these items:
\rhead{\emph{Your Full Name Here}}
\lhead{\emph{HW \#0 --- 3/14/15}}


\begin{document}

\thispagestyle{fancy}
\setstretch{2.5} %Use for 2.5 spacing
%\doublespacing %Use for double spacing

\textbf{Write the following in English Sentences. Say whether they are true or false.}
\begin{exercise}[2.7.1] $\forall x \in \RR, x^2 > 0$ \newline 
For all x in the real numbers, $x^2$ is greater than 0. \newline
This is false, for example if we let $x = 0$, then $x > 0$.
\end{exercise}
\begin{exercise}[2.7.2] $\forall x \in \RR, \exists n \in \NN, x^n \geq 0$ \newline
	For all x in the real numbers, there exists an n in the natural numbers such that $x^n \geq 0$. \newline
	This is true, if we let $n = 2$ then any x will be above or equal to zero.
\end{exercise}
\begin{exercise}[2.7.3] $\exists a \in \RR, \forall x \in \RR, ax=x$ \newline
	There exists a real number a, such that for all real numbers x, $ax = x.$ \newline
	This is true, if we let $a = 1$ then $ax = 1x = x$.
\end{exercise}
\begin{exercise}[2.7.4] $\forall X \in \mathcal{P} (\NN), X \subseteq \RR$ \newline
	For all X in the powerset of the natural numbers, X is a subset of the Real Numbers. \newline
	This is a true statement because every subset of the natural numbers is a subset of the real numbers.
\end{exercise}
\textbf{For the following, we're just staying whether its true or false.}
\begin{exercise}[2.7.5] $\forall n \in \NN, \exists X \in \mathcal{P} (\NN), |X| = n $ \newline
	True
\end{exercise}
\begin{exercise}[2.7.6] $\exists n \in \NN, \forall X \in \mathcal{P} (\NN),$ \newline
	False
\end{exercise}
\begin{exercise}[2.7.7] $\forall X \subseteq \NN, \exists n \in \ZZ, |X| = n$ \newline
	False
\end{exercise}
\begin{exercise}[2.7.8] $\forall n \in \ZZ, \exists X \subseteq \NN, |X| = n$ \newline
	False
\end{exercise}
\begin{exercise}[2.7.9] $\forall n \in \ZZ, \exists m \in \ZZ, m = n +5$ \newline 
	True
\end{exercise}
\begin{exercise}[2.7.10] $\exists m \in \ZZ, \forall n \in \ZZ, m = n+5$ \newline
True
\end{exercise}
\textbf{For these we are translating sentences into logic}
\begin{exercise}[2.9.1] if $f$ is a polynomial and its degree is greater than 2, then $f'$ is not constant.
	P: $f$ is a polynomial \newline
	Q: $f$ has a degree greater than 2 \newline
	R: $f'$ is constant \newline
	Translation: (P $\land$ Q) $\Rightarrow \neg$ R
\end{exercise}
\begin{exercise}[2.9.4]  For every prime number $p$, there is another prime number $q$ with $q > p$ \newline
	let $\PP$ be the set of prime numbers. \newline
	Translation: $\forall p \in \PP, \exists q \in \PP, q > p$  
\end{exercise}
\begin{exercise}[2.9.6] For every positive number $\varepsilon$, there is a positive number $M$ for which \newline $|f(x) - b| < \varepsilon, $ whenever $x > M$
\end{exercise}
\begin{exercise}[2.9.7]
\end{exercise}
 \end{document}
 \end

  
