% !TEX TS-program = pdflatexmk
\documentclass[12pt]{amsart}

%\usepackage[parfill]{parskip}    % Activate to begin paragraphs with an empty line rather than an indent

\usepackage[margin=1in]{geometry}

\usepackage{amsmath,amssymb,amsthm,latexsym,graphicx}
\usepackage[normalem]{ulem}
\usepackage{setspace} %used for doublespacing, etc.
\usepackage{hyperref}
\usepackage{cancel}
\usepackage[dvipsnames,usenames]{color}
\usepackage[all]{xy}
\usepackage{fancyhdr}
\pagestyle{fancy}
	\renewcommand{\headrulewidth}{0.5pt} % and the line
	\headsep=1cm
	
\DeclareGraphicsRule{.tif}{png}{.png}{`convert #1 `dirname #1`/`basename #1 .tif`.png}

%Some useful environments.
\newtheorem{theorem}{Theorem}
\newtheorem{corollary}[theorem]{Corollary}
\newtheorem{conjecture}[theorem]{Conjecture}
\newtheorem{lemma}[theorem]{Lemma}
\newtheorem{proposition}[theorem]{Proposition}
\newtheorem{definition}[theorem]{Definition}
\newtheorem{example}[theorem]{Example}
\newtheorem{axiom}{Axiom}
\theoremstyle{remark}
\newtheorem{remark}{Remark}
\newtheorem*{exercise}{Exercise}%[section]

%Some shortcuts helpful for our assignments
\newcommand{\bx}{\begin{exercise}}
\newcommand{\ex}{\end{exercise}}

%Some useful shortcuts for our favorite sets of numbers.
%Note, you can use these WITHOUT entering math mode
\def\RR{\ensuremath{\mathbb R}} 
\def\NN{\ensuremath{\mathbb N}}
\def\ZZ{\ensuremath{\mathbb Z}}
\def\QQ{{\ensuremath\mathbb Q}}
\def\CC{\ensuremath{\mathbb C}}
\def\EE{{\ensuremath\mathbb E}}

%Some useful shortcuts for formatting lists
\newcommand{\bc}{\begin{center}}
\newcommand{\ec}{\end{center}}
\newcommand{\be}{\begin{enumerate}}
\newcommand{\ee}{\end{enumerate}}
\newcommand{\bi}{\begin{itemize}}
\newcommand{\ei}{\end{itemize}}

%Some useful shortcuts for formatting mathematical symbols
\newcommand{\ol}[1]{\overline{#1}}
\newcommand{\oimp}[1]{\overset{#1}{\iff}} %labeled iff symbol
\newcommand{\bv}[1]{\ensuremath{ \vec{\mathbf{#1}}} } %makes a vector.
\newcommand{\mc}[1]{\ensuremath{\mathcal{#1}}} %put something in caligraphic font
\newcommand{\normale}{\trianglelefteq}
\newcommand{\normal}{\triangleleft}

%Code for formatting the proofs a little nicer for submitted homework
\makeatletter
\renewenvironment{proof}[1][\proofname]{\par\doublespacing
  \pushQED{\qed}%
  \normalfont \topsep6\p@\@plus6\p@\relax
  \list{}{%
    \settowidth{\leftmargin}{\itshape\proofname:\hskip\labelsep}%
    \setlength{\labelwidth}{0pt}%
    \setlength{\itemindent}{-\leftmargin}%
  }%
  \item[\hskip\labelsep\itshape#1\@addpunct{:}]\ignorespaces
}{%
  \popQED\endlist\@endpefalse
  \singlespacing
}
\makeatother

%Commenting tools for the professor
\newcommand{\mpg}[1]{\marginpar{ #1}} %to put comments in margins
\usepackage{soul}
\definecolor{highlight}{rgb}{1,0.6,0.6}
\sethlcolor{highlight}
\newcommand{\hlm}[1]{\colorbox{highlight}{$\displaystyle #1$}}
\newtheoremstyle{mycomment}{\topsep}{-0in}{\small \itshape \sffamily}{}{\small \itshape\sffamily}{:}{.5em}{}
\theoremstyle{mycomment}
\newtheorem*{acomment}{\color{BrickRed}{Comment}}
\newcommand{\com}[1]{{\color{OliveGreen}\begin{acomment}{#1} %#2 \color{black} 
\end{acomment}\noindent}}
\newcommand{\red}[1]{{\color{BrickRed} #1}}
\newcommand{\blue}[1]{{\color{MidnightBlue}#1}}
\newcommand{\green}[1]{{\color{OliveGreen}#1}}
\newcommand{\mwrong}[2]{\red{\cancel{#1}}\green{#2}}
\newcommand{\wrong}[2]{\red{\sout{#1}}\green{#2}}
\definecolor{OliveGreen}{rgb}{.3,.5,.2}
\definecolor{MidnightBlue}{rgb}{.3,.4,.6}
\newcommand{\pts}[1]{\hfill\blue{{#1}/5}}

\chead{MATH 265F}
\pagestyle{fancy}
%Modify these items:
\rhead{\emph{Christopher Munoz}}
\lhead{\emph{HW \#7}}

\begin{document}

\thispagestyle{fancy}

\section*{Chapter 6}
\begin{exercise}[2] Suppose $n\in\ZZ$. If $n^{2}$ is odd, then $n$ is odd.
\begin{proof}
  Suppose for the sake of contradiction that $n^2$ is odd and $n$ is not odd, Then $n$ is even, so $n = 2k$ for some $k\in\ZZ$. Therefore $n^2 = (2k)^2 = 4k^2 = 2(2k^2) = 2b$, where $b\in\ZZ$ by closure properties of the integers. So $n^2$ is even, this is a contradiction. So it must be the case that if $n^2$ is odd then $n$ is odd. 
\end{proof}
\end{exercise}

\begin{exercise}[3] Prove that $\sqrt[3]{2}$ is irrational.
\begin{proof}
  Suppose for the sake of contradiction that $\sqrt[3]{2}$ is not irrational. Then $\sqrt[3]{2}$ is a rational and is in the form $\sqrt[3]{2} = \frac{a}{b}$ where $a,b\in\ZZ$ and the ratio $a,b$ do not sure factors so that $\frac{a}{b}$ is in its lowest form. Observe that when cubing both sides, $2 = (\frac{a}{b})^3 = \frac{a^3}{b^3}$ so that $2b^3 = a^3$. This implies that $a$ is an even number and divisible by $2$. So $a = 2k$ for some $k\in\ZZ$. Substituting for $a$ gives $2b^3 = (2k)^3 = 8k^3$. Thus $b^3 = 4k^3 = 2(2k^3)$.  Because $b^3$ is even, this implies that $b$ is also an even number which is contraditory to $\frac{a}{b}$ existing in its lowest forms. Thus $\sqrt[3]{2}$ must be irrational.
\end{proof}
\end{exercise}

\begin{exercise}[4] Prove that $\sqrt{6}$ is irrational.
\begin{proof}
  Suppose for the sake of contradiction $\sqrt{6}$ is a rational number. Then there exists an $a,b\in\ZZ$ such that $\sqrt{6} = \frac{a}{b}$ and $\frac{a}{b}$ is in its lowest form with no common factors. Squaring both sides gives $6 = (\frac{a}{b})^2 = \frac{a^2}{b^2}$. Thus $6b^2 = a^2$ which implies that $a^2$ is divisible by $6$ and thus $a$ is divisible by $6$ or $a = 6k$ for some $k\in\ZZ$. Substituting in our original equation gives $6b^2 = (6k)^2 = 36k^2$. Simplifying this equation by division of $6$ shows that $b^2 = 6k^2$, this implies that $b^2$ is divisible by $6$. Since it follows that $b$ is also divisible by $6$ and that $a$ is divisible by $6$ we have a contradiction as $\frac{a}{b}$ cannot be in its lowest form. Thus $\sqrt{6}$ is an irrational number. 
\end{proof}
\end{exercise}

\begin{exercise}[8] Suppose $a,b,c\in\ZZ$. If $a^{2}+b^{2}=c^{2}$, then $a$ or $b$ is even.
\begin{proof}
For the sake of contradiction, suppose that $a,b,c\in\ZZ$ such that $a^2=b^2=c^2$ and $a$ and $b$ are odd. Then there exists an $k,j\in\ZZ$ such that $a = 2k + 1$ and $b = 2j+1$. We know that an odd number added to another odd number is an even number, so that $c^2$ is even and thus $c$ is even so that, $c = 2x$ for some $x\in\ZZ$. Observe that when we substitute $a,b,c$ on both sides of the equations, $a^2 + b^2 = (2k+1)^2 + (2j+1)^2 = 4k^2 + 4k + 1 + 4j^2 + 4j + 1 = 4k^2 + 4j^2 + 4k + 4j + 2 = 2(2k^2 + 2j^2 + 2k + 2j + 1)$ and $c^2 = (2x)^2 = 4x^2$, so that $2(k^2 + 2j^2 + 2k + 2j + 1) = 4x^2$. Dividing both sides by 2 gives $2k^2 + 2j^2 + 2k + 2j + 1 = 2x^2$. Note that the left hand side is odd and the right hand side is even. This is an impossibility since an odd number cannot equal an even number. Thus if $a^2 + b^2=c^2$, then $a$ or $b$ must be even.
\end{proof}
\end{exercise}

\begin{exercise}[9] Suppose $a,b\in\RR$. If $a$ is rational and $ab$ is irrational, then $b$ is irrational.
\begin{proof}
  Suppose for the sake of contradiction that $a$ is rational, $ab$ is irrational, and $b$ is not irrational. Then $b$ is a rational number and there exists a $c,d \in \ZZ$ such that $b = \frac{c}{d}$. Similarly there exists an $x,y \in \ZZ$ such that $a = \frac{x}{y}$ since $a$ is rational. Observe that $ab = \frac{cx}{dy}$ and that $cx$ and $dy$ are integers by closure properties of the integers. Then $ab$ is a rational number by definition which
  is a contradiction to our original premise that $ab$ is irrational. Thus if $a$ is rational and $ab$ is irrational, then $b$ must be irrational.
\end{proof}
\end{exercise}

\begin{exercise}[11] There exist no integers $a$ and $b$ for which $18a+6b=1$.
\begin{proof}
Suppose for the sake of contradiction that there does exist integers $a$ and $b$ for which $18a + 6b = 1$.
Then $2(9a + 3b) = 1$ which means $1$ even, a contradiction. Thus there exists no $a,b\in\ZZ$ that satisfies $18a + 6b = 1$.
\end{proof}
\end{exercise}

\begin{exercise}[12] For every positive $x\in\QQ$, there is a positive $y\in\QQ$ for which $y<x$.
\begin{proof}
  Suppose for the sake of contradiction that there exists a positive $x\in\QQ$, such that for all positive $y\in\QQ$ that $y \geq x$. Lets consider the possibility that $ y = \frac{x}{2}$, then $y = \frac{x}{2} \geq x$. This is a contradiction because obviously $x > \frac{x}{2}$.
\end{proof}
\end{exercise}

\begin{exercise}[16] If $a$ and $b$ are positive real numbers, then $a+b\ge 2\sqrt{ab}$.
\begin{proof}
  Suppose for the sake of contradiction that $a$ and $b$ are positive real numbers and that $a+b\ge 2\sqrt{ab}$ is false. That is to say $a+b < 2\sqrt{ab}$. Squaring both sides gives $(a + b)^2 = a^2 + 2ab + b^2 < (2\sqrt{ab})^2 = 4ab$. Subtracting both sides of the inequality by $4ab$ gives $a^2 - ab^2 + b^2 = (a-b)^2 < 0$. This is a contradiction since any real number squared must be greater than or equal to $0$, $(a-b)^2$ cannot be less than $0$.
\end{proof}
\end{exercise}

\begin{exercise}[19] The product of any five consecutive integers is divisible by 120. (For example, the product of 3, 4, 5, 6 and 7 is 2520, and $2520=120\cdot 21$.)
\begin{proof}
  Suppose we have a product of $5$ consecutive integers, this product of consecutive integers may be expressed as $n(n-1)(n-2)(n-3)(n-4)$ for some $n\in\ZZ$. Observe that $\binom{n}{5}$ is an integer and that $\binom{n}{5} = \frac{n!}{5!(n-5)!} = \frac{n!}{120(n-5)!} = \frac{n(n-1)(n-2)(n-3)(n-4)}{120}$. Thus $120$ divides our product of $5$ consecutive integers.
\end{proof}
\end{exercise}

\section*{Chapter 7}
\begin{exercise}[1] Suppose $x\in\ZZ$. Then $x$ is even if and only if $3x+5$ is odd.
\begin{proof}
Suppose that $x$ is even, then $x = 2k$ for some $k\in\ZZ$. Substituting for x, $3x + 5 = 3(2k) + 5 = 6k + 5 = 2(3k + 2) + 1$, an odd number. Showing that if $x$ is even then $3x + 5$ is odd. Conversely using contraposition, suppose that $x$ is not even. Then $x$ is odd and there exists a $k\in\ZZ$ such that $x=2k+1$. Substituting for $x$ gives $3x + 5 = 3(2k+1) + 5 = 6k+ 3 + 5 = 6k + 8 = 2(3k+4)$, an even number. Showing that if $x$ is odd then $3x+5$ is even. By contraposition it must be the case that if $3x+5$ is odd then $x$ is even.
\end{proof}
\end{exercise}
\begin{exercise}[4] Let $a$ be an integer. Then $a^{2}+4a+5$ is odd if and only if $a$ is even.
\begin{proof}
Lets suppose that $a^2 + 4a + 5$ is odd, then $a^2 + 4a + 5 = 2k+1$ for some $k\in\ZZ$. When we isolate $a^2$ we get $a^2 = 2k - 4a - 4 = 2(k-2a-2)$, so $a^2$ is even. This implies that $a$ is even. Thus if $a^2 + 4a + 5$ is odd then $a$ is even. Conversely if we suppose by contraposition that $a$ is odd, then there exists a $k\in\ZZ$ such that $a = 2k+1$. Thus $a^2 +4a + 5 = (2k+1)^2 + 4(2k+1) + 5 = 4k^2 + 4k + 1 + 8k + 4 + 5 = 4k^2 + 12k + 9 = 2(2k^2 + 6k + 5)$, an even number. Since when $a$ is odd, $a^2 + 4a + 5$ is even, it must follow by contraposition that if $a^2 + 4a + 5$ is odd then $a$ is even. \end{proof}
\end{exercise}
\begin{exercise}[7] Suppose $x,y\in\RR$. Then $(x+y)^{2}=x^{2}+y^{2}$ if and only if $x=0$ or $y=0$.
\begin{proof}
Suppose that $(x+y)^2 = x^2 + y^2$. Expanding out gives $x^2 + 2xy + y^2 = x^2 + y^2$, so $2xy = 0$. It follows that $xy = 0$. Thus either $x=0$ or $y= 0$. Conversely lets suppose $x=0$ or $y=0$, then we have $2$ cases to show. \\
\underline{Case 1:} Suppose $x = 0$, then $(x+y)^2 = 0 + y^2 = y^2$ and $x^2 + y^2 =  0 + y^2 = y^2$. So $(x+y)^2 = x^2 + y^2$ holds. \\
\underline{Case 2:} Suppose $y = 0$, then $(x+y)^2 = x^2 + 0 = x^2$ and $x^2 + y^2 = x^2 + 0 = x^2$. So $(x+y)^2 = x^2 + y^2$ holds for this case as well.\\
In all cases where either $x = 0$ or $y = 0$, the equation $(x+y)^2 = x^2 + y^2$ holds true.
\end{proof}
\end{exercise}

\begin{exercise}[Reflection Problem]
\begin{proof}
Length of time: This too me significantly less time than the previous homework, 1 to 2 minutes on some being able to quickly deduce strategy and what moves to make. Its often the case that more time was spent figuring out how I was going to write it down rather than finding an argument itself. \\

Difficulty: Again significantlly easier time than the previous homeworks, I would say the most challenging problem was actually problem 19 in chapter 6. I was not able to provide an argument without some assistance from the back of the book. My initial strategy was to multiply out $n(n+1)(n+2)(n+3)(n+4)$ but it did not give me anything oobvious I could work with. I even attempted to express consecutive integers as $n(n-1)(n+1)(n-2)(n+2)$ but ran into a similar wall. \\

Challenges:Besides problem 19, I had challenges in balancing the time, but I am proud to say I've caught up after falling behind.

Comparison: Again, aped Problem 19, my proofs involving irrationality I questioned after looking at the back. But I've come around to sticking with what I wrote.
\end{proof}
\end{exercise}.















 \end{document} 
