% !TEX TS-program = pdflatexmk
\documentclass[12pt]{amsart}

%\usepackage[parfill]{parskip}    % Activate to begin paragraphs with an empty line rather than an indent

\usepackage[margin=1in]{geometry}

\usepackage{amsmath,amssymb,amsthm,latexsym,graphicx}
\usepackage[normalem]{ulem}
\usepackage{setspace} %used for doublespacing, etc.
\usepackage{hyperref}
\usepackage{cancel}
\usepackage[dvipsnames,usenames]{color}
\usepackage[all]{xy}
\usepackage{fancyhdr}
\pagestyle{fancy}
	\renewcommand{\headrulewidth}{0.5pt} % and the line
	\headsep=1cm
	
\DeclareGraphicsRule{.tif}{png}{.png}{`convert #1 `dirname #1`/`basename #1 .tif`.png}

%Some useful environments.
\newtheorem{theorem}{Theorem}
\newtheorem{corollary}[theorem]{Corollary}
\newtheorem{conjecture}[theorem]{Conjecture}
\newtheorem{lemma}[theorem]{Lemma}
\newtheorem{proposition}[theorem]{Proposition}
\newtheorem{definition}[theorem]{Definition}
\newtheorem{example}[theorem]{Example}
\newtheorem{axiom}{Axiom}
\theoremstyle{remark}
\newtheorem{remark}{Remark}
\newtheorem*{exercise}{Exercise}%[section]

%Some shortcuts helpful for our assignments
\newcommand{\bx}{\begin{exercise}}
\newcommand{\ex}{\end{exercise}}

%Some useful shortcuts for our favorite fields
%Note, you can use these WITHOUT entering math mode
\def\RR{\ensuremath{\mathbb R}} 
\def\NN{\ensuremath{\mathbb N}}
\def\ZZ{\ensuremath{\mathbb Z}}
\def\QQ{{\ensuremath\mathbb Q}}
\def\CC{\ensuremath{\mathbb C}}
\def\EE{{\ensuremath\mathbb E}}

%Some useful shortcuts for formatting lists
\newcommand{\bc}{\begin{center}}
\newcommand{\ec}{\end{center}}
\newcommand{\be}{\begin{enumerate}}
\newcommand{\ee}{\end{enumerate}}
\newcommand{\bi}{\begin{itemize}}
\newcommand{\ei}{\end{itemize}}

%Some useful shortcuts for formatting mathematical symbols
\newcommand{\ol}[1]{\overline{#1}}
\newcommand{\oimp}[1]{\overset{#1}{\iff}} %labeled iff symbol
\newcommand{\bv}[1]{\ensuremath{ \vec{\mathbf{#1}}} } %makes a vector.
\newcommand{\mc}[1]{\ensuremath{\mathcal{#1}}} %put something in caligraphic font
\newcommand{\normale}{\trianglelefteq}
\newcommand{\normal}{\triangleleft}

%Commenting tools for the professor
\newcommand{\mpg}[1]{\marginpar{ #1}} %to put comments in margins
\usepackage{soul}
\definecolor{highlight}{rgb}{1,0.6,0.6}
\sethlcolor{highlight}
\newcommand{\hlm}[1]{\colorbox{highlight}{$\displaystyle #1$}}
\newtheoremstyle{mycomment}{\topsep}{-0in}{\small \itshape \sffamily}{}{\small \itshape\sffamily}{:}{.5em}{}
\theoremstyle{mycomment}
\newtheorem*{acomment}{\color{BrickRed}{Comment}}
\newcommand{\com}[1]{{\color{OliveGreen}\begin{acomment}{#1} %#2 \color{black} 
\end{acomment}\noindent}}
\newcommand{\red}[1]{{\color{BrickRed} #1}}
\newcommand{\blue}[1]{{\color{MidnightBlue}#1}}
\newcommand{\green}[1]{{\color{OliveGreen}#1}}
\newcommand{\mwrong}[2]{\red{\cancel{#1}}\green{#2}}
\newcommand{\wrong}[2]{\red{\sout{#1}}\green{#2}}
\definecolor{OliveGreen}{rgb}{.3,.5,.2}
\definecolor{MidnightBlue}{rgb}{.3,.4,.6}
\newcommand{\pts}[1]{\hfill\blue{{#1}/5}}

\chead{MATH 265F}
\pagestyle{fancy}
%Modify these items:
\rhead{\emph{Christopher Munoz}}
\lhead{\emph{HW \#5 --- 2/19/25}}

\begin{document}

\thispagestyle{fancy}

\section*{Chapter 4} \doublespace
\begin{exercise}[2] If $x$ is an odd integer, then $x^{3}$ is odd.
\begin{proof}
Suppose $x$ is an odd integer. Then by definition of an odd integer, $x = 2k + 1$ for some $k \in \ZZ$. Therefore $x^3 = (2k+1)(2k+1)(2k+1) = (4k^2 + 4k +1)(2k+1) = 8k^3 + 12k^2 + 6k + 1 = 2(4k^3+6k^2 + 3k) + 1  = 2n + 1$, where n = $(4k^3 + 6k^2 + 3k)$. Note that $n$ is an integer due to the closure properties under addition and multiplication in the integers. So $x^3 = 2n+1$, where $n$ is an integer. Thus $x^3$ is odd by definition of an odd number.
\end{proof}
\end{exercise} 

\begin{exercise}[4] Suppose $x,y\in\mathbb Z$. If $x$ and $y$ are odd, then $xy$ is odd.
%I'm showing here how you would enter the integers if you didn't have my nice shortcut at the top. In future, I'll probably just use my shortcut, so if I want the symbol for the integers I would just type \ZZ or $q\in\ZZ$.
\begin{proof} 
Suppose $x$ and $y$ are odd integers. Then $x = 2m+1$ for some $m \in \ZZ$ and $y = 2n+1$ for some $n \in \ZZ$ by definition of odd. Therefore $xy = (2m+1)(2n+1) = 4mn + 2m + 2n + 1 = 2(2mn + m + n) + 1 = 2p+1$, where $p = (2mn + m + n)$. Note that $p$ is an integer due to the closure properties under addition and multiplication in the integers. So $xy = 2p + 1$, where p is an integer. Thus $xy$ is odd by definition of an odd number. 
\end{proof}
\end{exercise}


\begin{exercise}[6] Suppose $a,b,c\in\ZZ$. If $a\mid b$ and $a\mid c$, then $a\mid (b+c)$. %Notice my use of the shortcut to get the symbols for the integers. If you don't have my shortcut in the header of your file, you would have to type $\mathbb Z$.
\begin{proof}
Suppose $a \mid b$ and $a \mid c$ and $a,b,c \in \ZZ$. By definition of divisibility, we know that $a \mid b$ means $ b = ak$ for some $k \in \ZZ$. Likewise  $a \mid c$ means $c = al$ for some $l \in \ZZ$. Therefore $(b + c) = ak + al = a(k + l) = am$, where $m = k + l$. Note that $m$ is an integer due to the closure properties under addition and multiplication in the integers. So $(b + c) = am$ where $m$ is an integer. Thus $a \mid (b + c)$ by definition of divisibility. 
\end{proof}
\end{exercise}

\begin{exercise}[11] Suppose $a,b,c,d\in\ZZ$. If $a\mid b$ and $c\mid d$, then $ac\mid bd$.
\begin{proof}
Suppose $ a \mid b $ and $c \mid d$ and $a,b,c \in \ZZ$. By definition of divisibility, we know that $a \mid b$ means $b = ak$ for some $k \in \ZZ$. Likewise we know that $c \mid d$ means  $d = cl$ for some $l \in \ZZ$. Thus $bd = akcl = ac(kl) = acn$, where $n = kl$. Note that $n$ is an integer due to the closure properties under multiplication in the integers. So $bd = acn$ where $n \in \ZZ$. Thus $ac \mid bd$ by definition of divisibility.
\end{proof}
\end{exercise}
\begin{exercise}[12] If $x\in \RR$ and $0<x<4$, then $\frac{4}{x(4-x)}\ge 1$.
\begin{proof}
Suppose $x \in \RR$ and $0 < x < 4$, we know that any real number squared is greater than or equal to $0$. Let us choose a real number $(x-2)$ in the interval $0 < x < 4$. Therefore $(x-2)^2 \geq 0$ is equivalent to $x^2 - 4x + 4 \geq 0$ which can be rewiretten as $ 4 \geq x(4-x)$. Dividing both sides by $x(4-x)$ we obtain $\frac{4}{x(4-x)} \geq 1$. Thus the statement holds.
\end{proof}
\end{exercise}

 \end{document} 
