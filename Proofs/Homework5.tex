% !TEX TS-program = pdflatexmk
\documentclass[12pt]{amsart}

%\usepackage[parfill]{parskip}    % Activate to begin paragraphs with an empty line rather than an indent

\usepackage[margin=1in]{geometry}

\usepackage{amsmath,amssymb,amsthm,latexsym,graphicx}
\usepackage[normalem]{ulem}
\usepackage{setspace} %used for doublespacing, etc.
\usepackage{hyperref}
\usepackage{cancel}
\usepackage[dvipsnames,usenames]{color}
\usepackage[all]{xy}
\usepackage{fancyhdr}
\pagestyle{fancy}
	\renewcommand{\headrulewidth}{0.5pt} % and the line
	\headsep=1cm
	
\DeclareGraphicsRule{.tif}{png}{.png}{`convert #1 `dirname #1`/`basename #1 .tif`.png}

%Some useful environments.
\newtheorem{theorem}{Theorem}
\newtheorem{corollary}[theorem]{Corollary}
\newtheorem{conjecture}[theorem]{Conjecture}
\newtheorem{lemma}[theorem]{Lemma}
\newtheorem{proposition}[theorem]{Proposition}
\newtheorem{definition}[theorem]{Definition}
\newtheorem{example}[theorem]{Example}
\newtheorem{axiom}{Axiom}
\theoremstyle{remark}
\newtheorem{remark}{Remark}
\newtheorem*{exercise}{Exercise}%[section]

%Some shortcuts helpful for our assignments
\newcommand{\bx}{\begin{exercise}}
\newcommand{\ex}{\end{exercise}}

%Some useful shortcuts for our favorite sets of numbers.
%Note, you can use these WITHOUT entering math mode
\def\RR{\ensuremath{\mathbb R}} 
\def\NN{\ensuremath{\mathbb N}}
\def\ZZ{\ensuremath{\mathbb Z}}
\def\QQ{{\ensuremath\mathbb Q}}
\def\CC{\ensuremath{\mathbb C}}
\def\EE{{\ensuremath\mathbb E}}

%Some useful shortcuts for formatting lists
\newcommand{\bc}{\begin{center}}
\newcommand{\ec}{\end{center}}
\newcommand{\be}{\begin{enumerate}}
\newcommand{\ee}{\end{enumerate}}
\newcommand{\bi}{\begin{itemize}}
\newcommand{\ei}{\end{itemize}}

%Some useful shortcuts for formatting mathematical symbols
\newcommand{\ol}[1]{\overline{#1}}
\newcommand{\oimp}[1]{\overset{#1}{\iff}} %labeled iff symbol
\newcommand{\bv}[1]{\ensuremath{ \vec{\mathbf{#1}}} } %makes a vector.
\newcommand{\mc}[1]{\ensuremath{\mathcal{#1}}} %put something in caligraphic font
\newcommand{\normale}{\trianglelefteq}
\newcommand{\normal}{\triangleleft}

%Commenting tools for the professor
\newcommand{\mpg}[1]{\marginpar{ #1}} %to put comments in margins
\usepackage{soul}
\definecolor{highlight}{rgb}{1,0.6,0.6}
\sethlcolor{highlight}
\newcommand{\hlm}[1]{\colorbox{highlight}{$\displaystyle #1$}}
\newtheoremstyle{mycomment}{\topsep}{-0in}{\small \itshape \sffamily}{}{\small \itshape\sffamily}{:}{.5em}{}
\theoremstyle{mycomment}
\newtheorem*{acomment}{\color{BrickRed}{Comment}}
\newcommand{\com}[1]{{\color{OliveGreen}\begin{acomment}{#1} %#2 \color{black} 
\end{acomment}\noindent}}
\newcommand{\red}[1]{{\color{BrickRed} #1}}
\newcommand{\blue}[1]{{\color{MidnightBlue}#1}}
\newcommand{\green}[1]{{\color{OliveGreen}#1}}
\newcommand{\mwrong}[2]{\red{\cancel{#1}}\green{#2}}
\newcommand{\wrong}[2]{\red{\sout{#1}}\green{#2}}
\definecolor{OliveGreen}{rgb}{.3,.5,.2}
\definecolor{MidnightBlue}{rgb}{.3,.4,.6}
\newcommand{\pts}[1]{\hfill\blue{{#1}/5}}

\chead{MATH 265F}
\pagestyle{fancy}
%Modify these items:
\rhead{\emph{Your Full Name Here}}
\lhead{\emph{HW \#5 --- 2/28/24}}

\begin{document}

\thispagestyle{fancy}
\section*{Chapter 4}
\spacing{2}
\begin{exercise}[14] If $n\in\ZZ$, then $5n^{2}+3n+7$ is odd. (Try cases.)
\begin{proof}
Suppose $n \in \ZZ$. Then $n$ must be either an even or odd integer. 
\\ \underline{Case 1:} Lets suppose that $n$ is an even integer. Then by the definition of an even integer, $n$ can be expressed as $n = 2k$, where $k \in \ZZ$. Therefore $5n^2 + 3n + 7 = 5(2k)^2 + 3(2k) + 7 = 20k^2 + 6k + 7 =2(10k^2 + 3k + 3) + 1 = 2m + 1$, where $m = 10k^2 + 3k + 3$. Note that $m$ is an integer because of the closure properties of the integers. Since $5n^{2}+3n+7 = 2m + 1$, then $5n^{2}+3n+7$ an odd integer by the definition of odd. Thus when $n$ is even, then $5n^{2}+3n+7$ is odd.
\\ \underline{Case 2:} Suppose that $n$ is an odd integer. Then by the definition of an odd integer, $n$ can be expressed as $n = 2k + 1$, where $k \in \ZZ$. Therefore $5n^2 + 3n + 7 = 5(2k+1)^2 + 3(2k+1) + 7 = 5(4k^2 +4k + 1)+ 6k+ 3 + 7 = 20k^2 + 20k + 5 + 6k + 3 + 7 = 20k^2 + 26k + 15 = 2(10k^2 + 13k + 7) + 1 = 2m + 1$, where m = $10k^2 + 13k +7$ and likewise $m \in \ZZ$. Since $5n^2 + 3n + 7 = 2m +1$, then $5n^2 + 3n + 7$ is odd by definition. Thus when $n$ is odd, then $5n^2 + 3n + 7$ is odd. 

In each case $5n^2 + 3n + 7$ is odd, satisfying all possible integer values for $n$.
\end{proof}
\end{exercise}

\begin{exercise}[16] If two integers have the same parity, then their sum is even. (Try cases.)
\begin{proof}
Suppose we have $x,y \in \ZZ$ such that they share the same parity, that is to say either $x$ and $y$ are both even or $x$ and $y$ are both odd. \\
\underline{Case 1:} Suppose $x$ is even and $y$ is even, then they can be express as $x = 2p$ and $y = 2q$ for some $p,q \in \ZZ$. Therefore $x + y = (2p) + (2q) = 2p + 2q = 2(p+q) = 2n$, where $n = p + q$ and $n \in \ZZ$ because of the closure properties of addition under the integers. Because $x + y = 2n$, that makes $x + y$ even by definition whenever $x$ and $y$ are even. \\
\underline{Case 2:} Suppose $x$ is odd and $y$ is odd, then $x = 2p + 1$ and $y = 2q + 1$ for some $p,q \in \ZZ$.Therefore $x + y = (2p + 1) + (2q + 1) = 2p + 1 + 2q + 1 = 2p + 2q + 2 = 2(p + q + 1) = 2n$, where $n = p + q + 1$ and $n \in \ZZ$ because of the closure properties of addition under the integers. Because $x + y = 2n$, our sum $x + y$ is even by definition. 

Thus for all cases in which two integers have the same parity, where either both integers are odd or both integers are even, we observe that their sum is even.
\end{proof}
\end{exercise}

\begin{exercise}[18] Suppose $x$ and $y$ are positive real numbers. If $x<y$, then $x^{2}<y^{2}$.
\begin{proof}
Suppose $x,y \in \RR+$ and that $x < y$. Multiplying both sides of the inequality by $x$ will reveal that $x^2 < yx$. Likewise when we multiply both sides of the inequality by $y$, we reveal that $xy < y^2$. Note that $xy = yx$ because of the commutative properties of multiplication. Combining our inequality results in $x^2 < xy < y^2$. Hence by the transitivity propety under inequalities in the real numbers, $x^2 < y^2$. Thus for all positive real numbers, if $x < y$ then $x^2 < y^2$.
\end{proof}
\end{exercise}

\begin{exercise}[20] If $a$ is an integer and $a^{2}\mid a$, then $a\in\{-1,0,1\}$.
\begin{proof}
	Suppose that $a \in \ZZ$ such that $a^{2}\mid a$. Then by definition of divisibility, there exists a $b \in \ZZ$ such that $a = a^2$. In order to show that $a$ is in the set of $\{-1,0,1\}$, it suffices to show that there exists such a $b$ for each value of $a$ such that $a = a^2b$ is true.
	\\ \underline{Case 1:} Suppose $a = -1$, then via substitution $a = a^2b$ gives us $-1 = (-1)^2b = 1b$, or $-1 = 1b$. If we let $b = -1$ then we find that $a^2 \mid a$ and that $a\in\{-1,0,1\}$. 
	\\ \underline{Case 2:} Suppose $a = 0$, then $a = a^2b$ holds for any value of $b$. Note that although the statement holds, the notion that $a^{2} \mid a$ for $a = 0$ is undefined.
	\\ \underline{Case 3:} Suppose $a = 1$, then $a = a^2b$ via substitution is $1 = (1)^2b = 1b$. This holds when we let $b = 1$.
\\	Thus if $a$ is an integer and $a^{2}\mid a$, then $a$ is in the set $\{-1, 0, 1\}.$
\end{proof}
\end{exercise}

\begin{exercise}[26] Every odd integer is a difference of two squares. 
\begin{proof}
Suppose $x$ is an odd integer, then by definition of odd $x = 2k + 1$ for some $k \in \ZZ$. Hence $x = 2k + 1 = k^2 + 2k + 1 - k^2 = (2k+1)^2 - k^2 = l^2 - k^2$, where $l = 2k+1$ and $l \in \ZZ$ due to the closure properties of the integers. Note that $l^2 - k^2$ is the difference of two squares.
\\ Thus every odd integer is a difference of two squares.
\end{proof}
\end{exercise}

\begin{exercise}[28] Let $a,b,c\in\ZZ$. Suppose $a$ and $b$ are not both  zero, and $c\ne 0$. Prove that $c\gcd(a,b)\le gcd(ca,cb)$.
\begin{proof}
	Suppose $a,b,c \in \ZZ$. Let $d = \gcd(a,b)$, then by definition $d \mid a$ and $d \mid b$. That means $a = dn$ and $b = dm$ for some $n,m \in \ZZ$. Multiplying both equations by $c$ we obseve that $ca = cdn$ and $cb = cdm$. Thus $cd \mid ca$ and $cd \mid cb$ where $cd$. Since $\gcd(ca, cb)$ is the greatest common divisor, $cd \leq \gcd(ca, cb)$. Thus $c*\gcd(a,b) \leq \gcd(ca, cb)$. Note that the inequality holds irrespective of whether $c < 0$ or $c > 0$.  
\end{proof}
\end{exercise}
\section*{Chapter 5}
\spacing{2}
\begin{exercise}[4] Suppose $a,b,c \in\ZZ$. If $a$ does not divide $bc$, then $a$ does not divide $b$.
\begin{proof}
Suppose
\end{proof}
\end{exercise}

\begin{exercise}[5] Suppose $x\in\RR$. %Note that I'm using my shortcuts for these special symbols from the ``favorite sets of numbers'' section in the header.
If $x^{2}+5x<-$ then $x<0$.
\begin{proof}
Write your answer here.
\end{proof}
\end{exercise}

\begin{exercise}[6] Suppose $x\in\RR$. If $x^{3}-x>0$ then $x>-1$.
\begin{proof}
Write your answer here.
\end{proof}
\end{exercise}

\begin{exercise}[7] Suppose $a,b\in\ZZ$. If both $ab$ and $a+b$ are even, then both $a$ and $b$ are even.
\begin{proof}
Write your answer here.
\end{proof}
\end{exercise}

\begin{exercise}[9] Suppose $n\in\ZZ$. If $3\nmid n^{2}$, then $3\nmid n$. % \nmid is a useful symbol to have around!
\begin{proof}
Write your answer here.
\end{proof}
\end{exercise}

\begin{exercise}[10] Suppose $x,y,z\in\ZZ$ and $x\ne 0$. %Note the \ne symbol!
If $x\nmid yz$, then $x\nmid y$ and $x\nmid z$.
\begin{proof}
Write your answer here.
\end{proof}
\end{exercise}

\begin{exercise}[16] Suppose $x,y\in\ZZ$. If $x+y$ is even, then $x$ and $y$ have the same parity.
\begin{proof}
Write your answer here.
\end{proof}
\end{exercise}

\begin{exercise}[18] If $a,b\in\ZZ$, then $(a+b)^{3}\equiv a^{3}+b^{3}\pmod 3$. %You might want to compare the commands \mod and \pmod to see how they behave differently. Also, if you are working with moduli that have more than one digit, you will have to write something like $\pmod{35}$ to get the desired behavior.
\begin{proof}
Write your answer here.
\end{proof}
\end{exercise}

\begin{exercise}[19] Let $a,b,c\in\ZZ$ and $n\in\NN$. If $a\equiv b\pmod n$ and $a\equiv c\pmod n$, then $c\equiv b\pmod n$.
\begin{proof}
Write your answer here.
\end{proof}
\end{exercise}

\begin{exercise}[22] Let $a\in\ZZ, n\in\NN$. If $a$ has remainder $r$ when divided by $n$, then $a\equiv r\pmod n$.
%Pay close attention to the role of the definitions in this argument!  It's very easy on this kind of problem to assume the conclusion by conflating a relatively obvious fact (the inter-relationship between these two claims) and what the claims *actually* say.
\begin{proof}
Write your answer here.
\end{proof}
\end{exercise}

\begin{exercise}[24] If $a\equiv b\pmod n$ and $c\equiv d\pmod n$, then $ac\equiv bd\pmod n$.
\begin{proof}
Write your answer here.
\end{proof}
\end{exercise}


\begin{exercise}[25] Let $n\in\NN$. If $2^{n}-1$ is prime, then $n$ is prime.
\begin{proof}
Write your answer here.
\end{proof}
\end{exercise}

\begin{exercise}[32] If $a\equiv b\pmod n$, then $a$ and $b$ have the same remainder when divided by $n$.
%Again, pay close attention to the roles of the definitions.
\begin{proof}
Write your answer here.
\end{proof}
\end{exercise}





 \end{document} 
