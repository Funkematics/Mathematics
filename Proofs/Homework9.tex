% !TEX TS-program = pdflatexmk
\documentclass[12pt]{amsart}

%\usepackage[parfill]{parskip}    % Activate to begin paragraphs with an empty line rather than an indent

\usepackage[margin=1in]{geometry}

\usepackage{amsmath,amssymb,amsthm,latexsym,graphicx}
\usepackage[normalem]{ulem}
\usepackage{setspace} %used for doublespacing, etc.
\usepackage{hyperref}
\usepackage{cancel}
\usepackage[dvipsnames,usenames]{color}
\usepackage[all]{xy}
\usepackage{fancyhdr}
\pagestyle{fancy}
	\renewcommand{\headrulewidth}{0.5pt} % and the line
	\headsep=1cm
	
\DeclareGraphicsRule{.tif}{png}{.png}{`convert #1 `dirname #1`/`basename #1 .tif`.png}

%Some useful environments.
\newtheorem{theorem}{Theorem}
\newtheorem{corollary}[theorem]{Corollary}
\newtheorem{conjecture}[theorem]{Conjecture}
\newtheorem{lemma}[theorem]{Lemma}
\newtheorem{proposition}[theorem]{Proposition}
\newtheorem{definition}[theorem]{Definition}
\newtheorem{example}[theorem]{Example}
\newtheorem{axiom}{Axiom}
\theoremstyle{remark}
\newtheorem{remark}{Remark}
\newtheorem*{exercise}{Exercise}%[section]

%Some shortcuts helpful for our assignments
\newcommand{\bx}{\begin{exercise}}
\newcommand{\ex}{\end{exercise}}

%Some useful shortcuts for our favorite sets of numbers.
%Note, you can use these WITHOUT entering math mode
\def\RR{\ensuremath{\mathbb R}} 
\def\NN{\ensuremath{\mathbb N}}
\def\ZZ{\ensuremath{\mathbb Z}}
\def\QQ{{\ensuremath\mathbb Q}}
\def\CC{\ensuremath{\mathbb C}}
\def\EE{{\ensuremath\mathbb E}}

%Some useful shortcuts for formatting lists
\newcommand{\bc}{\begin{center}}
\newcommand{\ec}{\end{center}}
\newcommand{\be}{\begin{enumerate}}
\newcommand{\ee}{\end{enumerate}}
\newcommand{\bi}{\begin{itemize}}
\newcommand{\ei}{\end{itemize}}

%Some useful shortcuts for formatting mathematical symbols
\newcommand{\ol}[1]{\overline{#1}}
\newcommand{\oimp}[1]{\overset{#1}{\iff}} %labeled iff symbol
\newcommand{\bv}[1]{\ensuremath{ \vec{\mathbf{#1}}} } %makes a vector.
\newcommand{\mc}[1]{\ensuremath{\mathcal{#1}}} %put something in caligraphic font
\newcommand{\normale}{\trianglelefteq}
\newcommand{\normal}{\triangleleft}

%Code for formatting the proofs a little nicer for submitted homework
\makeatletter
\renewenvironment{proof}[1][\proofname]{\par\doublespacing
  \pushQED{\qed}%
  \normalfont \topsep6\p@\@plus6\p@\relax
  \list{}{%
    \settowidth{\leftmargin}{\itshape\proofname:\hskip\labelsep}%
    \setlength{\labelwidth}{0pt}%
    \setlength{\itemindent}{-\leftmargin}%
  }%
  \item[\hskip\labelsep\itshape#1\@addpunct{:}]\ignorespaces
}{%
  \popQED\endlist\@endpefalse
  \singlespacing
}
\makeatother


%Commenting tools for the professor
\newcommand{\mpg}[1]{\marginpar{ #1}} %to put comments in margins
\usepackage{soul}
\definecolor{highlight}{rgb}{1,0.6,0.6}
\sethlcolor{highlight}
\newcommand{\hlm}[1]{\colorbox{highlight}{$\displaystyle #1$}}
\newtheoremstyle{mycomment}{\topsep}{-0in}{\small \itshape \sffamily}{}{\small \itshape\sffamily}{:}{.5em}{}
\theoremstyle{mycomment}
\newtheorem*{acomment}{\color{BrickRed}{Comment}}
\newcommand{\com}[1]{{\color{OliveGreen}\begin{acomment}{#1} %#2 \color{black} 
\end{acomment}\noindent}}
\newcommand{\red}[1]{{\color{BrickRed} #1}}
\newcommand{\blue}[1]{{\color{MidnightBlue}#1}}
\newcommand{\green}[1]{{\color{OliveGreen}#1}}
\newcommand{\mwrong}[2]{\red{\cancel{#1}}\green{#2}}
\newcommand{\wrong}[2]{\red{\sout{#1}}\green{#2}}
\definecolor{OliveGreen}{rgb}{.3,.5,.2}
\definecolor{MidnightBlue}{rgb}{.3,.4,.6}
\newcommand{\pts}[1]{\hfill\blue{{#1}/5}}

\chead{MATH 265F}
\pagestyle{fancy}
%Modify these items:
\rhead{\emph{Your Full Name Here}}
\lhead{\emph{HW \#9}}

\begin{document}

\thispagestyle{fancy}

\section*{Chapter 8} Prove the following statements.
%§8 16, 22, 26.
\begin{exercise}[16] If $A,B$ and $C$ are sets, then $A\times (B\cup C)=(A\times B)\cup (A\times C)$.
\begin{proof}
\end{proof}
\end{exercise}
\begin{exercise}[22] Let $A$ and $B$ be sets. Prove that $A\subseteq B$ if and only if $A\cap B=A$.
\begin{proof}%[Solution]
\end{proof}
\end{exercise}

\begin{exercise}[26] Prove that $\{4k+5:k\in\ZZ\}=\{4k+1:k\in\ZZ\}$.
\begin{proof}%[Solution]
\end{proof}
\end{exercise}
\section*{Chapter 9}

Each of the following statements is either true or false. If a statement is true, prove
it. If a statement is false, disprove it. 
%§9 3, 5, 8, 9, 12, 30, 34.
\begin{exercise}[3] If $n\in \ZZ$ and $n^{5}-n$ is even, then $n$ is even.
\begin{proof}%[Solution]
\end{proof}
\end{exercise}

\begin{exercise}[5] If $A, B, C$ and $D$ are sets, then $(A\times B)\cup(C\times D)=(A\cup C)\times (B\cup D)$.
\begin{proof}%[Solution]
\end{proof}
\end{exercise}

\begin{exercise}[8] If $A, B$ and $C$ are sets, and $A-(B\cup C)=(A-B)\cup (A-C)$.
\begin{proof}%[Solution]
\end{proof}
\end{exercise}

\begin{exercise}[9] If $A$ and $B$ are sets, then $\mc P(A)-\mc P(B)\subseteq \mc P(A\setminus B)$.
\begin{proof}%[Solution]
\end{proof}
\end{exercise}

\begin{exercise}[12] If $a,b,c\in\NN$ and $ab, bc$ and $ac$ all have the same parity, then $a,b$ and $c$ all have the same parity.
\begin{proof}%[Solution]
\end{proof}
\end{exercise}

\begin{exercise}[30] There exist integers $a$ and $b$ for which $42a+7b=1$.
\begin{proof}%[Solution]
\end{proof}
\end{exercise}

\begin{exercise}[34] If $X\subseteq A\cup B$, then $X\subseteq A$ or $X\subseteq B$.
\begin{proof}%[Solution]
\end{proof}
\end{exercise}



\begin{exercise}[Reflection Problem]
Answer the following questions:


\begin{proof} \ 


\begin{itemize}
\item How long did it take you to complete each problem? 

Write your answer here.
\item What was easy?

Write your answer here.
\item What was challenging? What made it challenging?

Write your answer here.
\item Compare your answers to the odd numbered exercises to those in the back of the textbook. What did you learn from this comparison?

Write your answer here.
\end{itemize}\end{proof}
\end{exercise}









 \end{document} 