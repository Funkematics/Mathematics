% !TEX TS-program = pdflatexmk
\documentclass[12pt]{amsart}

%\usepackage[parfill]{parskip}    % Activate to begin paragraphs with an empty line rather than an indent

\usepackage[margin=1in]{geometry}

\usepackage{amsmath,amssymb,amsthm,latexsym,graphicx}
\usepackage[normalem]{ulem}
\usepackage{setspace} %used for doublespacing, etc.
\usepackage{hyperref}
\usepackage{cancel}
\usepackage[dvipsnames,usenames]{color}
\usepackage[all]{xy}
\usepackage{fancyhdr}
\pagestyle{fancy}
	\renewcommand{\headrulewidth}{0.5pt} % and the line
	\headsep=1cm
	
\DeclareGraphicsRule{.tif}{png}{.png}{`convert #1 `dirname #1`/`basename #1 .tif`.png}

%Some useful environments.
\newtheorem{theorem}{Theorem}
\newtheorem{corollary}[theorem]{Corollary}
\newtheorem{conjecture}[theorem]{Conjecture}
\newtheorem{lemma}[theorem]{Lemma}
\newtheorem{proposition}[theorem]{Proposition}
\newtheorem{definition}[theorem]{Definition}
\newtheorem{example}[theorem]{Example}
\newtheorem{axiom}{Axiom}
\theoremstyle{remark}
\newtheorem{remark}{Remark}
\newtheorem*{exercise}{Exercise}%[section]

%Some shortcuts helpful for our assignments
\newcommand{\bx}{\begin{exercise}}
\newcommand{\ex}{\end{exercise}}

%Some useful shortcuts for our favorite sets of numbers.
%Note, you can use these WITHOUT entering math mode
\def\RR{\ensuremath{\mathbb R}} 
\def\NN{\ensuremath{\mathbb N}}
\def\ZZ{\ensuremath{\mathbb Z}}
\def\QQ{{\ensuremath\mathbb Q}}
\def\CC{\ensuremath{\mathbb C}}
\def\EE{{\ensuremath\mathbb E}}

%Some useful shortcuts for formatting lists
\newcommand{\bc}{\begin{center}}
\newcommand{\ec}{\end{center}}
\newcommand{\be}{\begin{enumerate}}
\newcommand{\ee}{\end{enumerate}}
\newcommand{\bi}{\begin{itemize}}
\newcommand{\ei}{\end{itemize}}

%Some useful shortcuts for formatting mathematical symbols
\newcommand{\ol}[1]{\overline{#1}}
\newcommand{\oimp}[1]{\overset{#1}{\iff}} %labeled iff symbol
\newcommand{\bv}[1]{\ensuremath{ \vec{\mathbf{#1}}} } %makes a vector.
\newcommand{\mc}[1]{\ensuremath{\mathcal{#1}}} %put something in caligraphic font
\newcommand{\normale}{\trianglelefteq}
\newcommand{\normal}{\triangleleft}

%Code for formatting the proofs a little nicer for submitted homework
\makeatletter
\renewenvironment{proof}[1][\proofname]{\par\doublespacing
  \pushQED{\qed}%
  \normalfont \topsep6\p@\@plus6\p@\relax
  \list{}{%
    \settowidth{\leftmargin}{\itshape\proofname:\hskip\labelsep}%
    \setlength{\labelwidth}{0pt}%
    \setlength{\itemindent}{-\leftmargin}%
  }%
  \item[\hskip\labelsep\itshape#1\@addpunct{:}]\ignorespaces
}{%
  \popQED\endlist\@endpefalse
  \singlespacing
}
\makeatother


%Commenting tools for the professor
\newcommand{\mpg}[1]{\marginpar{ #1}} %to put comments in margins
\usepackage{soul}
\definecolor{highlight}{rgb}{1,0.6,0.6}
\sethlcolor{highlight}
\newcommand{\hlm}[1]{\colorbox{highlight}{$\displaystyle #1$}}
\newtheoremstyle{mycomment}{\topsep}{-0in}{\small \itshape \sffamily}{}{\small \itshape\sffamily}{:}{.5em}{}
\theoremstyle{mycomment}
\newtheorem*{acomment}{\color{BrickRed}{Comment}}
\newcommand{\com}[1]{{\color{OliveGreen}\begin{acomment}{#1} %#2 \color{black} 
\end{acomment}\noindent}}
\newcommand{\red}[1]{{\color{BrickRed} #1}}
\newcommand{\blue}[1]{{\color{MidnightBlue}#1}}
\newcommand{\green}[1]{{\color{OliveGreen}#1}}
\newcommand{\mwrong}[2]{\red{\cancel{#1}}\green{#2}}
\newcommand{\wrong}[2]{\red{\sout{#1}}\green{#2}}
\definecolor{OliveGreen}{rgb}{.3,.5,.2}
\definecolor{MidnightBlue}{rgb}{.3,.4,.6}
\newcommand{\pts}[1]{\hfill\blue{{#1}/5}}

\chead{MATH 265F}
\pagestyle{fancy}
%Modify these items:
\rhead{\emph{Christopher Munoz}}
\lhead{\emph{HW \#9}}

\begin{document}

\thispagestyle{fancy}

\section*{Chapter 8} Prove the following statements.
%§8 16, 22, 26.
\begin{exercise}[16] If $A,B$ and $C$ are sets, then $A\times (B\cup C)=(A\times B)\cup (A\times C)$.
\begin{proof} Observe the following sequence of equalities:
  \begin{align*}
    A \times (B \cup C) & = \{(x,y):(x \in A) \land (y \in B \cup C)\} &&& (\text{def. of } \times)  \\
                        & = \{(x,y):(x \in A) \land (y \in B) \lor (y \in C)\} &&& (\text{def. of } \cup)\\
                        & = \{(x,y):(x \in A) \land (x \in A) \land ( y \in B) \lor (y \in C)\} &&& (A = A \land A) \\
                        & = \{(x,y):(x \in A) \land (y \in B) \lor (x \in A) \land (y \in C)\} &&&  (\text{distrib, law for sets})\\ %distrib law for sets
                        & = \{(x,y):(x \in A) \land (y  \in B)\} \cup \{(x,y): (x \in A) \land (y \in C)\} &&& (\text{def. of } \cup)\\
                        & = (A \times B) \cup (A \times C) &&& (\text{def. of } \times)
  \end{align*}
  Thus completes the proof.
\end{proof}
\end{exercise}
\begin{exercise}[22] Let $A$ and $B$ be sets. Prove that $A\subseteq B$ if and only if $A\cap B=A$.
\begin{proof}%[Solution]
Suppose $A \subseteq B$. Then by definition, for an arbitrary $x \in A$, then $x \in B$. Since $x \in A$ and $x \in B$ then by definition of the intersection of sets, $x \in A \cap B$. Given that $x \in A \cap B$ and $x \in A$, it follows that $A \cap B \subseteq A$. Furthermore $A \subseteq A \cap B$ since all elements $A$ are in $A \cap B$ as $B$ is a superset of $A$. Thus if $A \subseteq B$ then $A \cap B = A$. \\
Conversely if we suppose $A \cap B = A$, then there exists $x \in A$ and $x \in B$ such that all elements of $A$ are in $B$. Thus $ A \subseteq B$. 
\end{proof}
\end{exercise}

\begin{exercise}[26] Prove that $\{4k+5:k\in\ZZ\}=\{4k+1:k\in\ZZ\}$.
\begin{proof}%[Solution]
Suppose $x \in \{4k + 5: k \in \ZZ\}$. Then $x = 4a + 5$ for some $a \in \ZZ$. From this we get $x = 4(a + 1) + 1$. So $x = 4k + 1$ where $k = (a + 1)$ and $k \in \ZZ$ by closure properties of the integers. Hence $x \in \{4k+1:k\in\ZZ\}$. Subseqentially this means $\{4k + 5:k \in \ZZ\} \subseteq \{4k+1:k\in\ZZ\}$ \\
Conversely, suppose $x \in \{4k+1:k\in\ZZ\}$. Then $x = 4a + 1$ for some $a \in \ZZ$. If we let $a = b+1$, where $b \in \ZZ$, then we get $x = 4(b+1) + 1 = 4b + 5$. So $x \in \{4k+5:k\in\ZZ\}$. Thus $\{4k+1:k\in\ZZ\} \subseteq \{4k+5:k\in\ZZ\}$. \\
Since we established that $\{4k+5:k\in\ZZ\} \subseteq \{4k+1:k\in\ZZ\}$ and $\{4k + 1: k \in \ZZ\} \subseteq \{4k+5: k \in \ZZ \}$. By definition of equality $\{4k+5:k\in\ZZ\}=\{4k+1:k\in\ZZ\}.$
\end{proof}
\end{exercise}
\section*{Chapter 9}

Each of the following statements is either true or false. If a statement is true, prove
it. If a statement is false, disprove it. 
%§9 3, 5, 8, 9, 12, 30, 34.
\begin{exercise}[3] If $n\in \ZZ$ and $n^{5}-n$ is even, then $n$ is even.
\begin{proof}%[Solution]
  (Disproof by counterexample) Let $n = 3$, we know that $3$ is an odd number. Observe that $n^5 - n = 3^5 + 3 = 243 - 3 = 240$ which is an even number. Thus the statement is false. 
\end{proof}
\end{exercise}

\begin{exercise}[5] If $A, B, C$ and $D$ are sets, then $(A\times B)\cup(C\times D)=(A\cup C)\times (B\cup D)$.
\begin{proof}%[Solution]
  (Disproof by counterexample) Suppose $A = \{a\}, B = \{b\}, C = \{c\}$ and $D = \{d\}$. Then $(A \times B) \cup (C \times D) = \{(a,b)\} \cup \{(c,d)\} = \{(a,b),(b,c)\}$. Also $(A \cup C) \times (B \cup D) = \{(a,c)\} \times \{(b,d)\} = \{(a,b),(a,d),(c,b),(c,d)\}$, so you see that $(A\times B)\cup(C\times D) \neq (A\cup C)\times (B\cup D)$.
\end{proof}
\end{exercise}

\begin{exercise}[8] If $A, B$ and $C$ are sets, and $A-(B\cup C)=(A-B)\cup (A-C)$.
\begin{proof}%[Solution]
  (Disproof by counterexample) Let $A = \{a,b,c\}, B = \{b\}$, and $C = \{c\}$. Observe the following facts:
  \begin{align*}
    (B \cup C) & = \{b, c\} \\
    (A - B) &= \{a,b,c\} - \{b\} = \{a,c\} \\
    (A - C) &= \{a,b,c\} - \{c\} = \{a,b\} 
  \end{align*}
Thus $A - (B\cup C) = \{a,b,c\} - \{b,c\} = \{a\}$ and $(A-B) \cup (A - C) = \{a,b\} \cup \{a,c\} = \{a,b,c\}$. So you see that $A-(B\cup C)\neq(A-B)\cup (A-C)$.
\end{proof}
\end{exercise}

\begin{exercise}[9] If $A$ and $B$ are sets, then $\mc P(A)-\mc P(B)\subseteq \mc P(A - B)$.
\begin{proof}%[Solution]
  (Disproof by counterexample) Let $ A = \{a,b\}$ and $B = \{b\}$ Then $\mc P(A) - \mc P(B) = \{ \emptyset, \{a\}, \{b\}, \{a,b\} \} - \{ \emptyset \{b\}\} = \{\{a\}, \{a,b\}\} $ Also $\mc P(A-B) = \mc P(\{a\}) = \{\emptyset, \{a\}\}$. In this example we have $\mc P(A)-\mc P(B)\nsubseteq \mc P(A - B)$.
\end{proof}
\end{exercise}

\begin{exercise}[12] If $a,b,c\in\NN$ and $ab, bc$ and $ac$ all have the same parity, then $a,b$ and $c$ all have the same parity.
\begin{proof}%[Solution]
  (Direct) Suppose $ab, bc$ and $ac$ all have the same parity. We know that the product of 2 numbers will only be odd if both numbers are odd. Thus if $ab, bc$ and $ac$ are odd, then $a,b$ and $c$ must be odd. On the other hand we know that the product of an even and an odd number will be even and the product of an even and an even number is even. So for $ab, bc$ and $ac$ to share even parity, there must be at least 1 even number in each product. If we suppose $a$ is even, then $ab$ and $ac$ are even, for $bc$ to be even, $b$ or $c$ must be even. Now if we suppose that $a$ is odd then $b$ and $c$ must be even. This implies that $a,b,c$ must be even. In either case if $ab, bc$ and $ac$ share the same parity then so must $a,b$ and $c$.
\end{proof}
\end{exercise}

\begin{exercise}[30] There exist integers $a$ and $b$ for which $42a+7b=1$.
\begin{proof}%[SolutionR
  (Disproof by contradiction) Suppose for the sake of contradiction that there is some $a,b \in \ZZ$ for which $42a+7b=1$. Observe that $42a + 7b = 7(6a + b) = 1$, so $7k =1$ where $k = 6a + b$ and $k \in \ZZ$ by closure properties of the integers. Solving for $k$ in $7k = 1$ gives $k = \frac{1}{7}$, a contradiction as $k$ is an integer. This further implies that $a$ or $b$ must be a rational number to satisfy $k = \frac{1}{7} = 6a + b$ Thus there is no $a,b \in \ZZ $ that satisfies $42a + 7b = 1$.
\end{proof}
\end{exercise}

\begin{exercise}[34] If $X\subseteq A\cup B$, then $X\subseteq A$ or $X\subseteq B$.
\begin{proof}%[Solution]
  (Disproof by counterexample) Let $A = \{a\}$ and $B = \{b\}$. It follows that $\{a,b\} \subseteq A \cup B$. Note that $\{a,b\} \nsubseteq A$ and $\{a,b\} \nsubseteq B$. From this example we see that its not the case that if $X\subseteq A\cup B$, then $X\subseteq A$ or $X\subseteq B$.
\end{proof}
\end{exercise}



\begin{exercise}[Reflection Problem]
Answer the following questions:


\begin{proof} \ 


\begin{itemize}
\item How long did it take you to complete each problem? 

Write your answer here.
\item What was easy?

Write your answer here.
\item What was challenging? What made it challenging?

Write your answer here.
\item Compare your answers to the odd numbered exercises to those in the back of the textbook. What did you learn from this comparison?

Write your answer here.
\end{itemize}\end{proof}
\end{exercise}









 \end{document} 
