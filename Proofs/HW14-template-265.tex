% !TEX TS-program = pdflatexmk
\documentclass[12pt]{amsart}

%\usepackage[parfill]{parskip}    % Activate to begin paragraphs with an empty line rather than an indent

\usepackage[margin=1in]{geometry}

\usepackage{amsmath,amssymb,amsthm,latexsym,graphicx}
\usepackage[normalem]{ulem}
\usepackage{setspace} %used for doublespacing, etc.
\usepackage{hyperref}
\usepackage{cancel}
\usepackage[dvipsnames,usenames]{color}
\usepackage[all]{xy}
\usepackage{fancyhdr}
\pagestyle{fancy}
	\renewcommand{\headrulewidth}{0.5pt} % and the line
	\headsep=1cm
	
\DeclareGraphicsRule{.tif}{png}{.png}{`convert #1 `dirname #1`/`basename #1 .tif`.png}

%Some useful environments.
\newtheorem{theorem}{Theorem}
\newtheorem{corollary}[theorem]{Corollary}
\newtheorem{conjecture}[theorem]{Conjecture}
\newtheorem{lemma}[theorem]{Lemma}
\newtheorem{proposition}[theorem]{Proposition}
\newtheorem{definition}[theorem]{Definition}
\newtheorem{example}[theorem]{Example}
\newtheorem{axiom}{Axiom}
\theoremstyle{remark}
\newtheorem{remark}{Remark}
\newtheorem*{exercise}{Exercise}%[section]

%Some shortcuts helpful for our assignments
\newcommand{\bx}{\begin{exercise}}
\newcommand{\ex}{\end{exercise}}

%Some useful shortcuts for our favorite sets of numbers.
%Note, you can use these WITHOUT entering math mode
\def\RR{\ensuremath{\mathbb R}} 
\def\NN{\ensuremath{\mathbb N}}
\def\ZZ{\ensuremath{\mathbb Z}}
\def\QQ{{\ensuremath\mathbb Q}}
\def\CC{\ensuremath{\mathbb C}}
\def\EE{{\ensuremath\mathbb E}}

%Some useful shortcuts for formatting lists
\newcommand{\bc}{\begin{center}}
\newcommand{\ec}{\end{center}}
\newcommand{\be}{\begin{enumerate}}
\newcommand{\ee}{\end{enumerate}}
\newcommand{\bi}{\begin{itemize}}
\newcommand{\ei}{\end{itemize}}

%Some useful shortcuts for formatting mathematical symbols
\newcommand{\ol}[1]{\overline{#1}}
\newcommand{\oimp}[1]{\overset{#1}{\iff}} %labeled iff symbol
\newcommand{\bv}[1]{\ensuremath{ \vec{\mathbf{#1}}} } %makes a vector.
\newcommand{\mc}[1]{\ensuremath{\mathcal{#1}}} %put something in caligraphic font
\newcommand{\normale}{\trianglelefteq}
\newcommand{\normal}{\triangleleft}

%Code for formatting the proofs a little nicer for submitted homework
\makeatletter
\renewenvironment{proof}[1][\proofname]{\par\doublespacing
  \pushQED{\qed}%
  \normalfont \topsep6\p@\@plus6\p@\relax
  \list{}{%
    \settowidth{\leftmargin}{\itshape\proofname:\hskip\labelsep}%
    \setlength{\labelwidth}{0pt}%
    \setlength{\itemindent}{-\leftmargin}%
  }%
  \item[\hskip\labelsep\itshape#1\@addpunct{:}]\ignorespaces
}{%
  \popQED\endlist\@endpefalse
  \singlespacing
}
\makeatother


%Commenting tools for the professor
\newcommand{\mpg}[1]{\marginpar{ #1}} %to put comments in margins
\usepackage{soul}
\definecolor{highlight}{rgb}{1,0.6,0.6}
\sethlcolor{highlight}
\newcommand{\hlm}[1]{\colorbox{highlight}{$\displaystyle #1$}}
\newtheoremstyle{mycomment}{\topsep}{-0in}{\small \itshape \sffamily}{}{\small \itshape\sffamily}{:}{.5em}{}
\theoremstyle{mycomment}
\newtheorem*{acomment}{\color{BrickRed}{Comment}}
\newcommand{\com}[1]{{\color{OliveGreen}\begin{acomment}{#1} %#2 \color{black} 
\end{acomment}\noindent}}
\newcommand{\red}[1]{{\color{BrickRed} #1}}
\newcommand{\blue}[1]{{\color{MidnightBlue}#1}}
\newcommand{\green}[1]{{\color{OliveGreen}#1}}
\newcommand{\mwrong}[2]{\red{\cancel{#1}}\green{#2}}
\newcommand{\wrong}[2]{\red{\sout{#1}}\green{#2}}
\definecolor{OliveGreen}{rgb}{.3,.5,.2}
\definecolor{MidnightBlue}{rgb}{.3,.4,.6}
\newcommand{\pts}[1]{\hfill\blue{{#1}/5}}

\chead{MATH 265F}
\pagestyle{fancy}
%Modify these items:
\rhead{\emph{Your Full Name Here}}
\lhead{\emph{HW \#11 --- 4/10/24}}

\begin{document}

\thispagestyle{fancy}

%§14.1 1, 3, 4, 12, 13.
\section*{Section 14.1}
Show that the two given sets have equal cardinality by describing a bijection from one to the other. Describe your bijection with a formula (not as a table).
\begin{exercise}[1] $\RR$ and $(0,\infty)$.
\begin{proof}[Solution]
Write your answer here.
\end{proof}
\end{exercise}

\begin{exercise}[3] $\RR$ and $(0,1)$.
\begin{proof}[Solution]
Write your answer here.
\end{proof}
\end{exercise}

\begin{exercise}[4] The set of even integers and the set of odd integers.
\begin{proof}
Write your answer here.
\end{proof}
\end{exercise}

\begin{exercise}[12] $\NN$ and $\ZZ$ (SuggestionL: use Exercise 18 from \S12.2.)
\begin{proof}[Solution]
Write your answer here.
\end{proof}
\end{exercise}

\begin{exercise}[13] $\mc P(\NN)$ and $\mc P(\ZZ)$. (Suggestion: use Exercise 12, above.)
\begin{proof}
Write your answer here.
\end{proof}
\end{exercise}

%§14.2 1, 2, 7.
\section*{Section 14.2}
\begin{exercise}[1] Prove that the set $A=\{\ln(n):n\in \NN\}\subseteq \RR$ is countably infinite.
\begin{proof}
Write your answer here.
\end{proof}
\end{exercise}

\begin{exercise}[2] Prove that the set $A=\{(m,n)\in\NN\times\NN:m\le n\}$ is countably infinite.
\begin{proof}
Write your answer here.
\end{proof}
\end{exercise}

\begin{exercise}[7] Prove or disprove: The set $\QQ^{100}$ is countably infinite.
\begin{proof}
Write your answer here.
\end{proof}
\end{exercise}

%§14.3 1, 3, 7.
\section*{Section 14.3}
\begin{exercise}[1] Suppose $B$ is an uncountable set and $A$ is a set. Given that there is a surjective function $f:A\to B$, what can be said about the cardinality of $A$?
\begin{proof}[Solution]
Write your answer here.
\end{proof}
\end{exercise}


\begin{exercise}[3] Prove or disprove: If $A$ is uncountable, then $|A|=|\RR|$.
\begin{proof}
Write your answer here.
\end{proof}
\end{exercise}

\begin{exercise}[7] Prove or disprove: If $A\subseteq B$ and $A$ is countably infinite and $B$ is uncountable, then $B-A$ is uncountable.
\begin{proof}
Write your answer here.
\end{proof}
\end{exercise}



%§14.4 1, 2.
\section*{Section 14.4}
\begin{exercise}[1] Show that if $A\subseteq B$ and there is an injection $g:B\to A$, then $|A|=|B|$.
\begin{proof}
Write your answer here.
\end{proof}
\end{exercise}

\begin{exercise}[2] Show that $|\RR^{2}|=|\RR|$. Suggestion: Begin by showing $|(0,1)\times(0,1)|=|(0,1)|$.
 \begin{proof}
Write your answer here.
\end{proof}
\end{exercise}


%Reflection Problem
\section*{Reflection}
\begin{exercise}[Reflection Problem] \ 

\begin{proof}[Answers] \ \\
\begin{description}
\item[How long did it take you to complete each problem?]
\item[What was easy?]
\item[What was challenging? What made it challenging?]
\item[What did you learn from comparing your answers to those in the book?]
\end{description}
\end{proof}
\end{exercise}.









 \end{document} 