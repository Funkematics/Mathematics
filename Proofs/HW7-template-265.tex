% !TEX TS-program = pdflatexmk
\documentclass[12pt]{amsart}

%\usepackage[parfill]{parskip}    % Activate to begin paragraphs with an empty line rather than an indent

\usepackage[margin=1in]{geometry}

\usepackage{amsmath,amssymb,amsthm,latexsym,graphicx}
\usepackage[normalem]{ulem}
\usepackage{setspace} %used for doublespacing, etc.
\usepackage{hyperref}
\usepackage{cancel}
\usepackage[dvipsnames,usenames]{color}
\usepackage[all]{xy}
\usepackage{fancyhdr}
\pagestyle{fancy}
	\renewcommand{\headrulewidth}{0.5pt} % and the line
	\headsep=1cm
	
\DeclareGraphicsRule{.tif}{png}{.png}{`convert #1 `dirname #1`/`basename #1 .tif`.png}

%Some useful environments.
\newtheorem{theorem}{Theorem}
\newtheorem{corollary}[theorem]{Corollary}
\newtheorem{conjecture}[theorem]{Conjecture}
\newtheorem{lemma}[theorem]{Lemma}
\newtheorem{proposition}[theorem]{Proposition}
\newtheorem{definition}[theorem]{Definition}
\newtheorem{example}[theorem]{Example}
\newtheorem{axiom}{Axiom}
\theoremstyle{remark}
\newtheorem{remark}{Remark}
\newtheorem*{exercise}{Exercise}%[section]

%Some shortcuts helpful for our assignments
\newcommand{\bx}{\begin{exercise}}
\newcommand{\ex}{\end{exercise}}

%Some useful shortcuts for our favorite sets of numbers.
%Note, you can use these WITHOUT entering math mode
\def\RR{\ensuremath{\mathbb R}} 
\def\NN{\ensuremath{\mathbb N}}
\def\ZZ{\ensuremath{\mathbb Z}}
\def\QQ{{\ensuremath\mathbb Q}}
\def\CC{\ensuremath{\mathbb C}}
\def\EE{{\ensuremath\mathbb E}}

%Some useful shortcuts for formatting lists
\newcommand{\bc}{\begin{center}}
\newcommand{\ec}{\end{center}}
\newcommand{\be}{\begin{enumerate}}
\newcommand{\ee}{\end{enumerate}}
\newcommand{\bi}{\begin{itemize}}
\newcommand{\ei}{\end{itemize}}

%Some useful shortcuts for formatting mathematical symbols
\newcommand{\ol}[1]{\overline{#1}}
\newcommand{\oimp}[1]{\overset{#1}{\iff}} %labeled iff symbol
\newcommand{\bv}[1]{\ensuremath{ \vec{\mathbf{#1}}} } %makes a vector.
\newcommand{\mc}[1]{\ensuremath{\mathcal{#1}}} %put something in caligraphic font
\newcommand{\normale}{\trianglelefteq}
\newcommand{\normal}{\triangleleft}

%Code for formatting the proofs a little nicer for submitted homework
\makeatletter
\renewenvironment{proof}[1][\proofname]{\par\doublespacing
  \pushQED{\qed}%
  \normalfont \topsep6\p@\@plus6\p@\relax
  \list{}{%
    \settowidth{\leftmargin}{\itshape\proofname:\hskip\labelsep}%
    \setlength{\labelwidth}{0pt}%
    \setlength{\itemindent}{-\leftmargin}%
  }%
  \item[\hskip\labelsep\itshape#1\@addpunct{:}]\ignorespaces
}{%
  \popQED\endlist\@endpefalse
  \singlespacing
}
\makeatother

%Commenting tools for the professor
\newcommand{\mpg}[1]{\marginpar{ #1}} %to put comments in margins
\usepackage{soul}
\definecolor{highlight}{rgb}{1,0.6,0.6}
\sethlcolor{highlight}
\newcommand{\hlm}[1]{\colorbox{highlight}{$\displaystyle #1$}}
\newtheoremstyle{mycomment}{\topsep}{-0in}{\small \itshape \sffamily}{}{\small \itshape\sffamily}{:}{.5em}{}
\theoremstyle{mycomment}
\newtheorem*{acomment}{\color{BrickRed}{Comment}}
\newcommand{\com}[1]{{\color{OliveGreen}\begin{acomment}{#1} %#2 \color{black} 
\end{acomment}\noindent}}
\newcommand{\red}[1]{{\color{BrickRed} #1}}
\newcommand{\blue}[1]{{\color{MidnightBlue}#1}}
\newcommand{\green}[1]{{\color{OliveGreen}#1}}
\newcommand{\mwrong}[2]{\red{\cancel{#1}}\green{#2}}
\newcommand{\wrong}[2]{\red{\sout{#1}}\green{#2}}
\definecolor{OliveGreen}{rgb}{.3,.5,.2}
\definecolor{MidnightBlue}{rgb}{.3,.4,.6}
\newcommand{\pts}[1]{\hfill\blue{{#1}/5}}

\chead{MATH 265F}
\pagestyle{fancy}
%Modify these items:
\rhead{\emph{Your Full Name Here}}
\lhead{\emph{HW \#7}}

\begin{document}

\thispagestyle{fancy}

\section*{Chapter 6}
\begin{exercise}[2] Suppose $n\in\ZZ$. If $n^{2}$ is odd, then $n$ is odd.
\begin{proof}
Write your answer here.
\end{proof}
\end{exercise}

\begin{exercise}[3] Prove that $\sqrt[3]{2}$ is irrational.
\begin{proof}
Write your answer here.
\end{proof}
\end{exercise}

\begin{exercise}[4] Prove that $\sqrt{6}$ is irrational.
\begin{proof}
Write your answer here.
\end{proof}
\end{exercise}

\begin{exercise}[8] Suppose $a,b,c\in\ZZ$. If $a^{2}+b^{2}=c^{2}$, then $a$ or $b$ is even.
\begin{proof}
Write your answer here.
\end{proof}
\end{exercise}

\begin{exercise}[9] Suppose $a,b\in\RR$. If $a$ is rational and $ab$ is irrational, then $b$ is irrational.
\begin{proof}
Write your answer here.
\end{proof}
\end{exercise}

\begin{exercise}[11] There exist no integers $a$ and $b$ for which $18a+6b=1$.
\begin{proof}
Write your answer here.
\end{proof}
\end{exercise}

\begin{exercise}[12] For every positive $x\in\QQ$, there is a positive $y\in\QQ$ for which $y<x$.
\begin{proof}
Write your answer here.
\end{proof}
\end{exercise}

\begin{exercise}[16] If $a$ and $b$ are positive real numbers, then $a+b\ge 2\sqrt{ab}$.
\begin{proof}
Write your answer here.
\end{proof}
\end{exercise}

\begin{exercise}[19] The product of any five consecutive integers is divisible by 120. (For example, the product of 3, 4, 5, 6 and 7 is 2520, and $2520=120\cdot 21$.)
\begin{proof}
Write your answer here.
\end{proof}
\end{exercise}

\section*{Chapter 7}
\begin{exercise}[1] Suppose $x\in\ZZ$. Then $x$ is even if and only if $3x+5$ is odd.
\begin{proof}
Write your answer here.
\end{proof}
\end{exercise}
\begin{exercise}[4] Let $a$ be an integer. Then $a^{2}+4a+5$ is odd if an d only if $a$ is even.
\begin{proof}
Write your answer here.
\end{proof}
\end{exercise}
\begin{exercise}[7] Suppose $x,y\in\RR$. Then $(x+y)^{2}=x^{2}+y^{2}$ if and only if $x=0$ or $y=0$.
\begin{proof}
Write your answer here.
\end{proof}
\end{exercise}

\begin{exercise}[Reflection Problem]
\begin{proof}
Write your answer here.
\end{proof}
\end{exercise}.















 \end{document} 