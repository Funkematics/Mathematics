% !TEX TS-program = pdflatexmk
\documentclass[12pt]{amsart}

%\usepackage[parfill]{parskip}    % Activate to begin paragraphs with an empty line rather than an indent

\usepackage[margin=1in]{geometry}

\usepackage{amsmath,amssymb,amsthm,latexsym,graphicx}
\usepackage[normalem]{ulem}
\usepackage{setspace} %used for doublespacing, etc.
\usepackage{hyperref}
\usepackage{cancel}
\usepackage[dvipsnames,usenames]{color}
\usepackage[all]{xy}
\usepackage{fancyhdr}
\pagestyle{fancy}
	\renewcommand{\headrulewidth}{0.5pt} % and the line
	\headsep=1cm
	
\DeclareGraphicsRule{.tif}{png}{.png}{`convert #1 `dirname #1`/`basename #1 .tif`.png}

%Some useful environments.
\newtheorem{theorem}{Theorem}
\newtheorem{corollary}[theorem]{Corollary}
\newtheorem{conjecture}[theorem]{Conjecture}
\newtheorem{lemma}[theorem]{Lemma}
\newtheorem{proposition}[theorem]{Proposition}
\newtheorem{definition}[theorem]{Definition}
\newtheorem{example}[theorem]{Example}
\newtheorem{axiom}{Axiom}
\theoremstyle{remark}
\newtheorem{remark}{Remark}
\newtheorem*{exercise}{Exercise}%[section]

%Some shortcuts helpful for our assignments
\newcommand{\bx}{\begin{exercise}}
\newcommand{\ex}{\end{exercise}}

%Some useful shortcuts for our favorite sets of numbers.
%Note, you can use these WITHOUT entering math mode
\def\RR{\ensuremath{\mathbb R}} 
\def\NN{\ensuremath{\mathbb N}}
\def\ZZ{\ensuremath{\mathbb Z}}
\def\QQ{{\ensuremath\mathbb Q}}
\def\CC{\ensuremath{\mathbb C}}
\def\EE{{\ensuremath\mathbb E}}

%Some useful shortcuts for formatting lists
\newcommand{\bc}{\begin{center}}
\newcommand{\ec}{\end{center}}
\newcommand{\be}{\begin{enumerate}}
\newcommand{\ee}{\end{enumerate}}
\newcommand{\bi}{\begin{itemize}}
\newcommand{\ei}{\end{itemize}}

%Some useful shortcuts for formatting mathematical symbols
\newcommand{\ol}[1]{\overline{#1}}
\newcommand{\oimp}[1]{\overset{#1}{\iff}} %labeled iff symbol
\newcommand{\bv}[1]{\ensuremath{ \vec{\mathbf{#1}}} } %makes a vector.
\newcommand{\mc}[1]{\ensuremath{\mathcal{#1}}} %put something in caligraphic font
\newcommand{\normale}{\trianglelefteq}
\newcommand{\normal}{\triangleleft}

%Code for formatting the proofs a little nicer for submitted homework
\makeatletter
\renewenvironment{proof}[1][\proofname]{\par\doublespacing
  \pushQED{\qed}%
  \normalfont \topsep6\p@\@plus6\p@\relax
  \list{}{%
    \settowidth{\leftmargin}{\itshape\proofname:\hskip\labelsep}%
    \setlength{\labelwidth}{0pt}%
    \setlength{\itemindent}{-\leftmargin}%
  }%
  \item[\hskip\labelsep\itshape#1\@addpunct{:}]\ignorespaces
}{%
  \popQED\endlist\@endpefalse
  \singlespacing
}
\makeatother

%Commenting tools for the professor
\newcommand{\mpg}[1]{\marginpar{ #1}} %to put comments in margins
\usepackage{soul}
\definecolor{highlight}{rgb}{1,0.6,0.6}
\sethlcolor{highlight}
\newcommand{\hlm}[1]{\colorbox{highlight}{$\displaystyle #1$}}
\newtheoremstyle{mycomment}{\topsep}{-0in}{\small \itshape \sffamily}{}{\small \itshape\sffamily}{:}{.5em}{}
\theoremstyle{mycomment}
\newtheorem*{acomment}{\color{BrickRed}{Comment}}
\newcommand{\com}[1]{{\color{OliveGreen}\begin{acomment}{#1} %#2 \color{black} 
\end{acomment}\noindent}}
\newcommand{\red}[1]{{\color{BrickRed} #1}}
\newcommand{\blue}[1]{{\color{MidnightBlue}#1}}
\newcommand{\green}[1]{{\color{OliveGreen}#1}}
\newcommand{\mwrong}[2]{\red{\cancel{#1}}\green{#2}}
\newcommand{\wrong}[2]{\red{\sout{#1}}\green{#2}}
\definecolor{OliveGreen}{rgb}{.3,.5,.2}
\definecolor{MidnightBlue}{rgb}{.3,.4,.6}
\newcommand{\pts}[1]{\hfill\blue{{#1}/5}}

\chead{MATH 265F}
\pagestyle{fancy}
%Modify these items:
\rhead{\emph{Christopher Munoz}}
\lhead{\emph{HW \#10}}

\begin{document}

\thispagestyle{fancy}

\section*{Chapter 10} Prove the following statements with either induction, strong induction or proof by smallest counterexample.


\begin{exercise}[3] Prove that $1^{3}+2^{3}+3^{3}+4^{3}+\cdots n^{3}=\frac{n^{2}(n+1)^{2}}{4}$ for every positive integer $n$.
\begin{proof}%[Solution]
  (Weak Induction)\\ \underline{Base Case:} Observe that when $n = 1$ that $ n^3 = (1)^3 = \frac{(1)^{2}{((1)+1)^{2}}}{4} = \frac{4}{4} = 1$ which is true. \\
  \underline{Induction Hypothesis:} Suppose there is a $k \in \ZZ$ such that $1^{3}+2^{3}+3^{3}+4^{3}+\cdots k^{3}=\frac{k^{2}(k+1)^{2}}{4}$. \\
  \underline{Inductive Step:} We wish to show that the statement holds for $n = k+1$, i.e., that $1^{3}+2^{3}+3^{3}+4^{3}+\cdots k^{3} + (k + 1)^3 =\frac{(k+1)^2 ((k+1)+1)^2}{4}$. Observe the following:
  \begin{align*}
    1^{3}+2^{3}+3^{3}+4^{3}+\cdots k^{3} + (k + 1)^3 &= [1^{3}+2^{3}+3^{3}+4^{3}+\cdots k^{3}]+ (k + 1)^3 \\ 
                                                     &= \frac{k^{2}(k+1)^{2}}{4} + (k + 1)^3  \\
                                                     &= \frac{k^{2}(k+1)^{2}}{4} + \frac{4(k + 1)^3}{4} \\
                                                     &= \frac{k^{2}(k+1)^{2}+{4(k + 1)^3}}{4} \\
                                                     &= \frac{(k+1)^{2}(k^2+{4(k + 1))}}{4} \\
                                                     &= \frac{(k+1)^{2}(k^2+4k + 4))}{4} \\
                                                     &= \frac{(k+1)^2 (k+2)^2}{4} \\
                                                     &= \frac{(k+1)^2 ((k+1)+1)^2}{4}.
  \end{align*}
  Showing that the statement holds for $n = k + 1$. \\
  \underline{Conclusion:} Therefore, by induction on $n$, the statement $1^{3}+2^{3}+3^{3}+4^{3}+\cdots n^{3}=\frac{n^{2}(n+1)^{2}}{4}$ is true for every positive integer $n \geq 1$.
\end{proof}
\end{exercise}

\begin{exercise}[4] If $n\in\NN$, then $1\cdot 2+2\cdot 3+3\cdot 4+\cdots+n(n+1)=\frac{n(n+1)(n+2)}{3}$.
\begin{proof}%[Solution]
  \underline{Base Case:} Observe that when $n = 1$ that $\left[n(n+1) = \frac{n(n+1)(n+2)}{3}\right] = \left[(1)((1)+ 1) = \frac{(1)((1)+1)((1) + 2)}{3}\right] = \left[(1)(2) = \frac{6}{3} \right] = 2$ is true. \\
  \underline{Induction Hypothesis:} Suppose for all all $k$ with $1 \leq k < n$ that
  \begin{align*}
    1(2) + 2(3) + 3(4) + \cdots + k(k+1) = \frac{k(k+1)(k+2)}{3}.
  \end{align*}
  In particular, suppose that $k = n-1$ such that
  \begin{align*}
    1(2) + 2(3) + 3(4) + \cdots + n(n-1) = \frac{(n-1)(n)(n+1)}{3}
  \end{align*}
  \underline{Induction Step:} We need to show that $1(2) + 2(3) + 3(4)+...+ n(n+1) = \frac{n(n+1)(n+2)}{3}$.  Observe that
  \begin{align*}
    1(2) + 2(3) + 3(4) + \cdots + n(n+1) &= 1(2) + 2(3) + 3(4) + \cdots + (n-1)(n) + n(n+1) \\
                                         &= \biggl(1(2) + 2(3) + 3(4) + \cdots + (n-1)n\biggr) + n(n+1) \\
                                         &= \frac{(n-1)(n)(n+1)}{3} + n(n+1) \\
                                         &= \frac{(n-1)(n)(n+1)}{3} + \frac{3n(n+1)}{3} \\
                                         &= \frac{(n-1)(n)(n+1) + 3n(n+1)}{3} \\
                                         &= \frac{(n(n+1))((n-1)+3)}{3} \\
                                         &= \frac{n(n+1)(n+2)}{3}
  \end{align*}
Therefore, by principle of mathematical induction, $1\cdot 2+2\cdot 3+3\cdot 4+\cdots+n(n+1)=\frac{n(n+1)(n+2)}{3}$ is true for all $n \in \NN$.\\
(Note, this one uses the induction extras problem as a skeleton.)
\end{proof}
\end{exercise}

\begin{exercise}[5] If $n\in\NN$, then $2^{1}+2^{2}+2^{3}+\cdots+2^{n}=2^{n+1}-2$.
\begin{proof}%[Solution]
  Let $P(n)$ be the statement $2^{1}+2^{2}+2^{3}+\cdots+2^{n}=2^{n+1}-2$. We will demonstrate that the left hand side is equal to the right hand side.\\
  \underline{Base Case:} When $n = 1$, $P(n) = 2^{(1)} = 2^{(1)+1} - 2 = 4 - 2 = 2$. So $P(1)$ holds. \\
  \underline{Induction Hypothesis:} Suppose for all $k\in\NN$ and $n = k \geq 1$ that $P(k)$ is true. That means that $2^{1}+2^{2}+2^{3}+\cdots+2^{k}=2^{k+1}-2$. We want to show that $P(k+1)$ holds, that is that $2^{1}+2^{2}+2^{3}+\cdots+2^{(k+1)}=2^{(k+1)+1}-2$. \\
  \underline{Induction Step:} Observe that when $n = k+1$ that 
  \begin{align*}
    P(n) &= 2^{1} + 2^{2}+2^{3} + \cdots + 2^{(k+1)} \\
         &= 2^{1} + 2^{2} + 2^{3} + \cdots + 2^{k} + 2^{k+1} \\
         &= \biggl(2^{1} + 2^{2} + 2^{3} + \cdots + 2^{k}\biggr) + 2^{k+1} \\
         &= 2^{k+1} - 2 + 2^{k+1} \\
         &= 2(2^{k+1}) - 2 \\
         &= 2^{k+2} - 2 \\
         &= 2^{(k+1)+1} - 2
  \end{align*}
  Thus we have $2^{1} + 2^{2} + 2^{3} + \cdots + 2^{k} + 2^{k+1} = 2^{(k+1)+1} - 2$. Hence the statement is true for $n = k+1$, by mathematical induction $P(n)$ is true for all $n \in \NN$.
\end{proof}
\end{exercise}

\begin{exercise}[8] If $n\in\NN$, then $\frac{1}{2!}+\frac{2}{3!}+\frac{3}{4!}+\cdots+\frac{n}{(n+1)!}=1-\frac{1}{(n+1)!}$.
\begin{proof}%[Solution]
  Let $P(n)$ be the statement $\frac{1}{2!}+\frac{2}{3!}+\frac{3}{4!}+\cdots+\frac{n}{(n+1)!}=1-\frac{1}{(n+1)!}$ \\
  \underline{Base Case:} Observe that when $n = 1$, that $P(n) = \frac{1}{((1)+1)!} = \frac{1}{2!} = \frac{1}{2} = 1 - \frac{1}{((1) + 1!)}$. So $P(1)$ is true. \\
  \underline{Induction Hypothesis:} Suppose that for some $n = k \geq 1$, where $k \in \NN$ that $P(k)$ is correct. That is to say $\frac{1}{2!} + \frac{2}{3!} + \frac{3}{4!} + \cdots + \frac{k}{(k+1)!} = 1 - \frac{1}{(k+1)!}$. We want to show that $P(k+1)$ holds.\\
  \underline{Inductive step:} Observe that when $n = k+1$ that \\
  \begin{align*}
    P(n) &= \frac{1}{2!} + \frac{2}{3!} + \frac{3}{4!} + \cdots + \frac{(k+1)}{((k+1)+1)!} \\
         &= \frac{1}{2!} + \frac{2}{3!} + \frac{3}{4!} + \cdots + \frac{k}{(k+1)!} + \frac{(k+1)}{((k+1)+1)!}  \\
         &= \biggl(\frac{1}{2!} + \frac{2}{3!} + \frac{3}{4!} + \cdots + \frac{k}{(k+1)!}\biggr) + \frac{(k+1)}{((k+1)+1)!}  \\
         &= 1 - \frac{1}{(k+1)!} + \frac{(k+1)}{((k+1)+1)!}  \\
         &= 1 - \frac{((k+1)+1)}{((k+1)+1)!} + \frac{(k+1)}{((k+1)+1)!} \\
         &= 1 - \frac{1}{((k+1)+1)!}
  \end{align*}
  Thus by induction we have shown $P(n) = 1 - \frac{1}{(n+1)!}$ is true for all $n \in \NN$.
\end{proof}
\end{exercise}

\begin{exercise}[10] Prove that $3\mid(5^{2n}-1)$ for every integer $n\ge 0$.
\begin{proof}
  \underline{Base Case:} \\
  \underline{Induction Hypothesis:} \\
  \underline{Inductive Step:} \\
\end{proof}
\end{exercise}

\begin{exercise}[13] Prove that $6\mid(n^{3}-n)$ for every integer $n\ge 0$.
\begin{proof}
Write your answer here.
\end{proof}
\end{exercise}

\begin{exercise}[18] Suppose $A_{1},A_{2},\ldots, A_{n}$ are sets in some universal set $U$, and $n\ge 2$. Prove that $\overline{A_{1}\cup A_{2}\cup \cdots A_{n}}=\overline{A_{1}}\cap\overline{A_{2}}\cap\cdots \cap\overline{A_{n}}$.
\begin{proof}
Write your answer here.
\end{proof}
\end{exercise}

\begin{exercise}[19] Prove that $\displaystyle \frac{1}{1}+\frac{1}{4}+\frac{1}{9}+\cdots+\frac{1}{n^{2}}\le 2-\frac{1}{n}$ for every $n\in\NN$.
\begin{proof}
Write your answer here.
\end{proof}
\end{exercise}

\begin{exercise}[22] If $n\in\NN$, then $$\left(1-\frac{1}{2}\right)\left(1-\frac{1}{4}\right)\left(1-\frac{1}{8}\right)\left(1-\frac{1}{16}\right)\cdots\left(1-\frac{1}{2^{n}}\right)\ge \frac{1}{4}+\frac{1}{2^{n+1}}.$$
\begin{proof}
Write your answer here.
\end{proof}
\end{exercise}

\begin{exercise}[25] Concerning the Fibonacci sequence, prove that $F_{1}+F_{2}+F_{3}+F_{4}+\cdots+F_{n}=F_{n+2}-1$.
\begin{proof}
Write your answer here.
\end{proof}
\end{exercise}


\begin{exercise}[30] Here $F_{n}$ is the $n$th Fibonacci number. Prove that $$F_{n}=\frac{\left(\frac{1+\sqrt{5}}{2}\right)^{n}+\left(\frac{1-\sqrt{5}}{2}\right)^{n}}{\sqrt{5}}.$$

\emph{Hint:} There are multiple ways to do this... one is to use the fact that $a^{n-1}=\frac{a^{n}}{a}$, while others involve things like the fact if $\phi=\frac{1+\sqrt{5}}{2}$, then $\phi^{2}-\phi-1=0$.
\begin{proof}
Write your answer here.
\end{proof}
\end{exercise}

\begin{exercise}[33] Suppose $n$ (infinitely long) straight lines lie on a plane in such a way that no two of the lines are parallel, and no three of the lines intersect in a single point. Show that this arrangement divides the plane into $\frac{n^{2}+n+2}{2}$ regions.
\begin{proof}
Write your answer here.
\end{proof}
\end{exercise}



\begin{exercise}[Reflection Problem] \ 
\begin{itemize}
\item How long did it take you to complete each problem? 
\begin{proof}[Answer]
\end{proof}
\item What was easy?
\begin{proof}[Answer]
\end{proof}
\item What was challenging? What made it challenging?
\begin{proof}[Answer]
\end{proof}
\item Compare your answers to the odd numbered exercises to those in the back of the textbook. What did you learn from this comparison?
\begin{proof}[Answer]
\end{proof}
\end{itemize}
\end{exercise}.















 \end{document} 
