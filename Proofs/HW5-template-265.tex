% !TEX TS-program = pdflatexmk
\documentclass[12pt]{amsart}

%\usepackage[parfill]{parskip}    % Activate to begin paragraphs with an empty line rather than an indent

\usepackage[margin=1in]{geometry}

\usepackage{amsmath,amssymb,amsthm,latexsym,graphicx}
\usepackage[normalem]{ulem}
\usepackage{setspace} %used for doublespacing, etc.
\usepackage{hyperref}
\usepackage{cancel}
\usepackage[dvipsnames,usenames]{color}
\usepackage[all]{xy}
\usepackage{fancyhdr}
\pagestyle{fancy}
	\renewcommand{\headrulewidth}{0.5pt} % and the line
	\headsep=1cm
	
\DeclareGraphicsRule{.tif}{png}{.png}{`convert #1 `dirname #1`/`basename #1 .tif`.png}

%Some useful environments.
\newtheorem{theorem}{Theorem}
\newtheorem{corollary}[theorem]{Corollary}
\newtheorem{conjecture}[theorem]{Conjecture}
\newtheorem{lemma}[theorem]{Lemma}
\newtheorem{proposition}[theorem]{Proposition}
\newtheorem{definition}[theorem]{Definition}
\newtheorem{example}[theorem]{Example}
\newtheorem{axiom}{Axiom}
\theoremstyle{remark}
\newtheorem{remark}{Remark}
\newtheorem*{exercise}{Exercise}%[section]

%Some shortcuts helpful for our assignments
\newcommand{\bx}{\begin{exercise}}
\newcommand{\ex}{\end{exercise}}

%Some useful shortcuts for our favorite fields
%Note, you can use these WITHOUT entering math mode
\def\RR{\ensuremath{\mathbb R}} 
\def\NN{\ensuremath{\mathbb N}}
\def\ZZ{\ensuremath{\mathbb Z}}
\def\QQ{{\ensuremath\mathbb Q}}
\def\CC{\ensuremath{\mathbb C}}
\def\EE{{\ensuremath\mathbb E}}

%Some useful shortcuts for formatting lists
\newcommand{\bc}{\begin{center}}
\newcommand{\ec}{\end{center}}
\newcommand{\be}{\begin{enumerate}}
\newcommand{\ee}{\end{enumerate}}
\newcommand{\bi}{\begin{itemize}}
\newcommand{\ei}{\end{itemize}}

%Some useful shortcuts for formatting mathematical symbols
\newcommand{\ol}[1]{\overline{#1}}
\newcommand{\oimp}[1]{\overset{#1}{\iff}} %labeled iff symbol
\newcommand{\bv}[1]{\ensuremath{ \vec{\mathbf{#1}}} } %makes a vector.
\newcommand{\mc}[1]{\ensuremath{\mathcal{#1}}} %put something in caligraphic font
\newcommand{\normale}{\trianglelefteq}
\newcommand{\normal}{\triangleleft}

%Commenting tools for the professor
\newcommand{\mpg}[1]{\marginpar{ #1}} %to put comments in margins
\usepackage{soul}
\definecolor{highlight}{rgb}{1,0.6,0.6}
\sethlcolor{highlight}
\newcommand{\hlm}[1]{\colorbox{highlight}{$\displaystyle #1$}}
\newtheoremstyle{mycomment}{\topsep}{-0in}{\small \itshape \sffamily}{}{\small \itshape\sffamily}{:}{.5em}{}
\theoremstyle{mycomment}
\newtheorem*{acomment}{\color{BrickRed}{Comment}}
\newcommand{\com}[1]{{\color{OliveGreen}\begin{acomment}{#1} %#2 \color{black} 
\end{acomment}\noindent}}
\newcommand{\red}[1]{{\color{BrickRed} #1}}
\newcommand{\blue}[1]{{\color{MidnightBlue}#1}}
\newcommand{\green}[1]{{\color{OliveGreen}#1}}
\newcommand{\mwrong}[2]{\red{\cancel{#1}}\green{#2}}
\newcommand{\wrong}[2]{\red{\sout{#1}}\green{#2}}
\definecolor{OliveGreen}{rgb}{.3,.5,.2}
\definecolor{MidnightBlue}{rgb}{.3,.4,.6}
\newcommand{\pts}[1]{\hfill\blue{{#1}/5}}

\chead{MATH 265F}
\pagestyle{fancy}
%Modify these items:
\rhead{\emph{Your Full Name Here}}
\lhead{\emph{HW \#5 --- 2/19/25}}

\begin{document}

\thispagestyle{fancy}

\section*{Chapter 4}
\begin{exercise}[2] If $x$ is an odd integer, then $x^{3}$ is odd.
\begin{proof}
Write your answer here.
\end{proof}
\end{exercise}

\begin{exercise}[4] Suppose $x,y\in\mathbb Z$. If $x$ and $y$ are odd, then $xy$ is odd.
%I'm showing here how you would enter the integers if you didn't have my nice shortcut at the top. In future, I'll probably just use my shortcut, so if I want the symbol for the integers I would just type \ZZ or $q\in\ZZ$.
\begin{proof}
Write your answer here.
\end{proof}
\end{exercise}


\begin{exercise}[6] Suppose $a,b,c\in\ZZ$. If $a\mid b$ and $a\mid c$, then $a\mid (b+c)$. %Notice my use of the shortcut to get the symbols for the integers. If you don't have my shortcut in the header of your file, you would have to type $\mathbb Z$.
\begin{proof}
Write your answer here.
\end{proof}
\end{exercise}

\begin{exercise}[11] Suppose $a,b,c,d\in\ZZ$. If $a\mid b$ and $c\mid d$, then $ac\mid bd$.
\begin{proof}
Write your answer here.
\end{proof}
\end{exercise}
\begin{exercise}[12] If $x\in \RR$ and $0<x<4$, then $\frac{4}{x(4-x)}\ge 1$. %You should inspect what I'm doing here to get the fraction and the inequality.
\begin{proof}
Write your answer here.
\end{proof}
\end{exercise}

\begin{exercise}[14] If $n\in\ZZ$, then $5n^{2}+3n+7$ is odd. (Try cases.)
\begin{proof}
Write your answer here.
\end{proof}
\end{exercise}

\begin{exercise}[16] If two integers have the same parity, then their sum is even. (Try cases.)
\begin{proof}
Write your answer here.
\end{proof}
\end{exercise}

\begin{exercise}[18] Suppose $x$ and $y$ are positive real numbers. If $x<y$, then $x^{2}<y^{2}$.
\begin{proof}
Write your answer here.
\end{proof}
\end{exercise}

\begin{exercise}[20] If $a$ is an integer and $a^{2}\mid a$, then $a\in\{-1,0,1\}$.
\begin{proof}
Write your answer here.
\end{proof}
\end{exercise}

\begin{exercise}[26] Every odd integer is a difference of two squares. 
\begin{proof}
Write your answer here.
\end{proof}
\end{exercise}

\begin{exercise}[28] Let $a,b,c\in\ZZ$. Suppose $a$ and $b$ are not both  zero, and $c\ne 0$. Prove that $c\gcd(a,b)\le gcd(ca,cb)$.
\begin{proof}
Write your answer here.
\end{proof}
\end{exercise}

\begin{exercise}[Reflection Problem]
Answer the following questions:


\begin{proof} \ 


\begin{itemize}
\item How long did it take you to complete each problem? 

Write your answer here.
\item What was easy?

Write your answer here.
\item What was challenging? What made it challenging?

Write your answer here.
\item Compare your answers to the odd numbered exercises to those in the back of the textbook. What did you learn from this comparison?

Write your answer here.
\end{itemize}\end{proof}
\end{exercise}


 \end{document} 